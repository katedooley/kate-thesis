\chapter{Input Optics Design and Characterization}

\section{Function of the Input Optics}
\label{sec:role}

\begin{figure}
\begin{centering}
\includegraphics{figures/InputOpticsBlock_thesis.pdf}
\caption[Block diagram of the Input Optics subsystem.]{Block diagram
  of the Input Optics subsystem. The IO is located between the
  pre-stabilized laser and the recycling mirror and consists of four
  components: electro-optic modulator, mode cleaner, Farday isolator,
  and mode-matching telescope. The electro-optic modulator is the only
  IO component outside of the vacuum system. Diagram is not to scale.}
\label{fig:IOblock}
\end{centering}
\end{figure}

The Input Optics is one of the primary subsystems of the LIGO
interferometers. Its purpose is to deliver an aligned, spatially
pure, mode-matched beam with phase-modulation sidebands to the
power-recycled Fabry-P\'{e}rot Michelson interferometer. The IO also
prevents the backscattering of light into the laser and distributes
the control sidebands reflected from the interferometer (designated
the \emph{reflected port}) to photodiodes for sensing and controlling
the length and alignment of the interferometer. In addition, the IO
provides an intermediate level of frequency stabilization and must
have high overall optical efficiency. It must perform these functions
without limiting the strain sensitivity of the LIGO interferometer.
Finally, it must operate robustly and continuously over years of
operation. The conceptual design is found in
Ref.~\citep{Camp1996InputOutput}.

As shown in Fig.~\ref{fig:IOblock}, the Input Optics subsystem
consists of four components located between the pre-stabilized laser
and the power recycling mirror:
\begin{itemize}
\item electro-optic modulator (EOM) \vspace{-10 pt}
\item mode cleaner cavity (MC) \vspace{-10 pt}
\item Faraday isolator (FI) \vspace{-10 pt}
\item mode-matching telescope (MMT)
\end{itemize}
Each element is a common building block of many optical experiments
and not unique to LIGO. However, their roles specific to the
successful operation of interferometry for gravitational-wave
detection are of interest and demand further attention. Here, we
briefly review the purpose of each of the Input Optics components;
further details about the design requirements are in
Ref.~\citep{Camp1997Input}.




\subsection{Electro-optic Modulator} 
The Length Sensing and Control (LSC) and Angular Sensing and Control
(ASC) subsystems require phase modulation of the laser light at RF
frequencies. This modulation is produced by an EOM, generating
sidebands of the laser light which act as references against which
interferometer length and angle changes are measured
\citep{Fritschel2001Readout}. The sideband light must be either
resonant only in the recycling cavity or not resonant in the
interferometer at all. The sidebands must be offset from the carrier
by integer multiples of the mode cleaner free spectral range so that
neither MC length fluctations nor phase modulation of the sidebands
(due to phase noise of the RF oscillator) are converted to amplitude modulation.


\subsection{Mode Cleaner}
Stably aligned cavities, limited junk light, and a frequency and
amplitude stabilized laser are key features of any ultra sensitive
laser interferometer. The mode cleaner, at the heart of the IO, plays
a major role to this effect.

A three-mirror triangular ring cavity, the mode cleaner suppresses
laser output not in the fundamental TEM$_{00}$ mode, serving two major
purposes. It enables the robustness of the ASC since higher order
modes would otherwise contaminate the angular sensing signals of the
interferometer. Also, all non-TEM$_{00}$ light on the length sensing
photodiodes, including those used for the GW readout, contributes shot
noise but not signal and therefore diminishes the signal to noise
ratio. The mode cleaner is thus largely responsible for achieving an
aligned, minimally shot-noise-limited interferometer.

The mode cleaner also plays an active role in laser frequency
stabilization \citep{Zucker2002H1}. A frequency-stabilized laser is
necessary for ensuring that the signal at the anti-symmetric port is
due to arm length fluctuations rather than laser frequency
fluctuations. In principle, the two-arm geometry of LIGO facilitates
this distinction, but imbalances between the arms allow frequency
noise to couple into the gravitational wave channel. At low
frequencies ($<$~100 Hz) the average interferometer arm length drives
the mode cleaner length, which in turn adjusts the laser frequency. At
high frequencies (up to 20~kHz), the common arm length adds an
electronic offset to the MC error point, also resulting in a shift of
the laser frequency. As a result, the light transmitted through the MC
is matched to the very quiet arms.
 
The mode cleaner acts as a passive laser amplitude fluctuation
filter. Laser power fluctuations that couple to the antisymmetric port
cause noise in the GW readout.  The mode cleaner suppresses laser
amplitude noise above its pole frequency of about 4500~Hz. In
addition, the MC passively supresses beam jitter at frequencies above
10~Hz.


\subsection{Faraday Isolator}
Faraday isolators are four-port optical devices which utilize the
Faraday effect to allow for non-reciprocal polarization switching of
laser beams.  Any reflected light from the interferometer due to
impedance mismatch, mode mismatch, non-resonant sidebands, or signal
needs to be diverted to protect the laser from back propagating light,
which can introduce amplitude and phase noise.  This diversion of the
reflected light is also necessary for extracting length and angular
information about the interferometer's cavities. The Faraday isolator
accomplishes both needs.


\subsection{Mode-matching Telescope}
The lowest order mode cleaner and arm cavity spatial eigenmodes need
to be matched for maximal power buildup in the interferometer. The
mode-matching telescope is a set of three suspended concave mirrors
between the mode cleaner and interferometer that expand the beam from
a radius of 1.6~mm at the mode cleaner waist to a radius of 37~mm at
the recycling mirror as shown in Fig.~\ref{fig:ioprofile}. The MMT
should play a passive role by delivering properly shaped light to the
interferometer without introducing beam jitter or any significant
aberration that can reduce mode coupling.

\begin{figure}
\begin{centering}
\includegraphics{figures/ioprofile1W.pdf}
\caption[Beam profile through the Input Optics]{Beam profile through
  the Input Optics. The starting point is the mode cleaner waist and
  the changes in trajectory are due to the mode-matching telescope
  mirrors.}
\label{fig:ioprofile}
\end{centering}
\end{figure}
% /Users/kate/work/2010/09/23/1W/forplotting

\section{Thermal Problems in Initial LIGO}
\label{sec:problems}
The Initial LIGO interferometers were equipped with a 10~W laser, yet
operated with only 7~W input power to the interferometer due to
power-related problems with other subsystems. The EOM was located in
the 10~W beam and the other components experienced anywhere up to 7~W
power. The 7~W operational limit was not due to the failure of the
Input Optics; however, many aspects of the IO performance did degrade
with power.

One of the primary problems of the Initial LIGO Input Optics
\citep{Adhikari1998Input} was thermal deflection of the back
propagating beam due to thermally-induced refractive index gradients
in the Faraday isolator. A significant beam drift between the
interferometer's locked and unlocked states led to clipping of the
reflected beam on the photodiodes used for length and alignment
contro. Our measurements determined a deflection of approximately
100~\microrad/W in the FI.  This was mitigated at the time by the
design and implementation of an active beam steering servo on the beam
rejected by the isolator.

There were also known limits to the power the IO could sustain.
Thermal lensing in the Faraday isolator optics would start to
significantly alter the beam mode at powers greater than 10~W, leading
to a several percent reduction in mode matching to the interferometer
\citep{UFLIGOGroup2006Upgrading}.  Additionally, the absorptive FI
elements would create thermal birefringence, degrading the optical
efficiency and isolation ratio with power
\citep{Khazanov1999Investigation}.  The Initial LIGO New Focus
electro-optic modulators had an operational power limit of around
10~W. There was a high risk of damage to the crystals under the stress
of the 0.4~mm radius beam. Also, anisotropic thermal lensing with
focal lengths as severe as 3.3~m at 10~W made the EOMs unsuitable for
much higher power. Finally, the mode cleaner mirrors exhibited high
absorption (as much as 24 ppm per mirror), enough that thermal lensing
of the MC optics at Enhanced LIGO powers would induce higher order
modal frequency degeneracy and result in a power-dependent mode
mismatch into the interferometer \citep{Bullington2008Modal,
  Arain2007Note}. In fact, as input power increased from 1~W to 7~W
the mode matching decreased from 90\% to 83\%.

In addition to the thermal limitations of the Initial LIGO IO, optical
efficiency in delivering light from the laser into the interferometer
was not optimal. Of the light entering the Input Optics chain, only
60\% remained by the time it reached the power recycling
mirror. Moreover, since only 90\% at best of the light at the
recycling mirror was coupled into the arm cavity mode, room was left
for improvement in the implementation of the MMT.


\begin{sidewaysfigure}
\begin{centering}
\includegraphics[width=1.0\textwidth]{figures/iopaperIO_withHam2.pdf}
\caption[Enhanced LIGO Input Optics optical and sensing
  configuration]{Enhanced LIGO Input Optics optical and sensing
  configuration. The HAM1 (horizontal access module) vacuum chamber is
  featured in the center, with locations of all major optics
  superimposed. HAM2 is shown on the right, with its components. These
  tables are separated by 12~m. The primary beam path, beginning at
  the pre-stabilized laser and going to the power recycling mirror, is
  shown in red as a solid line, and auxiliary beams are different
  colors and dotted. The MMTs, MCs, and steering mirror (SM) are
  suspended; all other optics are fixed to the seismically isolated
  table. The laser and sensing and diagnostic photodiodes are on
  in-air tables.}
\label{fig:IOschematic}
\end{centering}
\end{sidewaysfigure}

\begin{figure}
\begin{centering}
\subfigure{\includegraphics[width=0.8\textwidth]{figures/IOpicture1.pdf}}
\subfigure{\includegraphics[width=0.8\textwidth]{figures/IOpicture2.pdf}}
\caption[Photographs of Enhanced LIGO Input
Optics \emph{in situ}]{Photographs of Enhanced LIGO Input Optics
  \emph{in situ}.} % \textcolor{blue}{fix font type and size}
\label{fig:IOpictures}
\end{centering}
\end{figure}



\section{Enhanced LIGO Input Optics Design}
\label{sec:design}
The Enhanced LIGO Input Optics design addressed the thermal effects
that compromised the performance of Initial LIGO, and accommodated up
to four times the power of Initial LIGO. Also, the design was a
prototype for handling the 165~W laser planned for Advanced
LIGO. Since the adverse thermal properties of the Initial LIGO IO
(beam drift, birefringence, and lensing) are all attributable
primarily to absorption of laser light by the optical elements, the
primary design consideration was finding optics with excellent
thermo-optical properties \citep{UFLIGOGroup2006Upgrading}. Both the
EOM and the FI were replaced for Enhanced LIGO. Only minor changes
were made to the MC and MMT. A detailed layout of the Enhanced LIGO IO
is shown in Figure \ref{fig:IOschematic}.


\subsection{Electro-optic Modulator Design}
We replaced the commercially-made New Focus 4003 resonant phase
modulator of Initial LIGO with an EOM design and construction of our
own. Both a new crystal choice and architectural design change allow
for superior performance.

The Enhanced LIGO EOM design uses a crystal of rubidium titanyl
phosphate (RTP), which has at most 1/10 the absorption coefficient at
1064 nm of the lithium niobate (LiNbO$_3$) crystal from Initial
LIGO. At 200~W the RTP should produce a thermal lens of 200 m and
higher order mode content of less than 1\%, compared to the 3.3~m lens
the LiNbO$_3$ produces at 10~W. The RTP has a minimal risk of damage,
since it has both twice the damage threshold of LiNbO$_3$ and is
subjected to a beam twice the size of that in Initial LIGO. RTP and
LiNbO$_3$ have similar electro-optic coefficients. Also, RTP's $dn/dT$
anisotropy is 50\% smaller. Table \ref{tab:EOMcrystals} compares the
properties of most interest of the two crystals.

\begin{table*}
\centering
\caption[Comparison of selected properties of the Initial and Enhanced
 LIGO EOM crystals]{Comparison of selected properties of the Initial and Enhanced
 LIGO EOM crystals, LiNbO$_3$ and RTP, respectively. RTP was
 preferred for Enhanced LIGO because of its lower absorption,
 superior thermal properties, and similar 
 electro-optic properties \citep{UFLIGOGroup2006Upgrading}.}  
\begin{tabular}{l l l l}
\hline
 & units & LiNbO$_3$ & RTP \\
\hline
damage threshold & MW/cm$^2$ & 280 & $>600$ \\
absorption coeff. at 1064 nm & cm$^{-1}$ & $< 0.005$ & $< 0.0005$ \\
electro-optic coeff. ($n_z^3 r_{33}$) & pm/V & 306 & 239 \\
$dn_y/dT$ & 10$^{-6}$/K & 5.4 & 2.79 \\
$dn_z/dT$ & 10$^{-6}$/K & 37.9 & 9.24 \\
\hline
\end{tabular}
\label{tab:EOMcrystals}
\end{table*}

We procured the RTP crystals from Raicol and packaged them into
specially designed custom built modulators. The crystal dimensions are
$4 \times 4 \times 40$ mm and their faces are wedged by $2.85^\circ$
and anti-reflection (AR) coated. The wedge serves to separate the
polarizations and prevents an etalon effect, resulting in a
suppression of amplitude modulation. Only one crystal is used in the
EOM in order to reduce the number of surface reflections. Three
separate pairs of electrodes, each with its own resonant LC circuit,
are placed across the crystal in series, producing the three required
sets of RF sidebands: 24.5~MHz, 33.3~MHZ and 61.2~MHz. A diagram is
shown in Fig. \ref{fig:EOM}. Reference
\citep{Quetschke2008ElectroOptic} contains further details about the
modulator architecture.

\begin{figure}
\begin{centering}
  \subfigure[A single RTP crystal is sandwiched between three sets of
  electrodes that apply three different modulation frequencies. The
  wedged ends of the crystal separate the polarizations of the
  light. The p-polarized light is used in the
  interferometer.]{\includegraphics{figures/EOMthesis.pdf}}
  \subfigure[A schematic for each of the three impedance matching
  circuits of the EOM. For the three sets of electrodes, each of which
  creates its own $C_{crystal}$, a capacitor is placed parallel to the
  LC circuit formed by the crystal and a hand-wound inductor.  The
  circuits provide 50~$\Omega$ input impedance on resonance and are
  housed in a separate box from the
  crystal.]{\includegraphics{figures/EOMcircuit_thesis.pdf}}
\caption[Electro-optic modulator design]{Electro-optic modulator
  design.}
\label{fig:EOM}
\end{centering}
\end{figure}

\subsection{Mode Cleaner Design}
The mode cleaner is a suspended 12.2~m long triangular ring cavity
with finesse $\mathcal{F}$=1282 (refer to Appendix~\ref{sec:MCpole}
for a measurement of the finesse) and free spectral range of
12.243~MHz. The three mirror architecture was selected over the
standard two mirror linear filter cavity because it acts as a
polarization filter and because it eliminates direct path back
propagation to the laser \citep{Raab1992Estimation}. A pick-off of the
reflected beam is naturally facilitated for use in generating control
signals. A potential downside to the three mirror design is the
introduction of astigmatism, but this effect is negligible due to the
small opening angle of the mode cleaner. 

The MC has a round-trip length of 24.5~m. The beam waist has a radius of
1.63~mm and is located between two 45$^\circ$ flat mirrors, MC1 and
MC3 (see Fig. \ref{fig:IOschematic}). A concave third mirror, MC2,
with an 18.15~m radius of curvature forms the far point of the mode
cleaner's isoceles triangle shape. The power stored in the MC is 408
times the amount coupled in, equivalent to about 2.7~kW in Initial
LIGO and at most 11~kW for Enhanced LIGO. The peak irradiances are
32~kW/cm$^2$ and 132~kW/cm$^2$ for Initial LIGO and Enhanced LIGO,
respectively. The scattering losses of each mirror due to measured
surface microroughness are 22~ppm.

The mode cleaner mirrors are 75 mm in diameter and 25 mm thick. The
substrate material is fused silica and the mirror coating is made of
alternating layers of silica and tantala. In order to reduce the
absorption of heat in these materials and therefore improve the
transmission and modal quality of the beam in the mode cleaner, we
removed particulate by drag wiping the surface of the MC mirrors with
methanol and optical tissues. The mode cleaner was otherwise identical
to that in Initial LIGO.



\subsection{Faraday Isolator Design}
The Enhanced LIGO Faraday isolator design required not only the use of
low absorption optics, but additional design choices to mitigate any
residual thermal lensing and birefringence. In additon, trade-offs
between optical throughput in the forward direction, optical isolation
in the backwards direction, and feasibility of physical access of the
return beam for signal use were considered. The result is that the
Enhanced LIGO Faraday isolator needed a completely new architecture
and new optics compared to both the Initial LIGO FI and commercially
available isolators.

Figure \ref{fig:FI} shows a schematic of the Enhanced LIGO Faraday
Isolator. It begins and ends with low absorption calcite wedge
polarizers (CWP). Between the CWPs is a thin film polarizer (TFP), a
deuterated potassium dihydrogen phosphate (DKDP) element, a half-wave
plate (HWP), and a Faraday rotator. The rotator is made of two low
absorption terbium gallium garnet (TGG) crystals sandwiching a quartz
rotator (QR) inside a 7-disc magnet with a maximum field strength of
1.16~T. The forward propagating beam upon passing through the TGG, QR,
TGG, and HWP elements is rotated by $+22.5^\circ - 67.5^\circ +
22.5^\circ + 22.5^\circ = 0^\circ$. In the reverse direction, the
rotation through HWP, TGG, QR, TGG is $-22.5^\circ + 22.5^\circ +
67.5^\circ + 22.5^\circ = 90^\circ$. The TGG crystals are
non-reciprocal devices while the QR is reciprocal.

\begin{figure}
\begin{centering}
\subfigure{\includegraphics[width=0.9\columnwidth]{figures/FI_cropped2.jpg}}
\subfigure{\includegraphics{figures/FI_thesis.pdf}}
\caption[Faraday isolator photograph and schematic.]{Faraday isolator
  photograph and schematic. The Faraday isolator preserves the
  polarization of the light in the forward-going direction and rotates
  it by 90 degrees in the reverse direction. Light from the MC enters
  from the left and exits at the right towards the interferometer. It
  is ideally p-polarized, but any s-polarization contamination is
  promptly diverted $\sim 10$ mrad by the CWP and then reflected by
  the TFP and dumped. The p-polarized reflected beam from the
  interferometer enters from the right and is rotated to s-polarized
  light which is picked-off by the TFP and sent to the Interferometer
  Sensing and Control (ISC) table. Any imperfections in the Faraday
  rotation of the interferometer return beam results in p-polarized
  light traveling backwards along the original input path.}
\label{fig:FI}
\end{centering}
\end{figure}

\subsubsection{Thermal birefringence} 
Thermal birefringence is addressed in the Faraday rotator by the use
of the two TGG crystals and one quartz rotator rather than the typical
single TGG \citep{Khazanov2000Suppression}.  In this configuration, any
thermal polarization distortions that the beam experiences while
passing through the first TGG rotator will be partially undone upon
passing through the second. 
%\textcolor{blue}{(Add sentence of further explanation.)} 
The multiple elements in the magnet required a larger magnetic field
than in Initial LIGO and a Faraday rotator housing that is 15.5~cm in
diameter by 16.1~cm long. The TGG diameter is 20~mm.

\subsubsection{Thermal lensing}  
Thermal lensing in the Faraday isolator is addressed by including
DKDP, a negative $dn/dT$ material, in the beam path. Absorption of
light in the DKDP results in a de-focusing of the beam, which
partially compensates for the thermal focusing induced by absorption
in the TGGs \citep{Mueller2002Method, Khazanov2004Compensation}.  The
optical path length (thickness) of the DKDP is chosen to slightly
over-compensate the positive thermal lens induced in the TGG crystals,
anticipating other positive thermal lenses in the system.

\subsubsection{Polarizers}  
The polarizers used (two CWPs and one TFP) each offer advantages and
disadvantages related to optical efficiency in the forward-propagating
direction, optical isolation in the reflected direction, and thermal
beam drift. The CWPs have very high extinction ratios ($>10^5$) and
high transmission ($>$ 99\%) contributing to good optical efficiency
and isolation performance. However, the angle separating the exiting
orthogonal polarizations of light is very small, on the order of 10
mrad. This requires relatively large distances to pick off the beams
needed for interferometer sensing and control. In addition, thermally
induced index of refraction gradients due to the 4.95$^{\circ}$ wedge
angle of the CWPs result in thermal drift. However, the CWPs for the
Enhanced LIGO Faraday have a measured low absorption of 0.0013
cm$^{-1}$
%(from eLIGO wiki) 
with an expected thermal lens of 60~m at 30~W and drift of less than
1.3 $\mu$rad/W \citep{UFLIGOGroup2006Upgrading}.

The advantages of the thin film polarizer over the calcite wedge
polarizer are that it exhibits negligible thermal drift when compared
with CWPs and it operates at the Brewster angle of 55$^\circ$, thus
diverting the return beam in an easily accessible way. However, the
TFP has a lower transmission than the CWP, about 96\%, and an
extinction ratio of only 10$^3$.

Thus, the combination of CWPs and a TFP combines the best of each to
provide a high extinction ratio (from the CWPs) and ease of reflected
beam extraction (from the TFP). The downsides that remain when using
both polarizers are that there is still some thermal drift from the
CWPs. Also the transmission is reduced due to the TFP and to the fact
that there are 16 surfaces from which light can scatter.

\subsubsection{Heat conduction}
\label{sec:heatconduction}
Faraday isolators operating in a vacuum environment suffer from
increased heating with respect to those operating in air. Convective
cooling at the faces of the optical components is no longer an
effective heat removal channel, so proper heat sinking is essential to
minimize thermal lensing and depolarization. It has been shown that
Faraday isolators carefully aligned in air can experience a dramatic
reduction in isolation ratio ($>$ 10-15 dB) when placed in vacuum
\citep{TheVIRGOCollaboration2008Invacuum}. The dominant cause is the
coupling of the photoelastic effect to the temperature gradient
induced by laser beam absorption. Also of importance is the
temperature dependence of the Verdet constant--different spatial parts
of the beam experience different linear polarizations in the presence
of a temperature gradient.

To improve heat conduction away from the Faraday rotator optical
components, we designed housing for the TGG and quartz crystals that
provided improved heat sinking to the Faraday rotator. We also wrapped
the TGGs with indium foil as pictured in Fig. \ref{fig:TGG} to improve
contact with the housing, and we cushioned the DKDP and the HWP with
indium wire in their aluminum holders. This has the additional effect
of avoiding the development of thermal stresses in the crystals, an
especially important consideration for the very fragile DKDP.

\begin{figure}
\begin{centering}
\includegraphics{figures/TGG_scaled.jpg}
\caption[Photo of an indium-wrapped TGG crystal]{Photo of TGG crystal
  with indium foil wrapping.}
\label{fig:TGG}
\end{centering}
\end{figure}


\subsection{Mode-matching Telescope Design}
% from May 31, 2007 elog
The mode matching into the interferometer (at Livingston) was measured
to be at best 90\% in Initial LIGO. Because of the stringent
requirements placed on the LIGO vacuum system to reduce phase noise
through scattering by residual gas, standard opto-mechanical
translators are not permitted in the vacuum; it is therefore not
possible to physically move the mode matching telescope mirrors while
operating the interferometer. Through a combination of needing to move
the MMTs in order to fit the new Faraday isolator on the in-vacuum
optics table and additional measurements and models to determine how
to improve the coupling, a new set of MMT positions was chosen for
Enhanced LIGO. Fundamental design considerations are discussed in
Ref. \citep{Delker1997Design}.



\section{Performance of the Enhanced LIGO Input Optics}
\label{sec:performance}
The most convincing figure of merit for the Input Optics performance
is that the Enhanced LIGO interferometers achieved low-noise operation
with 20 W input power without thermal issues from the
IO. Additionally, the Input Optics were operated successfully up to
the available 30 W of power.  (Instabilities with other interferometer
subsystems limited the Enhanced LIGO science run operation to 20~W.)
We present in this section detailed measurements of the Input Optics
performance during Enhanced LIGO. Specific measurements and results
presented in figures and the text come from Livingston; performance at
Hanford was similar and is included in tables summarizing the results.



\subsection{Optical efficiency}
The optical efficiency of the Enhanced LIGO Input Optics from EOM to
recycling mirror was 75\%, a marked improvement over the approximate
60\% that was measured for Initial LIGO. A substantial part of the
improvement came from the discovery and subsequent correction of a
6.5\% loss at the second of the in-vacuum steering mirrors directing
light into the MC (refer to Fig. \ref{fig:IOschematic}). A 45$^\circ$
reflecting mirror had been used for a beam with an 8$^\circ$ angle of
incidence. Losses attributable to the mode cleaner and Faraday
isolator are described in the following sections. A summary of the IO
power budget is found in Table \ref{tab:pwrbudget}.

\begin{table}
\centering
\caption[Enhanced LIGO Input Optics power budget.]{Enhanced LIGO Input
  Optics power budget. Errors are $\pm 1\%$, except for the TFP loss
  whose error is $\pm 0.1\%$. The 
  composite mode cleaner transmission is the percentage of power after the MC to
  before the MC and is the product of the MC visibility and
  transmission. Initial LIGO values,
  where known, are included in parentheses and have errors of several percent.}
\begin{tabular}{l l l}
\hline
 & Livingston & Hanford \\
\hline
Mode cleaner visibility & 92\% & 97\% \\
Mode cleaner transmission & 88\% & 90\% \\
Composite MC transmission & 81\% (72\%) & 87\% \\
Faraday transmission &       93\% (86\%) & 94\% (86\%) \\
\hspace{0.5cm} - Thin film polarizer loss & 4.0\% & 2.7\% \\ 
IO efficiency (PSL to RM) & 75\% (60\%) & 82\% \\
\hline
\end{tabular}
\label{tab:pwrbudget}
\end{table}


\subsubsection{Mode cleaner losses} 
The mode cleaner was the greatest single source of power loss in both
Initial and Enhanced LIGO. The mode cleaner visibility, defined here as
\begin{equation}
\mbox{visibility} = 1 - \frac{P_{reflected}}{P_{in}},
\end{equation}
the ratio of the amount of light coupled into the MC to the amount
impinging the mode cleaner input mirror, was 92\%. Losses are the result of higher
order mode content and mode mismatch into the MC. The visibility was
constant within 0.04\% up to 30~W input power at both sites, providing a positive
indication that thermal aberrations in the mode cleaner were
negligible. 

Of the light coupled into the mode cleaner, 88\% was transmitted,
corresponding to an average loss of 98 ppm per mirror. Based on the
mirrors' known surface micro-roughness, the scatter loss is expected
to be 22 ppm/mirror. Part of the discrepancy between expectation and
measurement was determined to come from poor AR coatings. We measured
a 1.3\% reflection from the AR coatings on MC mirrors at both
Livingston and Hanford, a transmitted power loss equivalent to 10~ppm
of intracavity loss per mirror.

Another source of MC losses is through absorption of heat by
particulates residing on the mirror's surface. We measured the
absorption with a technique that makes use of the frequency shift of
the thermally driven drumhead eigenfrequencies of the mirror substrate
\citep{Punturo2007Mirror}. The frequency shift directly correlates with
the MC absorption via the substrate's change in Young's modulus with
temperature, $dY/dT$. A finite element model (COMSOL) was used to
compute the expected frequency shift from a temperature change of the
substrate resulting from the mirror coating absorption. The
eigenfrequencies for each mirror at room temperature are 28164~Hz,
28209~Hz, and 28237~Hz, respectively.

We cycled the power into the mode cleaner between 0.9~W and 5.1~W at 3
hour intervals, allowing enough time for a thermal characteristic time
constant to be established.  At the same time, we recorded the
frequencies of the high Q drumhead mode peaks as found in the mode
cleaner frequency error signal, heterodyned down by 28~kHz. See Figure
\ref{fig:MCabsorption}. Correcting for ambient temperature
fluctuations, we find a frequency shift of 0.043, 0.043, and 0.072
Hz/W. As a result of drag-wiping the mirrors, the absorption decreased
from 18.7, 5.5 and 12.8 ppm per mirror, respectively, to 2.1, 2.0, and
3.4 ppm per mirror. The final results for both Livingston and Hanford
are shown in Table \ref{tab:MCabsorption2}.

\begin{figure}
\begin{centering}
\includegraphics[width=1.0\columnwidth]{figures/MCdrumheadFeb08_raw.pdf}
%\includegraphics[width=1.0\columnwidth]{figures/MCabsorption_labelled.pdf}
\caption[Data from the mode cleaner absorption measurement]{Data from
  the mode cleaner absorption measurement.\footnote{Credit: Valera
    Frolov} Power into the MC was cycled between 0.9~W and 5.1~W at 3
  hour intervals (bottom frame) and the change in frequency of the
  drumhead mode of each mirror was recorded (top frame). The ambient
  temperature (middle frame) was also recorded in order to correct for
  its effects.}
%\textcolor{blue}{This is Valera's plot from Feb. 9, 2008}
\label{fig:MCabsorption}
\end{centering}
\end{figure}

\begin{table}
\centering
\caption[Absorption values for the Livingston and Hanford mode
cleaner mirrors]{Absorption values for the Livingston and Hanford mode
  cleaner mirrors before (in parentheses) and after drag wiping. The precision is $\pm 10\%$.} 
%(from the March 8, 2008 elog and uses Muzammil's factor of $14/48$ correction.)
\begin{tabular}{l l l}
\hline
mirror & Livingston & Hanford\\
% mirror & before & after & before & after \\
% \hline\hline
% MC1 & 18.7 & 2.1  & 6.1  & 5.8 \\
% MC2 & 5.5  & 2.0  & 23.9 & 7.6 \\
% MC3 & 12.8 & 3.4  & 12.5 & 15.6 \\
\hline
MC1 & 2.1 ppm (18.7 ppm) & 5.8 (6.1 ppm) \\
MC2 & 2.0 ppm (5.5 ppm) & 7.6 (23.9 ppm) \\
MC3 & 3.4 ppm (12.8 ppm) & 15.6 (12.5 ppm) \\
\hline
\end{tabular}
\label{tab:MCabsorption2}
\end{table}


\subsubsection{Faraday isolator losses} 
The Faraday isolator was the second greatest source of power loss with
its transmission of 93\%. This was an improvement over the
86\% transmission of the Initial LIGO FI. The most lossy element in the
Faraday isolator was the thin film polarizer, accounting for 4\% of
total losses. The integrated losses from AR coatings and absorption in the
TGGs, CWPs, HWP, and DKDP account for the remaining 3\% of missing power. 


\subsection{Faraday Isolation Ratio}
The isolation ratio is defined as the ratio of power incident on the
Faraday in the reverse direction (the light reflected from the
interferometer) to the power transmitted in the 
reverse direction and is often quoted in decibels: isolation ratio~=~$10
\log_{10}(P_{in-reverse}/P_{out-reverse})$.  We measured the isolation ratio of the
Faraday isolator as a function of input power both in air prior to
installation and \emph{in situ} during Enhanced LIGO operation.

To measure the in-vacuum isolation ratio, we misaligned the
interferometer arms so that the input beam would be 
promptly reflected off of the $97\%$ reflective recycling mirror. This
also has the consequence that 
the Faraday isolator is subjected to twice the input
power. Our isolation monitor was a pick-off of the backwards 
transmitted beam taken immediately after transmission
through the Faraday that we sent out of a vacuum chamber
viewport. Refer to the \emph{isolation check beam} in 
Fig. \ref{fig:IOschematic}. The in air measurement was done similarly,
except in an optics lab with a reflecting mirror placed directly after
the Faraday. 

\begin{figure}
\begin{centering}
\includegraphics[width=1.0\columnwidth]{figures/FaradayIR.pdf}
\caption[Faraday isolator isolation ratio as measured in air and in
vacuum]{Faraday isolator isolation ratio as measured in air prior to
  installation and \emph{in situ} in vacuum. The isolation worsens by
  a factor of 6 upon placement of the Faraday in vacuum due to lack of
  air convection. The solid line is a linear fit to the in-vacuum
  data, indicating a degradation in isolation of 0.02 dB/W.}
\label{fig:IR}
\end{centering}
\end{figure}

Figure \ref{fig:IR} shows our isolation ratio data. Most notably, we
observe an isolation decrease of a factor of six upon placing the
Faraday isolator in vacuum, a result consistent with that reported by
Ref. \citep{TheVIRGOCollaboration2008Invacuum}. In air the isolation
ratio is a constant 34.46 $\pm$ 0.04 dB from low power up to 47~W, and
in vacuum the isolation ratio is 26.5 dB at low power. The underlying
cause is the absence of cooling by air convection. If we attribute the
loss to the TGGs, then based on the change in TGG polarization
rotation angle necessary to produce the measured isolation drop of
8~dB and the temperature dependence of the TGG's Verdet constant, we
can put an upper limit of 11~K on the crystal temperature rise from
air to vacuum. Furthermore, a degradation of 0.02~dB/W is measured in
vacuum.

\subsection{Thermal steering}
We measured the \emph{in situ} thermal angular drift of both the beam
transmitted through the mode cleaner and of the reflected beam from
the Faraday isolator with up to 25~W input power. Just as for the
isolation ratio measurement, we misaligned the interferometer arms so
that the input beam would be promptly reflected off of the recycling
mirror. The Faraday rotator was thus subjected to up to 50~W total
and the MC to 25~W. 

Pitch and yaw motion of the mode cleaner transmitted and
interferometer reflected beams were recorded using the quadrant
photodiode (QPD) on the Input Optics table and the RF alignment
detectors on the Interferometer Sensing and Control table (see
Fig. \ref{fig:IOschematic}). There are no lenses between the MC waist
and its measurement QPD, so only the path length between the two were
needed to calibrate in radians the pitch and yaw signals on the
QPD. The interferometer reflected beam, however, passes through
several lenses. Thus, ray transfer matrices and the two alignment
detectors were necessary to extract the Faraday drift
calibration. Details of the calibration method are presented in
Appendix~\ref{sec:driftcal}.

Figure \ref{fig:drift} shows the calibrated beam steering data. The
angle of the beam out of the mode cleaner does not change measurably
as a function of input power in yaw (4.7~nrad/W) and changes by only
440~nrad/W in pitch. For the Faraday isolator, we record a beam drift
originating at the center of the Faraday rotator of 1.8~\microrad/W in
yaw and 3.2~\microrad/W in pitch. Therefore, when ramping the input
power up to 30~W during a full interferometer lock, the upper limit on
the drift experienced by the reflected beam is about 100
\microrad. This is a thirty-fold reduction with respect to the Initial
LIGO Faraday isolator and represents a fifth of the beam's divergence
angle, $\theta_{div}$~=~490 \microrad.

\begin{figure}
\begin{centering}
% \includegraphics[width=1.0\columnwidth]{figures/MC_FI_drift_labelled.pdf}
\subfigure[Angular motion
  of the beam at the MC waist and FI rotator as the input power is
  stepped. The beam is double-passed through the Faraday isolator, so
  it experiences twice the input power.]{\includegraphics[width=1.0\columnwidth]{figures/forthesis_refldriftx10.pdf}}
\subfigure[Average beam angle per
  power level in the MC and FI. Linear fits to the data are also
  shown. The slopes for MC yaw, MC pitch, FI yaw, and FI pitch,
  respectively, are 0.0047, 0.44, 1.8, and 3.2 \microrad/W.]{\includegraphics[width=1.0\columnwidth]{figures/alldrift.pdf}}
\caption[Mode cleaner and Faraday isolator thermal drift data.]{Mode
  cleaner and Faraday isolator thermal drift data.}
\label{fig:drift}
\end{centering}
\end{figure}


\subsection{Thermal Lensing}
We measured the profiles of both the beam transmitted through the
mode cleaner and the reflected beam picked off by the Faraday isolator
at low ($\sim$~1~W) and high ($\sim$~25~W) input powers to assess the
degree of thermal lensing induced in the MC and FI. Again, we
misaligned the interferometer arms so that the input beam would be
promptly reflected off the recycling mirror. We picked off a fraction
of the reflected beam on the Interferometer Sensing and Control table
and of the mode cleaner transmitted beam on the Input Optics table
(refer to Fig. \ref{fig:IOschematic}), placed lenses in each of their
paths, and measured the beam diameters at several locations on either
side of the waists created by the lenses. A change in the beam waist
size or position as a function of laser power indicates the presence
of a thermal lens.

\begin{figure}
\begin{centering}
\includegraphics[width=1.0\columnwidth]{figures/MCTrans_datafit.pdf}
\caption[Profile at high and low powers of mode cleaner transmitted
beam]{Profile at high and low powers of a pick-off of the beam
  transmitted through the mode cleaner. The precision of the beam
  profiler is $\pm 5\%$. Within the error of the measurement, there
  are no obvious degradations.}
\label{fig:MC_lensing}
\end{centering}
\end{figure}

\begin{figure}
\begin{centering}
\includegraphics[width=1.0\columnwidth]{figures/REFL_datafit.pdf}
\caption[Faraday isolator thermal lensing data]{Faraday isolator
  thermal lensing data. With 25 W into the Faraday isolator
  (corresponding to 50 W in double pass), the beam has a steeper
  divergence than a pure TEM$_{00}$ beam, indicating the presence of
  higher order modes. Errors are $\pm 5.0\%$ for each data point.}
\label{fig:FI_lensing}
\end{centering}
\end{figure}

As seen in Fig. \ref{fig:MC_lensing} and \ref{fig:FI_lensing}, the
waists of the two sets of data are collocated--no thermal lens is
measured. For the Faraday isolator, the divergence of the low and high
power beams differs, indicating that the beam quality degrades with
power. The $M^2$ factor at 1~W is 1.04 indicating the beam is 
nearly perfectly a TEM$_{00}$ mode. At 25~W, $M^2$ increases to 1.19,
corresponding to increased higher order mode content. The percentage
of power in higher order modes depends strongly on the mode order and
relative phases of the modes, and thus cannot be determined from this
measurement \citep{Kwee2007Laser}.

The results for the mode cleaner data are consistent with no thermal
lensing. The high and low power beam profiles are within each
other's error bars and well below our requirements. 

% We also measured the thermal lensing of the electro-optic modulator
% prior to its installation in Enhanced LIGO by comparing beam profiles
% of a 160~W beam with and without the EOM in its path. The data for
% both cross-sections of the beam is presented in
% Fig.~\ref{fig:EOMlensing}. We observe no significant thermal lensing
% in the y-direction and a small effect in the x-direction. An upper
% limit for the thermal lens in the x-direction can be calculated to be
% greater than 4~m, which is 10 times larger than the Rayleigh range of
% the spatial mode. The mode matching degradation is therefore less
% than 1\%. Although a direct test for Advanced LIGO because of the
% power used, this measurement also serves to demonstrate the
% effectiveness of the EOM design for Enhanced LIGO powers.

% \begin{figure}
% \begin{centering}
% %\includegraphics[width=1.0\columnwidth]{figures/Picture1.png}
% \caption{EOM thermal lensing data.}
% \label{fig:EOMlensing}
% \end{centering}
% \end{figure}


\subsection{Mode-matching}
We measured the effectiveness of the mode-matching telescope by taking
the ratio of power at the reflected port when all of the
interferometer cavities are on resonance to the power in the reflected
beam when the cavities are unlocked. Since the impedance matching is
near perfect, all light at the reflected port during interferometer
lock is attributable to a mode mismatch. In Initial LIGO, anywhere
between 10\% and 17\% of the light was rejected by the cavity due to
poor, power-dependent mode matching.  After translating the
mode-matching telescope mirrors during a vacuum chamber incursion and
upgrading the other IO components, the ratio we measured was 8\%
independent of input power. The MMT succeeds at coupling 92\% of the
light into the interferometer at all times, marking both an
improvement in MMT mirror placement and success in eliminating
measurable thermal issues. Appendix~\ref{sec:MM} details of the
mode-matching measurement.


\section{Implications for Advanced LIGO}
\label{sec:aLIGO}
As with other Advanced LIGO interferometer components, Enhanced LIGO
served as a technology demonstrator for the Advanced LIGO Input
Optics, albeit at lower laser powers The performance of the Enhanced
LIGO Input Optics components, at 20~W of input power allows us to
infer their performance in Advanced LIGO.  The requirements for the
Advanced LIGO Input Optics demand are for similar performance to
Enhanced LIGO, but with almost 8 times the laser power.

%EOM 
The Enhanced LIGO electro-optic modulator showed no thermal lensing,
degraded transmission, nor damage in over 1 million hours of sustained
operation. Measurements of the thermal lensing in RTP at powers up to
160 W show a relative power loss of $< 0.4\%$, indicating that thermal
lensing should be negligible in Advanced LIGO.  Peak irradiances in
the EOM will be approximately four times that of Enhanced LIGO (a 45\%
larger beam diameter will somewhat offset the increased power).
Testing of RTP at 10 times the expected Advanced LIGO irradiance over
100~hours show no signs of damage or degraded transmission.

The mode cleaner showed no measurable change in operational state as a
function of input power.  This bodes well for the Advanced LIGO mode
cleaner.  Compared with the Enhanced LIGO mode cleaner, the Advanced
LIGO mode cleaner is designed with a lower finesse (520) than Initial
LIGO (1282).  For 150~W input power, the Advanced LIGO mode cleaner
will operate with 3 times greater stored power than Initial LIGO.  The
corresponding peak irradiance is 400~kW/m$^2$, well below the
continuous wave coating damage threshold.  Absorption in the Advanced
LIGO mode cleaner mirror optical coatings has been measured at
0.5~ppm, roughly four times less than the best mirror coating
absorption in Enhanced LIGO, so the expected thermal loading due to
coating absorption should be reduced in Advanced LIGO.  The larger
Advanced LIGO mode cleaner mirror substrates and higher input powers
result in a significantly higher contribution to bulk absorption,
roughly 20 times Enhanced LIGO, however the expected thermal lensing
leads to small change ($< 0.5 \%$) in the output mode
\citep{Arain2007Note}.

The Enhanced LIGO data obtained from the Faraday isolator allows us to
make several predictions about how it will perform in Advanced LIGO.
The measured isolation ratio decrease of 0.02~dB/W will result in a
loss of 3~dB for a 150~W power level expected for Advanced LIGO
relative to its cold state.  However, the Advanced LIGO Faraday
isolator will employ an adjustable half wave plate \emph{in situ},
which will allow for a partial restoration of the isolation ratio. The
maximum thermally induced angular steering expected is 480 \microrad
(using a drift rate of 3.2 \microrad/W), approximately equal to the
beam divergence angle. This has some implications for the Advanced
LIGO length and alignment sensing and control system, since the
reflected Faraday isolator beam is used as a sensing beam. Operation
of Advanced LIGO at high powers will likely require the use of a beam
stabilization servo to lock the position of the reflected beam on the
sensing photodiodes.  Although no measurable thermal lensing was
observed (no change in the beam waist size or position), the measured
presence of higher order modes in the FI at high powers is suggestive
of imperfect thermal lens compensation by the DKDP.  This potentially
can be reduced by a careful selection of the thickness of the DKDP to
better match the absorbed power in the TGG crystals.

\section{Summary}
\label{sec:summary}
In summary, we have presented a comprehensive investigation of the
Enhanced LIGO Input Optics, including the function, design, and
performance of the IO.  Several improvements to the design and
implementation of the Enhanced LIGO IO over the Initial LIGO IO have
lead to improved optical throughput and coupling to the main
interferometer through a substantial reduction in thermo-optical
effects in the major IO optical components, including the
electro-optic modulators, mode cleaner, and Faraday isolator.  The IO
performance in Enhanced LIGO enables us to infer its performance in
Advanced LIGO, and indicates that high power interferometry will be
possible without severe thermal effects.


% \textcolor{blue}{from Guido: summary of the improvements due to the
%   changes in the IO. More power, better shot noise sensitivity,
%   better range, better upper limits, lessons learned for aLIGO
%   (summary like, the details should be in the previous
%   subsections). Make it glorious and complain that other subsystems
%   (TCS) wasn't able to handle the power. Otherwise this becomes very
%   technical and boring.}

% \section{Radiation pressure in MC}
% \textcolor{blue}{Maybe write up April 4, 2009 notebook derivation of
%   radiation pressure length spring in MC. Also, Rana has a noise
%   budget elog entry April 3, 2009.}
