% The Editorial Office Requirements for the Table of Contents cause a
%significant problem in Latex if there is only one Appendix. The
%Appendix is no longer labeled "A" in the TOC but has the word
%"APPENDIX" placed in front of the title of the Appendix. This can be
%done without issue IF nothing needs to be numbered by LaTeX in the
%Appendix. Unfortunately, most of the time something needs to be
%numbered in that single Appendix. For this reason we have included
%the IFTHENELSE switch found in this document and at the beginning of
%AppendixA. We assume that if you have any appendices, that you have
%more than one.  So the default setting is noa = 2 (number of
%appendices = 2). Note: you don't need the actual number of appendices
%here 1 or 2 are the only relevant numbers. You just make sure to
%input the Appendices you do have in this file.  If, however, you DO
%only have one appendix change the line: \setcounter{noa}{2} to
%\setcounter{noa}{1} And comment (or delete) all of the
%input{AppendixB} commands except the first one.  Then open the
%AppendixA.tex file and continue there.

%you can add/substract individual appendices through by using the
% /include{appendix'X'} and creating/deleting the appropriate files
\appendix 
\clearpage
\newcounter{noa} % noa= no. of appendices ... set to 1 for 1 and more otherwise.
\setcounter{noa}{2} % ........................... CHANGE VALUE ONLY HERE
\ifthenelse{\value{noa} = 1}
%...................then
{}
%...................else
{\addtocontents{toc}{\protect\addvspace{10pt}\noindent{APPENDIX}\protect\hfill\par}}
%...................
\chapter{Appendix A}



\section{Overlap integrals}
A measure of the mode matching can be given by the amount of power
coupled from one mode into another. This is calculated as the square
of the overlap integral of two fields, $\psi_1$ and $\psi_2$, for a
particular z-axis (propagation direction) cross-section:
\begin{equation}
P = \left| \langle \psi \mid \psi^{\prime} \rangle \right|^2 = \left[\int_{-\infty}^\infty \psi^\star(x) \psi^{\prime}(x) \, dx \right]^2
\end{equation}
We are interested in the lowest order Hermite-Gaussian mode:
\begin{equation}
\psi(x,z) = u_0(x,z) = \left[ \frac{2}{\pi w_0^2} \right]^{1/4} \left[
  \frac{q_0}{q(z)} \right]^{1/2} \exp{\left[ \frac{-i k x^2}{2 q(z)}
  \right] }
\end{equation}
which can be rewritten as a function of $x$ and $q$ using
$q_0=i \mbox{Im}(q)$, and $w_0^2 = -2 q_0 i/k$:
\begin{equation}
u_0(x,q) = \left[ \frac{-k \mbox{Im}(q)}{\pi q^2} \right]^{1/4} \exp{\left[ \frac{-i k x^2}{2 q}
  \right] }
\label{eq:u0_xq}
\end{equation}
Eq. (\ref{eq:u0_xq}) is normalized such that $\langle u_0 \mid u_0
\rangle = 1$.

We want to know the square of the overlap integral for two fields
given by different $q$ parameters at one location $z$. First, the
overlap integral:
\begin{align}
\langle q_1 \mid q_2 \rangle &= \int_{-\infty}^\infty u_0(x, q_1) u_0(x, q_2) \, dx \\
 &= \left[ \frac{k^2 \mathrm{Im}(q^\star_1) \mathrm{Im}(q_2)}{\pi^2 q^{\star 2}_1 q^2_2} \right]^{1/4} 
 \int_{-\infty}^\infty \exp{\left[ -\left[ \frac{-i k}{2 q^\star_1} + \frac{i k}{2 q_2} \right]x^2 \right]} \, dx \\
 &= \left[ -\frac{k^2 \mathrm{Im}(q_1) \mathrm{Im}(q_2)}{\pi^2
     q^{\star 2}_1 q^2_2} \right]^{1/4} \sqrt{\frac{\pi}{\left[ \frac{-i k}{2 q^\star_1} + \frac{i k}{2 q_2}
   \right]}} \\
 &= \left[ \mathrm{Im}(q_1) \mathrm{Im}(q_2) \right]^{1/4}
 \sqrt{\frac{2}{q^\star_1 q_2 \left( 1/q_2 - 1/q^\star_1 \right)}} 
\end{align}
Then, the power is given by:
\begin{align}
\left| \langle q_1 \mid q_2 \rangle \right|^2
&= \frac{2 \sqrt{\mathrm{Im}(q_1) \mathrm{Im}(q_2)}}  
{\sqrt{q^\star_1 q_1 q^\star_2 q_2 \left[\frac{1}{q_2} - \frac{1}{q^\star_1} \right]
    \left[\frac{1}{q^\star_2} - \frac{1}{q_1} \right]}} \\
&= \frac{2 \sqrt{\mathrm{Im}(q_1) \mathrm{Im}(q_2)}}{|q_2 - q^\star_1 |} 
\label{eq:power_qs}
\end{align}
Note that Eq. (\ref{eq:power_qs}) simplifies to 1 when $q_1 = q_2$ as
expected and that this whole formulation assumes that the beams are
propagating along the same $z$-axis.




\section{Gaussian beam on a split photodetector}
\label{app:beamonQPD}
The power per area of a Gaussian beam traveling along the $z$-axis is 
\begin{equation}
p(x,y) = \frac{2 P_0}{\pi w^2} \mbox{exp}\left[\frac{-2x^2}{w^2}\right] \mbox{exp}\left[\frac{-2y^2}{w^2}\right]
\end{equation}
where $w$ is the beam radius at $z$.  This has been normalized such
that $\int_{-\infty}^\infty \int_{-\infty}^\infty p(x,y) \,dx \,dy =
P_0$. Then, for a beam displaced by $x_0$ from the center of a split
photodetector, the power on the left side is
\begin{eqnarray} 
P_{left} &=& \frac{2 P_0}{\pi w^2} \int_{-\infty}^{x_0} \mbox{exp}\left[\frac{-2x^2}{w^2}\right]  \,dx \int_{-\infty}^\infty \mbox{exp}\left[\frac{-2y^2}{w^2}\right] \,dy \\
&=& \sqrt{\frac{2}{\pi}} \frac{P_0}{w} \left[ \int_{-\infty}^{0} \mbox{exp}\left[\frac{-2x^2}{w^2}\right]  \,dx + \int_0^{x_0} \mbox{exp}\left[\frac{-2x^2}{w^2}\right] \,dy \right] \\
&=& \sqrt{\frac{2}{\pi}} \frac{P_0}{w} \left[ \frac{w}{2} \sqrt{\frac{\pi}{2}} + \frac{w}{2} \sqrt{\frac{\pi}{2}} \int_0^{\sqrt{2} x_0 / w} \mbox{exp}\left[-t^2 \right] \,dt  \right] \\
&=& \frac{P_0}{2} \left[1 + \mbox{erf}\left[ \frac{\sqrt{2} x_0}{w} \right] \right].
\end{eqnarray}
where $\mathrm{erf}(t_0) \equiv \frac{2}{\sqrt{\pi}} \int_0^{t_0}
\mbox{exp}\left[-t^2 \right] \,dt$. The power on the right side is:
\begin{equation}
P_{right} = \frac{P_0}{2} \left[1 - \mbox{erf}\left[ \frac{\sqrt{2} x_0}{w} \right] \right].
\end{equation}
We create a normalized yaw as
\begin{equation}
\mathrm{YAW} = \frac{P_{left} - P_{right}}{P_0} = \mathrm{erf}\left[ \frac{\sqrt{2} x_0}{w} \right].
\end{equation}
Using the Taylor series expansion of the error function, we have a
first order estimate for the relationship between normalized yaw and
beam displacement $x_0$ for a beam of \textcolor{blue}{diameter(??)} $w$:
\begin{equation}
\frac{x_0}{\mathrm{YAW}} \approx \frac{w}{2} \sqrt{\frac{\pi}{2}}.
\end{equation}
The same equation holds true for pitch.




\section{ABCD matrix formalism}
For the input beam model and for the Input Optics beam drift
calibrations, the ABCD matrix formalism is a useful tool to propagate
a Gaussian beam. I choose to ignore the fact that the MC beam passes
through the substrate of MC3 on its way to the Faraday. I also treat
the beam splitter as a flat mirror and ignore the presence of its
substrate. I use the thickness of the large optic substrates, $t=0.01$
m, and account for index of refraction effects when passing through
optics. Signs of radii of curvature are defined per the front face of
the optic; for example, all main LIGO optics have a positive $R$.

For a beam that strikes a flat interface and exits at a curved
interface (ie. forward-going transmission through RM ):
\begin{equation}
\left\llbracket \begin{array}{c c}
A & B \\
C & D \end{array} \right\rrbracket = 
\left\llbracket \begin{array}{c c}
1 & 0 \\
(n_2-n_1)/Rn_1 & n_2/n_1 \end{array} \right\rrbracket
\left\llbracket \begin{array}{c c}
1 & t \\
0 & 1 \end{array} \right\rrbracket
\left\llbracket \begin{array}{c c}
1 & 0 \\
0 & n_1/n_2 \end{array} \right\rrbracket
\end{equation}

For a beam that strikes a curved interface and exits at a flat
interface (ie. transmission through ETM):
\begin{equation}
\left\llbracket \begin{array}{c c}
A & B \\
C & D \end{array} \right\rrbracket = 
\left\llbracket \begin{array}{c c}
1 & 0 \\
0 & n_2/n_1 \end{array} \right\rrbracket
\left\llbracket \begin{array}{c c}
1 & t \\
0 & 1 \end{array} \right\rrbracket
\left\llbracket \begin{array}{c c}
1 & 0 \\
(n_1-n_2)/-Rn_2 & n_1/n_2 \end{array} \right\rrbracket
\end{equation}

For a beam that strikes a flat interface, travels through the
substrate, reflects off the back of a curved interface, travels
through substrate and exits at the original flat interface (ie. single
bounce off RM):
\begin{equation}
\left\llbracket \begin{array}{c c}
A & B \\
C & D \end{array} \right\rrbracket = 
\left\llbracket \begin{array}{c c}
1 & 0 \\
0 & n_2/n_1 \end{array} \right\rrbracket
\left\llbracket \begin{array}{c c}
1 & t \\
0 & 1 \end{array} \right\rrbracket
\left\llbracket \begin{array}{c c}
1 & 0 \\
2/R & 1 \end{array} \right\rrbracket
\left\llbracket \begin{array}{c c}
1 & t \\
0 & 1 \end{array} \right\rrbracket
\left\llbracket \begin{array}{c c}
1 & 0 \\
0 & n_1/n_2 \end{array} \right\rrbracket
\end{equation}

Finally, for prompt reflection off a curved interface (ie. reflection
off MMTs):
\begin{equation}
\left\llbracket \begin{array}{c c}
A & B \\
C & D \end{array} \right\rrbracket = 
\left\llbracket \begin{array}{c c}
1 & 0 \\
-\frac{2}{R} & 1 \end{array} \right\rrbracket
\end{equation}
and for propagation a distance $d$ through vacuum:
\begin{equation}
\left\llbracket \begin{array}{c c}
A & B \\
C & D \end{array} \right\rrbracket = 
\left\llbracket \begin{array}{c c}
1 & d \\
0 & 1 \end{array} \right\rrbracket.
\end{equation}

For each of these $n_1 = 1$ is the index of refraction of vacuum and
$n_2 = 1.44963$ is the index of refraction of the fused silica used
for the optics. Table \ref{tab:ROCs} shows the radii of curvature of
each of the optics for both sites.



\subsection{Beam Drift Calibration}
We use the ABCD matrix formulation to convert pitch and yaw data of
WFS3 and WFS4 into a position and angle at the Faraday isolator. The
basic relationship between beam displacement and angle at one location
to displacement and angle at another location is given by:
\begin{equation}
\left \llbracket \begin{array}{c} 
x_1 \\
x^{\prime}_1 \end{array} \right\rrbracket =
\left\llbracket \begin{array}{c c}
A & B \\
C & D \end{array} \right\rrbracket
\left \llbracket \begin{array}{c} 
x_0 \\
x^{\prime}_0 \end{array} \right\rrbracket
\label{eq:abcd}
\end{equation}

For this application, we want to relate the beam positions on the WFS,
$x_{3}$ and $x_{4}$, to the beam position and angle, $x_{FI}$ and
$x^{\prime}_{FI}$, at the Faraday isolator. Using only the top
equation of Eq. \ref{eq:abcd} since the WFS are sensitive to beam
position only and not angle, we can write a new relation
\begin{equation}
\left \llbracket \begin{array}{c} 
x_3 \\
x_4 \end{array} \right\rrbracket =
\left\llbracket \begin{array}{c c}
A_3 & B_3 \\
A_4 & B_4 \end{array} \right\rrbracket
\left \llbracket \begin{array}{c} 
x_{FI} \\
x^{\prime}_{FI} \end{array} \right\rrbracket
\end{equation}
where ${A_3, B_3}$ and ${A_4, B_4}$ are the A and B ABCD matrix
elements for the beam paths from the Faraday isolator to WFS3 and
WFS4, respectively. Taking the inverse and writing $x_3$ and $x_4$ as
a function of the pitch and yaw recorded by the WFS (see Appendix
\ref{app:beamonQPD}), the useful equation is
\begin{equation}
\left \llbracket \begin{array}{c} 
x_{FI} \\
x^{\prime}_{FI} \end{array} \right\rrbracket =
\left\llbracket \begin{array}{c c}
A_3 & B_3 \\
A_4 & B_4 \end{array} \right\rrbracket^{-1}
\left\llbracket \begin{array}{c c}
w_3 & 0 \\
0 & w_4 \end{array} \right\rrbracket
\frac{1}{2} \sqrt{\frac{\pi}{2}} 
\left \llbracket \begin{array}{c} 
DOF_3 \\
DOF_4 \end{array} \right\rrbracket
\end{equation}
where $w_3$ and $w_4$ are the radii of the beam at each WFS and
DOF can mean PIT or YAW.




\section{Mode cleaner pole}
\label{sec:MCpole}
Optical cavities act as low pass filters for intensity variations of
the light sent into them. The model for an intensity noise transfer
function of a cavity is that of a single pole:
\begin{equation}
\frac{E_{after}}{E_{before}} = \frac{1}{1 + s/s_0} = \frac{s_0}{s_0 - s}
\end{equation}
where $s$ is a complex parameter. However, we are interested in only
purely sinusoidal variations in intensity so we let $s$ be purely
imaginary, $s=i\omega$, where $\omega$ is an angular frequency.

We measured the intensity noise transfer function of the Livingston
mode cleaner upon completion of the Enhanced LIGO Input Optics
upgrade. We modulated the intensity of the laser light going into the
MC by injecting a swept-sine excitation in \texttt{L1:PSL-ISS\_EXC}
and measured the power variation of the light in two places: before
and after the mode cleaner. We used a single photodetector (PDA55) in
order to eliminate the PD response, and therefore made the measurement
twice. We ensured there was 1V DC on the PD in both locations.

\begin{figure}
\begin{centering}
\subfigure{\includegraphics{figures/MCpole_mag.pdf}}
\subfigure{\includegraphics{figures/MCpole_phase.pdf}}
\caption[Livingston mode cleaner intensity noise transfer
function]{Livingston mode cleaner intensity noise transfer
  function. Red open circles are data; solid blue line is a single
  pole fit. Relating fit parameters to the model, the pole frequency
  is $f_{MC}=4762$ Hz.}
\label{fig:mcpole}
\end{centering}
\end{figure}

Figure \ref{fig:mcpole} shows the transfer function data and the fit
(to both magnitude and phase simultaneously). The fit has a pole
frequency of $f_p=4762$ Hz. The $1/e$ ringdown time of the mode
cleaner is therefore $\tau = 1/4\pi f_{MC} = 16.7$ \micro s and the
finesse is $\mathcal{F} = \mathrm{FSR}/2f_p = 1282$.





\section{Carrier Mode Matching into the Interferometer}

The mode matching into the interferometer can be measured by comparing
the power reflected from the interferometer during a full lock to the
power reflected with a single bounce off the RM.

First, we will look at the interferometer visibility--what percentage
of power sent to the interferometer is reflected by it. 

Then, we'll
consider the sources of reflected light and deduce from the visiblity
measurements just how much of the reflected light is due to a mode
mismatch. 

For Livingston, we will see that the visibility is a very
close approximation of the mode mismatch. However, at Hanford, much of
the reflected light is due to an impedance mismatch, so further
analysis of the visibility is presented to determine the mode
matching.


\subsection{Interferometer visibility}
The visibility is a measure of how much carrier light is reflected from the
locked interferometer compared to how much carrier light is sent in. Ideally,
all of the light should be coupled into the interferometer. However,
any mode mismatch of the input beam to the cavity or any mismatch of
interferometer losses to PRM reflectivity (impedance mismatch, see
Section \ref{sec:impedance}) will
cause carrier light to be reflected. The visibility summarizes the
compound effect of all of these sources of reflected light.

To measure the visibility, we need to know only a few numbers. We must
have a measure of how much light is sent to the interferometer for
normalization purposes and we must have a measure of the DC reflected
power when the interferometer is not locked (all light is reflected
off of the RM) and when it is locked. The visibility is then given
by:
\begin{equation}
\mbox{visibility} = 1 -
\frac{P_{REFL_{locked}}}{P_{REFL_{unlocked}}} \frac{P_{IN_{unlocked}}}{P_{IN_{locked}}}
\end{equation}

We have two measures of how much light is being sent into the
interferometer and several of how much light is
reflected. \texttt{L1:IOO-MC\_PWR\_IN} is a pick off of the light on
the PSL table just before it 
enters vaccuum and \texttt{L1:IOO-MC\_TRANS\_SUM} is a pick off of the light just after
the MC. \texttt{L1:LSC-REFL\_DC} is a record of the power at the REFL
port as seen by one of the LSC REFL PDs. There are others, like WFS3
sum, WFS4 sum, and the trigger PD that we could also look at. Together, these channels tell
a story about the percentage of incoming light rejected by the interferometer.

\begin{figure}
\begin{centering}
\includegraphics{figures/llo_mm1630bigger.pdf}
\caption[End of an 8.7 W lock at Livingston on Feb. 23, 2010]{End of
  an 8.7 W lock at Livingston on Feb. 23, 2010. The mode cleaner
  re-locks at 0.5~W about 20~seconds after lock loss and then the
  power is increased to 2~W. Comparison of the minimum intereferometer
  reflected power during lock and the maximum reflected power out of
  lock provides a measure of interferometer mode matching.}
\label{fig:llomm1630}
\end{centering}
\end{figure}

An example for LLO is shown in Figure \ref{fig:llomm1630}. The lock
stretch ends at $t=5$ min. Note that the amount of reflected light is 
increasing up to the end of the lock as the interferometer is losing
stability. When lock is lost, the common mode servo kicks the mode
cleaner out of lock too, and the MC trans power drops to 0. About 15 seconds later the MC relocks and then
the power into it increases. About 15 seconds after that, at 5.5 minutes, the MC WFS turn
on, improving the alignment of the MC to the input beam and we
see another step in the power getting through the MC. The
interferometer is still not locked, so all light (except for $2.7\%$) is reflected off of
the recycling mirror, as seen by REFL DC. The downward spikes are the
result of interferometer flashes, instances of all mirrors lining up
correctly to let some light in. 

The reason we look at two records of the light sent into the
interferometer is because from one power to the next, each gives a
different ratio of power increase. Since we don't
have a good reason to trust one more than the other, we'll take the
average of the two ratios for use in the vilibility calculation. In
this example, PWR IN says that the locked state compared to the
unlocked state had $87/6.5 = 13.4$ times more power. However, MC TRANS
says that ratio is $326/21 = 15.5$. We'll meet in the middle and use a
locked to unlocked input power ratio of 14.5. 

Then, we will look at the average minimum of REFL when the
interferometer is locked (here, about 290) and the average maximum
when it is not (241) for a measure of power reflected. The dark value,
of course, must be subtracted from each--in this case it's -22.5. The
visibility is thus:
\begin{equation}
\mathrm{visibility} = 1 - \frac{290-22.5}{15.5(241-22.5)} = 92.1\%.
\end{equation}
The exact same calculation made for a pseudo-random sampling of
Livingston lock stretches is shown in Table~\ref{table:llo_vis}.

\begin{table}
\centering
\begin{tabular}{l l l l l l}
 & & & & Visibility: & \\
Segment & Date & Time of day & PWR IN & REFL DC & trigger PD \\
\hline\hline
0169 & July 31, 2009 & afternoon & 7.2 W & 92.1\% & 92.0\% \\
1009 & Nov. 20, 2009 & evening & 5.7 W & 91.7\% & 91.8\% \\
1380 & Jan 6, 2010 & evening & 5.5 W & 92.3\% \\
1630 & Feb 23, 2010 & morning & 8.7 W & 92.1\% & 91.8\% \\
2180 & May 20, 2010 & evening & 11.6 W  & 91.6\% & 91.7\% \\
2223 & May 27, 2010 & evening & 11.5 W  & 91.9\% & 91.2\% \\
2424 & June 29, 2010 & night & 9.0 W & 91.9\% \\
2999 & Sep. 14, 2010 & evening & 9.7 W & 91.6\% \\
3159 & Oct. 2, 2010 & morning & 4.1 W & 91.8\%\\
3272 & Oct. 20, 2010 & night & 7.1 W & 91.5\% & 92.1\% \\
\hline
\end{tabular}
\caption{Livingston}
\label{table:llo_vis}
\end{table}

\begin{table}
\centering
\begin{tabular}{l l l l l l l l}
 & & & & Visibility: & & \\
Segment & Date & Time of day & PWR IN & REFLPD1 & WFS3 DC & WFS4 DC &
trigger PD\\
\hline\hline
0128 & July 20, 2009 & morning & 7.9 W & 75.5\% & 94.0\% & 93.9\% & 94.8\%\\
0685 & Nov. 9, 2009 & evening & 13.9 W & 72.5\% & 93.9\% & 94.1\% & 94.7\%\\
0999 & Jan 1, 2010 & morning & 14.0 W & 81.1\% & 93.8\% & 94.2\% & 94.4\%\\
2026 & July 9, 2010 & night & 20.3 W & 80.1\% & 92.6\% & 92.6\% & 93.4\% \\
2340 & Oct 5, 2010 & night & 20.1 W & 78.0\% & 91.6\% & 92.1\% & 93.3\%\\
\hline
\end{tabular}
\caption{Hanford}
\label{table:lho_mm}
\end{table}


\subsection{Impedance matching}
\label{sec:impedance}
The lower limit for the amount of power in the interferometer's reflected beam 
can be calculated simply based on the reflectivities of the 
mirrors. A cavity is impedance matched when the input and output
couplers have the same reflectivity--all 
light sent at the cavity is coupled into it. If there's a difference between the reflectivities of the two 
mirrors, the cavity is over- or under-coupled and light will be reflected. 

Treating the interferometer arms as a single mirror that forms a cavity with the RM, we want the RM 
transmission to match the transmission of the arms. This must include all losses in the interferometer 
such as absorption, scattering, and ETM transmission. Design estimates resulted in an RM power 
transmission of 2.7\%. 

If losses don't equal 2.7\% in reality, then we have an impedance mismatch and we will see light at the 
REFL port. We can measure the losses implicitly via the carrier recycling gain and therefore provide a 
good estimate of power in REFL due to impedance mismatch. The
amplitude reflectivity of the interferometer is:
\begin{equation} 
r_{ifo} = \frac{r_{arms}-r_{rm}}{1-r_{rm}r_{arms}}.
\end{equation}
The composite arm cavity amplitude
reflectivity is $r_{arms}$ and the RM amplitude reflectivity is
$r_{rm} = \sqrt{0.973}$. 

It is not so simple to know what $r_{arms}$ is
in practice. A precise measure of all losses in the arms would be
needed. Therefore, we turn to writing $r_{arms}$ in terms of a
quantity that we can measure quite well, the power recycled Michelson
carrier gain $G_{cr}$:
\begin{equation}
G_{cr} = g_{cr}^2 = \left[\frac{t_{rm}}{1-r_{rm}r_{arms}}\right]^2.
\end{equation}
Solving for $r_{arms}$ as a function of 

Figure \ref{fig:reflectivity} shows $R_{ifo} = r_{ifo}^2$ as a
function of $G_{cr}$. Curves for a couple different RM reflectivities
are shown to give an idea of how the interferometer reflectivity would
change for minor mis-approximations of the RM reflectivity. 
\begin{figure}
\begin{centering}
\includegraphics[width=1.0\textwidth]{figures/R_ifo.pdf}
\caption[Interferometer reflectivity due to impedance
mismatch]{Interferometer reflectivity due to impedance mismatch. The
  percentage of power incident on the RM that is reflected by the
  interferometer is a function of carrier recycling gain and RM
  reflectivity. The carrier recycling gain is 39 for LLO and 75 for
  Hanford. The recycling mirror power transmission is nominally
  2.7\%.}
\label{fig:reflectivity}
\end{centering}
\end{figure}

Experimentally, the recycling gain is measured as
\begin{equation}
G_{cr} = g_{cr}^2 = T_{rm} \frac{\mathrm{NPTRX + NPTRY}}{2},
\end{equation}
which assumes the NPTRs are calibrated such that NPTRX=NPTRY=1 during a
single arm lock. Figure \ref{fig:Gcr} shows the measured carrier
recycling gain for each interferometer during the S6 science
segments. 

\begin{figure}
\begin{centering}
%\includegraphics[width=1.0\textwidth]{figures/S6_Gcr.pdf}
\caption{Carrier recycling gain during S6.}
\label{fig:Gcr}
\end{centering}
\end{figure}



\subsection{Mode matching}
Any difference between the amount of light seen in REFL and what's expected from impedance mismatch 
can be contributed to imperfect mode matching.

For Livingston, the story is pretty straightforward. The interferometer is nearly perfectly impedance 
matched, so all light at the REFL port is due to imperfect mode matching. Thus, the LLO mode 
mismatch during S6 was 8\%.

We can't quite put together a clean case for Hanford yet as there's so much uncertainty in both the 
recycling gain and the visibility numbers. The carrier recycling gain may be as low as 53 or as high as 75. 
The visibility is either 77.5\% or 93.5\%. Table \ref{table:Gcr}
summarizes the possiblities.

\begin{table}
\centering
\begin{tabular}{l l l l l}
& $G_{cr}$ & impedance mismatch & visibility & mode mismatch\\
\hline\hline
LLO & 39 & 0.07\% & 91.84\% $\pm$ 0.07\% & 8\%\\
LHO & 53 & 4.0\% & 93.5\% & 2.5\% \\
       &      &            & 77.5\% & 18.5\% \\
       & 75 & 18.4\% & 93.5\% & impossible\\
       &      &             & 77.5\% & 4.1\% \\ 
\hline
\end{tabular}
\caption{Summary of interferometer parameters used to determine mode matching.}
\label{table:Gcr}
\end{table}

\chapter{ANGULAR SENSING AND CONTROL CALIBRATIONS}
\label{ch:ASCcal}
The typical method of calibrating a digital channel is to inject a
signal of known amplitude into the system and take the ratio with the
amplitude of the digital measurement of the signal.
% Two ways of
% determining the physical amplitude of the injected signal are
% \textcolor{blue}{(see if I can classify calibration methods)}
% \begin{itemize}
% \item 
% \item 
% \end{itemize}
I describe in this appendix the calibrations I made of some of the
angular sensing channels.
%, all the while demonstrating particular examples of these methods.

% \section{Calibrations}
% Because the data is collected digitally, the units are in digital
% counts. This must be converted into physical units in order to
% facilitate comparison to models and to make meaningful statements. The
% typical method of calibrating a digital channel is to inject a signal
% of known amplitude into the system and take the ratio with the
% amplitude of the digital measurement of the signal. Two ways of
% determining the physical amplitude of the injected signal are
% \textcolor{blue}{(see if I can classify calibration methods)}
% \begin{itemize}
% \item 
% \item 
% \end{itemize}
% I describe in this section the calibrations I made of some of the
% angular sensing channels, all the while demonstrating particular
% examples of these methods.


\section{Beam Spot Motion}
A quantity of interest is how much the beam moves on the ITMs and
ETMs. It is this beam spot motion which, together with the mirror
angular motion, creates a length signal that contributes noise to
DARM. An elegant way of following the motion of the beam on the test
masses is to track pickoffs of the light transmitted or reflected from
the mirrors. We have such signals naturally available for the ETMs and
ITMs from the QPDs which are otherwise used for ASC sensing. For
example, QPDX and QPDY see the light transmitted through each of the
ETMs and WFS2 sees the pickoff of light from the wedge of ITMX.

To calibrate the counts of the QPD and WFS2 pitch and yaw error
signals,\footnote{\texttt{L1:ASC-QPDY\_\{PIT,YAW\}\_IN1} and
\texttt{L1:ASC-WFS2\_DC\{Pitch, Yaw\}Mon}} I moved the beam a
known distance on the test mass, $\Delta x$, and recorded the
corresponding $\Delta y$ of the QPD and WFS2 readback. The ratio
$\Delta x /\Delta y$ is the calibration from counts to meters. The
details of the procedure are described below.


\subsection{Moving the Beam} 
Moving the beam on the mirrors in a controlled fashion is
straightforward because of the ASC system. All that we need to do is
introduce an offset to the setpoint of the of the beam centering
aspect of the ASC servo. For the ETMs we put a DC offset in the
\texttt{L1:ASC-QPD\{X,Y\}\_\{PIT, YAW\}\_\{OFFSET\}} channel and for
the ITMs we changed the $X$ and $Y$ targets of the beam splitter beam
centering servo.

\subsection{Measuring How Much the Beam Has Moved} 
The more difficult task is measuring just how much the beam has
moved. For this, we make use of the lever arm mechanism of angle to
length coupling as explained in Section~\ref{sec:tolerance}. The
concept of the measurement is to move the axis of rotation of the
mirror so that it passes through the center of the beam. We use the
OSEMs to change the location of the axis of rotation, and we use DARM
to determine when the axis is aligned with the beam center. For
example, if we drive the top two OSEMs more than the bottom two OSEMs,
we've created an axis of rotation that sits below the center of
mass. The result of such tuning is an effective rebalance of the
center of mass of the mirror so that it is aligned with the center of
the beam. The procedure is:
\begin{itemize}
\item Shake the mirror at some frequency $f$ (we use
39.5 Hz) during a full lock \vspace{-10pt}
\item Demodulate DARM at $f$ for several different sets of OSEM gains \vspace{-10pt}
\item Fit a quadratic to the demodulated data to pinpoint the OSEM gains that
  minimize the coupling to DARM
\end{itemize}

\begin{figure}
\begin{centering}
\includegraphics{figures/geometry_mirror_osems.pdf}
\caption[Diagram of mirror and OSEM geometry]{Geometry of OSEMs and
  mirror as used for calculating the location of the axis of rotation
  when the torques are unequal.}
\label{fig:mirror_osem_geometry}
\end{centering}
\end{figure}

Relating the OSEM gains to absolute beam position on the mirror
requires only the geometry of the mirror and OSEM setup as sketched in
Figure \ref{fig:mirror_osem_geometry}. We estimate the OSEM locations as
being on the edge of the mirror such that the length $d$ of one side
of the square that they form is given by $d =\sqrt{2} R$, where $R =
12.5$ cm is the radius of the mirror. Then, collapsing the four OSEMs
into a representative two at the centers of two opposite sides of the
square and assigning them gains of $1 + \alpha$ and $-(1-\alpha)$ for
a force $F$, we can evaluate where the pivot point $x$ is located by
setting the sum of the torques equal to zero:
\begin{equation}
F [1+\alpha] x = F [1-\alpha][d-x].
\end{equation}
Therefore, the beam location relative to center, $\Delta x$, is
\begin{equation}
\Delta x := \frac{d}{2} - x = \alpha \frac{d}{2},
\end{equation}
and for a change in a pitch or yaw coil gain, the change in beam
position, $\Delta x$, is:
\begin{equation}
|\Delta{x}| = \frac{|\Delta{\mbox{gain}}| R}{\sqrt{2}} .
\end{equation}

The final calibrations of these channels are shown in Table
\ref{table:bsmcal}. \footnote{A minor technicality is that since there
  are no filters between the QPD error signals and the offset channel,
  their units are exactly the same. Thus, calculating meters of beam
  spot motion as a function of offset serves to calibrate the error
  point. For convenience, this is what I did.}

\begin{table}
\centering
\caption[Beam spot motion calibrations]{Calibrations to be used with
 the QPDX, QPDY, and WFS2 DC pitch and yaw error signals for a
 measure of beam spot motion.} 
\begin{tabular}{l l l l}
\hline
         & ETMX & ETMY & ITMs \\
\hline
pitch & $1.03\times10^{-5}$ m/ct & $1.21\times10^{-5}$ m/ct & $5.52\times10^{-2}$ m/ct \\
yaw & $0.88\times10^{-5}$ m/ct & $0.80\times10^{-5}$ m/ct & $4.79\times10^{-2}$ m/ct \\
\hline
\end{tabular}
\label{table:bsmcal}
\end{table}



\section{Angular Mirror Motion}
The optical levers provide a straightfoward measure of individual
mirror motion. The channels I calibrated were of the form
\texttt{L1:SUS-ETMX\_OPLEV\_\{P,Y\}ERROR}, the optical lever error
signals for each of the large optics. I made use of the dependence of
power in a misaligned cavity to calibrate the ETM and ITM optical
levers, and used a less precise, rudimentary method to calibrate the
RM, BS, and MMT3 optical levers.


\subsection{ETM and ITM Optical Levers} 
I calibrated the arm cavity optical levers by tracking the power loss
in the locked arm as one of its mirrors is tilted. The closed form
expression for cavity power as a function of mirror tilt is derived in
Appendix \ref{sec:cavitypower}. All that is needed is a quadratic fit
to the data collected. From the fit parameters, I can determine the
factor, $\Delta \theta / \Delta y$, which converts the digital counts
of the optical lever channel, $y$, to units of radians.

To make the measurement, I locked a single arm and maximized the power
build up. Then I slowly stepped the pitch or yaw pointing of one of
the mirrors away to one side of resonance, and then back and to the
other side, repeating this several times. All the while, I recorded
the optical lever error signal of the mirror whose angle I was
changing, and the power in the arm as determined from the amount of
light transmitted through the
ETMs.\footnote{\texttt{L1:LSC-NPTR\{X,Y\}\_OUT16}}

From Eq.~\ref{eq:pwr_disptilt}, we see that the power in the arm, $P$,
is a function of the form 
\begin{equation}
P = P_{max} \exp{[-b (y-y_0)^2]},
\label{eq:OLcalfit}
\end{equation}
where $y_0$ is the DC offset of the optical lever channel and $b$ is
related to physical cavity axis displacement $a$ and tilt $\alpha$ by
$by^2~=~(a/w_0)^2+(\alpha/\theta_0)^2$. In order to relate the optical
lever signal, $y$, to physical cavity parameters, we divide by
$\Delta{\theta}^2$ and rearrange to get:
\begin{equation}
\frac{\Delta \theta}{\Delta y} = \sqrt{b} \left[
  \left[\frac{\Delta a/\Delta\theta}{w_0}\right]^2 +
  \left[\frac{\Delta\alpha/\Delta\theta}{\theta_0}\right]^2
\right]^{-\frac{1}{2}} .
\end{equation}
The terms in the numerators on the right hand side are fixed constants
based on the cavity geometry and can be calculated using
Eq. \ref{eq:cavitydisptilt_mirrorangle}. The measurement data and fits
are shown for both pitch and yaw in Figure \ref{fig:OLcal}. The ETM
optical levers make use of a broader range of optical lever signal
than do the ITMs. (Also note that the maximum power in the y-arm is about
10\% less than that in the x-arm. This is true at both Hanford and
Livingston, and is due to the priority given to the x-arm in the
alignment scheme, as explained in Appendix
\ref{sec:initial_alignment}.)

\begin{figure}
\begin{centering}
\subfigure{\includegraphics[width=0.5\columnwidth]{figures/olcal_pitch.pdf}}\subfigure{\includegraphics[width=0.5\columnwidth]{figures/olcal_yaw.pdf}}
%\includegraphics[width=1.0\columnwidth]{figures/oplevcal_pitch.pdf}
\caption[Optical lever calibration data.]{Optical lever calibration data and
  fits to Eq. \ref{eq:OLcalfit}.}
\label{fig:OLcal}
\end{centering}
\end{figure}


\subsection{RM, BS, and MMT3 Optical Levers}
To calibrate the RM, BS and MMT3 optical levers, I used my own eyes
and the camera images and known dimensions of the ETM beam cages. With
the interferometer unlocked, I moved the optics in pitch and yaw,
tracking the beam's movement on the ETM cages. In order to calibrate
the RM, I used the reflection off ITMY to the RM and onto the ETMY
cage. The BS and MMT3 required only straight shots to the ETMs. For
yaw, I moved the mirrors until the beam was centered on each vertical
suspension post, and for pitch I moved the beam from the center of the
mirror to the top of the cage. The beam moves by $x = 2\theta$ on
across the cage when the mirror moves by $\theta$, so with small angle
approximations, the mirror angle is simply $x/2L$ where $L$ is the
distance from mirror to ETM cage.

The final $\Delta\theta / \Delta x$ calibrations of all optical levers
are in Table \ref{table:oplevcal}. 

\begin{table}
\centering
\caption[Optical lever calibrations]{Calibrations to be used
  with the optical lever error signals 
  for a measure of angular mirror motion. Units are \microrad/ct.}
\begin{tabular}{l l l l l l l l}
\hline
        & ETMX & ETMY & ITMX & ITMY & RM & BS & MMT3 \\
\hline
pitch & 49.4 & 43.0 & 14.9 & 15.6 & 61.9 & 47.3 & 57.4 \\
yaw & 50.7 & 43.3 & 20.1 & 20.2 & 42.5 & 63.5 & 55.5 \\
\hline
\end{tabular}
\label{table:oplevcal}
\end{table}





\section{WFS Error Signals}
The WFS error signals\footnote{i.e. \texttt{L1:ASC-WFS1\_Q1}} are physically
Watts of power at the detectors, which the WFS electronics convert
into a voltage. To turn WFS counts into voltage of signal at the
output of the detector, we must backtrack through the electronics and
calibrate the WFS demodulation chain.

% The WFS error signals (i.e. \texttt{L1:ASC-WFS1\_Q1}) are physically
% Watts of power at the detectors. Converting the error signal in
% digital counts to Watts requires working backwards through the
% electronics, and can be divided into two parts. First, we must
% calibrate the WFS demodulation chain to backtrack WFS counts into
% voltage of signal at the input to the mixer. Second, we must convert
% voltage at the mixer into Watts at the sensor based on properties of
% the photodetector RF electronics. \textcolor{blue}{For now, it
%   suffices to calibrate the signal in Volts.}

%\subsubsection{Counts to Volts} 
The analog to digital RF chain for the WFS includes a demodulation
board, a whitening board, an anti-alias board, and the ADC. I
calibrated this chain by injecting a sine wave of the same frequency
as a typical WFS signal, yet of known voltage into the WFS
demodulation board. Comparing the peak to peak voltage of this input
sine wave to the peak to peak amplitude of the resulting digital
counts signal provides the Volts per count conversion. The
calibrations are presented in Table \ref{table:demodcal}.

\begin{table}
\centering
\caption[Demodulation chain calibration for each quadrant of each
 WFS]{Demodulation chain calibration for each quadrant of each
 WFS. Units are \micro V/count.}
\begin{tabular}{l l l l l l}
\hline
 & q1 & q2 & q3 & q4 & average \\
\hline
 WFS1  & 0.35 &   0.32 &   0.34 &   0.35 & 0.34 \\
 WFS2  & 8.8 &   8.6 &   8.7  &  8.5 & 8.7       \\
 WFS3  & 6.4 &   5.8 &   5.8 &   5.7 & 5.9     \\
 WFS4  & 6.3 &   5.4 &   5.3  &  7.4 & 6.1    \\
\hline
\end{tabular}
\label{table:demodcal}
\end{table}

It should be noted that the demodulation chain calibration numbers for
all quadrants of a particular WFS differ no more than 20\% from the
average. The demodulation chain does not significantly distort the
error signals.


% \subsubsection{Volts to Watts}
% The voltage created per Watt of signal on the WFS is determined by the
% responsivity (Amps/Watt) and transimpedance (Volts/Amp) of the
% diode. A model of the 25~MHz WFS transimpedance is shown in
% Figure \ref{fig:wfs25MHzTF}, indicating a magnitude of 90~dB at
% 25~MHz. The responsivity is found in Section \ref{sec:pds}. It is
% 0.86~A/W. 

% The final WFS calibration numbers are found in Table \ref{tab:WFScal}.

% \begin{figure}
% \begin{centering}
% \includegraphics[width=1.0\textwidth]{figures/wfs25MHzTF.pdf}
% \caption{Transimpedance of the 25 MHz resonant RF WFS front end
%   electronics. The model was made using software called LISO.}
% \label{fig:wfs25MHzTF}
% \end{centering}
% \end{figure}

% \begin{table}
% \centering
% \caption[]{WFS error signal calibrations from digital counts to Watts.}
% \begin{tabular}{l l l l}
% \hline
% WFS1 & WFS2 & WFS3 & WFS4 \\
% \hline
% $3.9\times10^{-16}$ & $1.0\times10^{-14}$ & & \\
% \hline
% \end{tabular}
% \label{table:WFScal}
% \end{table}



 
\chapter{Appendix}

\section{Misaligned cavity axis}
\label{sec:misaligned_cavity}
Here I provide the geometric argument that shows how to calculate the
tilt $a$ and displacement $\alpha$ of a cavity as a function of mirror
misalignment. Cavity tilt is defined by the angle formed between
the line that connects the two beam spots (as given by Eq. \ref{eq:x}) and the line joining the
centers of the mirrors. Cavity displacement uses the same two lines,
yet is defined by the distance between them at the location of the
waist of the resonant spatial mode. Based on pure geometry, the cavity
displacement and tilt are:
\begin{equation}
\left\llbracket \begin{array}{c}
a \\
\alpha \end{array} \right\rrbracket = \frac{1}{L}
\left\llbracket \begin{array}{cc}
z_2 & z_1\\
-1 & 1\end{array} \right\rrbracket
\left\llbracket \begin{array}{c}
x_1\\
x_2 \end{array} \right\rrbracket
\label{eq:cavitydisptilt}
\end{equation}
where $z_i$ is the distance to the waist from mirror $i$ calculated as:
\begin{eqnarray}
z_1 &=& \frac{g_2 (1-g_1) L}{g_1+g_2 - 2 g_1g_2} \\
z_2 &=& L - z_1.
\end{eqnarray}

Clearly, we can combine Eqs. (\ref{eq:x}) and
(\ref{eq:cavitydisptilt}) to arrive at an equation directly relating
mirror tilt to cavity displacement and tilt: 
\begin{equation}
\left\llbracket \begin{array}{c}
a \\
\alpha \end{array} \right\rrbracket = \frac{1}{1-g_1g_2}
\left\llbracket \begin{array}{cc}
g_2z_2 + z_1 & z_2 + g_1z_1\\
-g_2 + 1 & -1 + g_1\end{array} \right\rrbracket
\left\llbracket \begin{array}{c}
\theta_1\\
\theta_2 \end{array} \right\rrbracket.
\label{eq:cavitydisptilt_mirrorangle}
\end{equation}



\section{Power in a misaligned cavity}
\label{sec:cavitypower}
I'll show how to calculate the power in a cavity as a function of
cavity axis displacement and tilt. Combined with the results of
Eq. \ref{eq:cavitydisptilt_mirrorangle} we determine how the power
build-up in a cavity depends on a single mirror's angular
displacement.
% %This can be compared with data to calibrate
% %the optical lever error signals as will be shown in Section \ref{sec:oplevcal}. 

The field of a lowest-order Gaussian laser beam along one axis at the beam waist is:
\begin{equation}
\psi(x) = U_0(x) = \left[ \frac{2}{\pi w_0^2} \right]^{1/4} \exp{\left[-\left[\frac{x}{w_0}\right]^2\right]}
\end{equation}
where $w_0$ is the beam waist radius and $U_0$ is the lowest-order
Hermite polynomial. The Hermite polynomials are orthonormal,
ie. $\langle U_i \mid U_j \rangle = \delta_{ij}$. For example, the
next to lowest order polynomial is:
\begin{equation}
U_1(x) = \left( \frac{2}{\pi w_0^2} \right)^{1/4} \frac{2x}{w_0}
\exp{[-(x/w_0)^2]} = \frac{2x}{w_0} U_0(x)
\end{equation}

\subsection{Displaced cavity}
The field of a cavity with a \emph{displaced} z-axis at the cavity waist is:
\begin{align}
\psi \prime (x) =& \psi(x-a) \\
 =& U_0(x-a) \\
 =& c_0U_0(x) + c_1 U_1(x) + c_2 U_2(x) + ...\\
\end{align}
where $a$ is the displacement of the axis and $c_i$ are constants.

%\subsubsection{Power} 
We want to know $c_0$, the projection of the displaced cavity field onto the beam field: 
\begin{eqnarray}
c_0 &=& \langle \psi \mid \psi \prime \rangle \\
&=& \int_{-\infty}^\infty \psi(x) \psi \prime(x) \, dx \\
 &=& \exp{[-a^2/2 w_0^2]}
\label{eq:c_0}
\end{eqnarray}
The power in this mode is the square of the overlap of the two fields: 
\begin{eqnarray}
P_0&=& \left| \langle \psi \mid \psi \prime \rangle \right| ^2\\ 
&=& \exp{[-[a/w_0]^2]} 
\end{eqnarray}

%\subsubsection{$U_1$ field}
For the purpose of wavefront sensing, we need to know the amplitude,
$c_1$, of the first order $U_1$ field. This can be approximated as
demonstrated in Anderson \cite{Anderson1984Alignment} using
the Taylor series expansion of the exponential in $\psi \prime(x) =
U_0(x-a)$, assuming a displacement $a$ that's small compared to waist
size $w_0$.
\begin{eqnarray}
\psi \prime(x) &=& \left[ \frac{2}{\pi w_0^2} \right]^{1/4}
\exp{\left[-\left[\frac{x-a}{w_0}\right]^2\right]} \\
&=& \left[ \frac{2}{\pi w_0^2} \right]^{1/4} \left[1 -
  \left[\frac{x-a}{w_0} \right]^2 + \mathcal{O}(a^4) \right] \\
&=& \left[ \frac{2}{\pi w_0^2} \right]^{1/4} \left[ \frac{2xa}{w_0^2}
  \left[1-\frac{x^2}{w_0^2} + ... \right] + \left[ 1 -
    \frac{x^2}{w_0^2} + \frac{1}{2} \frac{x^4}{w_0^4}
    - ... \right] + \mathcal{O}(a^2) \right] \\
&=& \left[ \frac{2}{\pi w_0^2} \right]^{1/4} \left[1 +
  \frac{2xa}{w_0^2} + \mathcal{O}(a^2) \right] \exp{\left[ -\left[\frac{x}{w_0}\right]^2
  \right]} \\
&=& U_0(x) + \frac{a}{w_0} U_1(x) + ...
\end{eqnarray}
Notice that here we find $c_0 = 1$, which is consistent with the
exact result of Eq.~\ref{eq:c_0} when we apply our $a^2 \approx 0$
approximation. We find that the amplitude of the first order
Hermite-Gauss field for a displaced cavity is
\begin{equation}c_1 = a/w_0.
\end{equation}



\subsection{Tilted cavity}
The field of a cavity with a \emph{tilted} z-axis at the cavity waist is a tad more complex to derive. We assume the tilt, $\alpha$, is small such that $\sin{\alpha} \approx \alpha$ and $\cos{\alpha} \approx 1$. Also, we assume the beam divergence angle, $\theta_0=\lambda / \pi w_0$, is small such that the wavefronts near the waist can be considered parallel to one another. 

Here, the important quantity to consider is the phase of the cavity field at the cross-section of the beam waist.  The phase is either advanced or retarded compared to that of the beam:
\begin{eqnarray}
\psi \prime(x) &=& \psi(x \prime) \exp{[-i k z \prime]} \\
 & \approx& \psi(x \cos{\alpha}) \exp{[-i k x \sin{\alpha}]} \\
 &\approx& \psi(x) \exp{[- i k x \alpha]} \\
 &=& U_0(x) \exp{[- i k x \alpha]}
\label{eq:psi_tilted}
\end{eqnarray}
where $k=2\pi / \lambda$ and $\lambda$ is the wavelength of the laser light.

%\subsubsection{Power} 
The overlap of the fields of the beam and tilted cavity is $ \exp{[-\alpha^2/2 \theta_0^2]} $. Therefore the power is:
\begin{equation}
P_0 = \exp{[-(\alpha/\theta_0)^2]}.
\end{equation}

%\subsubsection{$U_1$ field}
An expansion of the exponential in Eq. \ref{eq:psi_tilted} for a small
tilt $\alpha$ gives:
\begin{eqnarray}
\psi \prime(x) &=& U_0(x) [1 + i k x \alpha + \mathcal{O}(\alpha^2) ] \\
&=& U_0(x) + \frac{i k \alpha w_0}{2} U_1(x) + \mathcal{O}(\alpha^2).
\end{eqnarray}
Therefore, the amplitude of the first order Hermite-Gauss field for a
tilted cavity is 
\begin{equation}
c_1 = i k \alpha w_0 / 2.
\end{equation}


\subsection{Displaced and tilted cavity}
The most general case, of course, is to have a cavity axis that is both displaced \emph{and} tilted at the beam waist:
\begin{equation}
\psi \prime(x) = \psi(x-a) \exp{[-i k (x-a) \alpha]}.
\end{equation}
We find:
\begin{equation}
\langle \psi \mid \psi \prime \rangle = \exp{\left[- \frac{a^2}{2 w_0^2} \right]} \exp{\left[-\frac{\alpha^2}{2 \theta_0^2}\right]} \exp{\left[- \frac{i a \alpha}{x_0 \theta_0}\right]}
\end{equation}
and
\begin{equation}
P_0 = \exp{\left[- \frac{a^2}{w_0^2} \right]}
\exp{\left[-\frac{\alpha^2}{\theta_0^2}\right]}.
\label{eq:pwr_disptilt}
\end{equation}



\section{Initial DC alignment of the interferometer}
\label{sec:initial_alignment}
After any kind of in-vacuum work, the DC alignment of the mirrors is
usually too poor for the interferometer to lock. A bootstrapping
process of tweaking the alignment by hand is necessary, assuming the
mirrors start out pointing in generally the right direction, as is
usually the case. As pointed out in \ref{sec:alignment_overview}, the
QPDs at the end stations are the fixed reference points for the
overall alignment, so this process begins with making sure the light
reaches them. We then adjust the rest of the mirrors to maximize power
build-up in the arms and to maximze spatial overlap of the light
reflected from each arm.

An outline of the process is presented here. ``Misalign'' means to
intentionally point a mirror so far away from any known good positions
as to eliminate it from the configuration.  ``Align'' and ``restore''
mean to bring a mirror or configuration to the best known
position(s). Centering the beam on a mirror is accomplished by using
the suspension cage surrounding the mirror as a reference. Camera
images and QPD readback provide the signals used for beam centering. 

\begin{itemize}
\item[] \textbf{X-arm} \vspace{-10pt}
\item restore the x-arm (misalign RM, ITMY, and ETMY, align ITMX and ETMX) \vspace{-10pt}
\item use ITMX to center the beam on QPDX \vspace{-10pt}
\item use ETMX to center the beam on ITMX \vspace{-10pt}
\item with x-arm locked, use MMT3 to maximize the x-arm power build-up (NPTRX, can expect
  about 95\%) \vspace{-10pt}
\item save the MMT3, ITMX, and ETMX alignment settings 
\item[] \textbf{Y-arm} \vspace{-10pt}
\item restore the y-arm (misalign ITMX and ETMX, align ITMY and ETMY) \vspace{-10pt}
\item use ITMY to center the beam on QPDY \vspace{-10pt}
\item use ETMY to center the beam on ITMY \vspace{-10pt}
\item with y-arm locked, use BS to maximize the y-arm power build-up (NPTRY, can expect
  about 90\%) \vspace{-10pt}
\item save the BS, ITMY, and ETMY alignment settings 
\item[] \textbf{Relative x-arm and y-arm} \vspace{-10pt}
\item note AS beam position on camera while toggling between x-arm and
  y-arm locks\vspace{-10pt}
\item use ETMs to align the two AS beams \vspace{-10pt}
\item restore the Michelson (misalign ETMs, align ITMs) \vspace{-10pt}
\item use BS to make AS port as dark as possible \vspace{-10pt}
\item re-do y-arm alignment if ambitious 
\item[] \textbf{Recycling mirror} \vspace{-10pt}
\item restore the PRM (misalign ETMs, align ITMs and RM)
  \vspace{-10pt}
\item use RM to center beam on ETMY cage 
\item[] \textbf{Restore full interferometer--off you go!}
\end{itemize}



\section{Channel names}
It is often confusing why there are so many channels with almost
identical names. And just what those names mean can also sometimes be
elusive. The best advice I have to offer about how to figure out for
one's self what the root of the channel name conveys about the data it
represents is to browse the MEDM screens. The MEDM screens pictorially
show the flow of which channels are derived from others, and whether
they sit before or after an excitation point, for example. They also
sometimes offer hints as to exactly what calculation is done under the
scene in the front-end code to compute new channels. 

The topic I want to cover here is the difference between the
\texttt{OUT}, \texttt{OUT16}, \texttt{OUTPUT}, and \texttt{OUTMON}
channel name extenstions. They appear in all of the subsystems, and it
is crucial to realize that some are good for analyzing data and some
good for getting only a sense in real-time of what's going on. Some
are typically recorded to disk, and some are not. These concepts
extend to other common name extensions as well. 



