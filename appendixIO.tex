\chapter{Appendix A}



\section{Overlap integrals}
A measure of the mode matching can be given by the amount of power
coupled from one mode into another. This is calculated as the square
of the overlap integral of two fields, $\psi_1$ and $\psi_2$, for a
particular z-axis (propagation direction) cross-section:
\begin{equation}
P = \left| \langle \psi \mid \psi^{\prime} \rangle \right|^2 = \left[\int_{-\infty}^\infty \psi^\star(x) \psi^{\prime}(x) \, dx \right]^2
\end{equation}
We are interested in the lowest order Hermite-Gaussian mode:
\begin{equation}
\psi(x,z) = u_0(x,z) = \left[ \frac{2}{\pi w_0^2} \right]^{1/4} \left[
  \frac{q_0}{q(z)} \right]^{1/2} \exp{\left[ \frac{-i k x^2}{2 q(z)}
  \right] }
\end{equation}
which can be rewritten as a function of $x$ and $q$ using
$q_0=i \mbox{Im}(q)$, and $w_0^2 = -2 q_0 i/k$:
\begin{equation}
u_0(x,q) = \left[ \frac{-k \mbox{Im}(q)}{\pi q^2} \right]^{1/4} \exp{\left[ \frac{-i k x^2}{2 q}
  \right] }
\label{eq:u0_xq}
\end{equation}
Eq. (\ref{eq:u0_xq}) is normalized such that $\langle u_0 \mid u_0
\rangle = 1$.

We want to know the square of the overlap integral for two fields
given by different $q$ parameters at one location $z$. First, the
overlap integral:
\begin{align}
\langle q_1 \mid q_2 \rangle &= \int_{-\infty}^\infty u_0(x, q_1) u_0(x, q_2) \, dx \\
 &= \left[ \frac{k^2 \mathrm{Im}(q^\star_1) \mathrm{Im}(q_2)}{\pi^2 q^{\star 2}_1 q^2_2} \right]^{1/4} 
 \int_{-\infty}^\infty \exp{\left[ -\left[ \frac{-i k}{2 q^\star_1} + \frac{i k}{2 q_2} \right]x^2 \right]} \, dx \\
 &= \left[ -\frac{k^2 \mathrm{Im}(q_1) \mathrm{Im}(q_2)}{\pi^2
     q^{\star 2}_1 q^2_2} \right]^{1/4} \sqrt{\frac{\pi}{\left[ \frac{-i k}{2 q^\star_1} + \frac{i k}{2 q_2}
   \right]}} \\
 &= \left[ \mathrm{Im}(q_1) \mathrm{Im}(q_2) \right]^{1/4}
 \sqrt{\frac{2}{q^\star_1 q_2 \left( 1/q_2 - 1/q^\star_1 \right)}} 
\end{align}
Then, the power is given by:
\begin{align}
\left| \langle q_1 \mid q_2 \rangle \right|^2
&= \frac{2 \sqrt{\mathrm{Im}(q_1) \mathrm{Im}(q_2)}}  
{\sqrt{q^\star_1 q_1 q^\star_2 q_2 \left[\frac{1}{q_2} - \frac{1}{q^\star_1} \right]
    \left[\frac{1}{q^\star_2} - \frac{1}{q_1} \right]}} \\
&= \frac{2 \sqrt{\mathrm{Im}(q_1) \mathrm{Im}(q_2)}}{|q_2 - q^\star_1 |} 
\label{eq:power_qs}
\end{align}
Note that Eq. (\ref{eq:power_qs}) simplifies to 1 when $q_1 = q_2$ as
expected and that this whole formulation assumes that the beams are
propagating along the same $z$-axis.




\section{Gaussian beam on a split photodetector}
\label{app:beamonQPD}
The power per area of a Gaussian beam traveling along the $z$-axis is 
\begin{equation}
p(x,y) = \frac{2 P_0}{\pi w^2} \mbox{exp}\left[\frac{-2x^2}{w^2}\right] \mbox{exp}\left[\frac{-2y^2}{w^2}\right]
\end{equation}
where $w$ is the beam radius at $z$.  This has been normalized such
that $\int_{-\infty}^\infty \int_{-\infty}^\infty p(x,y) \,dx \,dy =
P_0$. Then, for a beam displaced by $x_0$ from the center of a split
photodetector, the power on the left side is
\begin{eqnarray} 
P_{left} &=& \frac{2 P_0}{\pi w^2} \int_{-\infty}^{x_0} \mbox{exp}\left[\frac{-2x^2}{w^2}\right]  \,dx \int_{-\infty}^\infty \mbox{exp}\left[\frac{-2y^2}{w^2}\right] \,dy \\
&=& \sqrt{\frac{2}{\pi}} \frac{P_0}{w} \left[ \int_{-\infty}^{0} \mbox{exp}\left[\frac{-2x^2}{w^2}\right]  \,dx + \int_0^{x_0} \mbox{exp}\left[\frac{-2x^2}{w^2}\right] \,dy \right] \\
&=& \sqrt{\frac{2}{\pi}} \frac{P_0}{w} \left[ \frac{w}{2} \sqrt{\frac{\pi}{2}} + \frac{w}{2} \sqrt{\frac{\pi}{2}} \int_0^{\sqrt{2} x_0 / w} \mbox{exp}\left[-t^2 \right] \,dt  \right] \\
&=& \frac{P_0}{2} \left[1 + \mbox{erf}\left[ \frac{\sqrt{2} x_0}{w} \right] \right].
\end{eqnarray}
where $\mathrm{erf}(t_0) \equiv \frac{2}{\sqrt{\pi}} \int_0^{t_0}
\mbox{exp}\left[-t^2 \right] \,dt$. The power on the right side is:
\begin{equation}
P_{right} = \frac{P_0}{2} \left[1 - \mbox{erf}\left[ \frac{\sqrt{2} x_0}{w} \right] \right].
\end{equation}
We create a normalized yaw as
\begin{equation}
\mathrm{YAW} = \frac{P_{left} - P_{right}}{P_0} = \mathrm{erf}\left[ \frac{\sqrt{2} x_0}{w} \right].
\end{equation}
Using the Taylor series expansion of the error function, we have a
first order estimate for the relationship between normalized yaw and
beam displacement $x_0$ for a beam of \textcolor{blue}{diameter(??)} $w$:
\begin{equation}
\frac{x_0}{\mathrm{YAW}} \approx \frac{w}{2} \sqrt{\frac{\pi}{2}}.
\end{equation}
The same equation holds true for pitch.




\section{ABCD matrix formalism}
For the input beam model and for the Input Optics beam drift
calibrations, the ABCD matrix formalism is a useful tool to propagate
a Gaussian beam. I choose to ignore the fact that the MC beam passes
through the substrate of MC3 on its way to the Faraday. I also treat
the beam splitter as a flat mirror and ignore the presence of its
substrate. I use the thickness of the large optic substrates, $t=0.01$
m, and account for index of refraction effects when passing through
optics. Signs of radii of curvature are defined per the front face of
the optic; for example, all main LIGO optics have a positive $R$.

For a beam that strikes a flat interface and exits at a curved
interface (ie. forward-going transmission through RM ):
\begin{equation}
\left\llbracket \begin{array}{c c}
A & B \\
C & D \end{array} \right\rrbracket = 
\left\llbracket \begin{array}{c c}
1 & 0 \\
(n_2-n_1)/Rn_1 & n_2/n_1 \end{array} \right\rrbracket
\left\llbracket \begin{array}{c c}
1 & t \\
0 & 1 \end{array} \right\rrbracket
\left\llbracket \begin{array}{c c}
1 & 0 \\
0 & n_1/n_2 \end{array} \right\rrbracket
\end{equation}

For a beam that strikes a curved interface and exits at a flat
interface (ie. transmission through ETM):
\begin{equation}
\left\llbracket \begin{array}{c c}
A & B \\
C & D \end{array} \right\rrbracket = 
\left\llbracket \begin{array}{c c}
1 & 0 \\
0 & n_2/n_1 \end{array} \right\rrbracket
\left\llbracket \begin{array}{c c}
1 & t \\
0 & 1 \end{array} \right\rrbracket
\left\llbracket \begin{array}{c c}
1 & 0 \\
(n_1-n_2)/-Rn_2 & n_1/n_2 \end{array} \right\rrbracket
\end{equation}

For a beam that strikes a flat interface, travels through the
substrate, reflects off the back of a curved interface, travels
through substrate and exits at the original flat interface (ie. single
bounce off RM):
\begin{equation}
\left\llbracket \begin{array}{c c}
A & B \\
C & D \end{array} \right\rrbracket = 
\left\llbracket \begin{array}{c c}
1 & 0 \\
0 & n_2/n_1 \end{array} \right\rrbracket
\left\llbracket \begin{array}{c c}
1 & t \\
0 & 1 \end{array} \right\rrbracket
\left\llbracket \begin{array}{c c}
1 & 0 \\
2/R & 1 \end{array} \right\rrbracket
\left\llbracket \begin{array}{c c}
1 & t \\
0 & 1 \end{array} \right\rrbracket
\left\llbracket \begin{array}{c c}
1 & 0 \\
0 & n_1/n_2 \end{array} \right\rrbracket
\end{equation}

Finally, for prompt reflection off a curved interface (ie. reflection
off MMTs):
\begin{equation}
\left\llbracket \begin{array}{c c}
A & B \\
C & D \end{array} \right\rrbracket = 
\left\llbracket \begin{array}{c c}
1 & 0 \\
-\frac{2}{R} & 1 \end{array} \right\rrbracket
\end{equation}
and for propagation a distance $d$ through vacuum:
\begin{equation}
\left\llbracket \begin{array}{c c}
A & B \\
C & D \end{array} \right\rrbracket = 
\left\llbracket \begin{array}{c c}
1 & d \\
0 & 1 \end{array} \right\rrbracket.
\end{equation}

For each of these $n_1 = 1$ is the index of refraction of vacuum and
$n_2 = 1.44963$ is the index of refraction of the fused silica used
for the optics. Table \ref{tab:ROCs} shows the radii of curvature of
each of the optics for both sites.



\subsection{Beam Drift Calibration}
We use the ABCD matrix formulation to convert pitch and yaw data of
WFS3 and WFS4 into a position and angle at the Faraday isolator. The
basic relationship between beam displacement and angle at one location
to displacement and angle at another location is given by:
\begin{equation}
\left \llbracket \begin{array}{c} 
x_1 \\
x^{\prime}_1 \end{array} \right\rrbracket =
\left\llbracket \begin{array}{c c}
A & B \\
C & D \end{array} \right\rrbracket
\left \llbracket \begin{array}{c} 
x_0 \\
x^{\prime}_0 \end{array} \right\rrbracket
\label{eq:abcd}
\end{equation}

For this application, we want to relate the beam positions on the WFS,
$x_{3}$ and $x_{4}$, to the beam position and angle, $x_{FI}$ and
$x^{\prime}_{FI}$, at the Faraday isolator. Using only the top
equation of Eq. \ref{eq:abcd} since the WFS are sensitive to beam
position only and not angle, we can write a new relation
\begin{equation}
\left \llbracket \begin{array}{c} 
x_3 \\
x_4 \end{array} \right\rrbracket =
\left\llbracket \begin{array}{c c}
A_3 & B_3 \\
A_4 & B_4 \end{array} \right\rrbracket
\left \llbracket \begin{array}{c} 
x_{FI} \\
x^{\prime}_{FI} \end{array} \right\rrbracket
\end{equation}
where ${A_3, B_3}$ and ${A_4, B_4}$ are the A and B ABCD matrix
elements for the beam paths from the Faraday isolator to WFS3 and
WFS4, respectively. Taking the inverse and writing $x_3$ and $x_4$ as
a function of the pitch and yaw recorded by the WFS (see Appendix
\ref{app:beamonQPD}), the useful equation is
\begin{equation}
\left \llbracket \begin{array}{c} 
x_{FI} \\
x^{\prime}_{FI} \end{array} \right\rrbracket =
\left\llbracket \begin{array}{c c}
A_3 & B_3 \\
A_4 & B_4 \end{array} \right\rrbracket^{-1}
\left\llbracket \begin{array}{c c}
w_3 & 0 \\
0 & w_4 \end{array} \right\rrbracket
\frac{1}{2} \sqrt{\frac{\pi}{2}} 
\left \llbracket \begin{array}{c} 
DOF_3 \\
DOF_4 \end{array} \right\rrbracket
\end{equation}
where $w_3$ and $w_4$ are the radii of the beam at each WFS and
DOF can mean PIT or YAW.




\section{Mode cleaner pole}
\label{sec:MCpole}
Optical cavities act as low pass filters for intensity variations of
the light sent into them. The model for an intensity noise transfer
function of a cavity is that of a single pole:
\begin{equation}
\frac{E_{after}}{E_{before}} = \frac{1}{1 + s/s_0} = \frac{s_0}{s_0 - s}
\end{equation}
where $s$ is a complex parameter. However, we are interested in only
purely sinusoidal variations in intensity so we let $s$ be purely
imaginary, $s=i\omega$, where $\omega$ is an angular frequency.

We measured the intensity noise transfer function of the Livingston
mode cleaner upon completion of the Enhanced LIGO Input Optics
upgrade. We modulated the intensity of the laser light going into the
MC by injecting a swept-sine excitation in \texttt{L1:PSL-ISS\_EXC}
and measured the power variation of the light in two places: before
and after the mode cleaner. We used a single photodetector (PDA55) in
order to eliminate the PD response, and therefore made the measurement
twice. We ensured there was 1V DC on the PD in both locations.

\begin{figure}
\begin{centering}
\subfigure{\includegraphics{figures/MCpole_mag.pdf}}
\subfigure{\includegraphics{figures/MCpole_phase.pdf}}
\caption[Livingston mode cleaner intensity noise transfer
function]{Livingston mode cleaner intensity noise transfer
  function. Red open circles are data; solid blue line is a single
  pole fit. Relating fit parameters to the model, the pole frequency
  is $f_{MC}=4762$ Hz.}
\label{fig:mcpole}
\end{centering}
\end{figure}

Figure \ref{fig:mcpole} shows the transfer function data and the fit
(to both magnitude and phase simultaneously). The fit has a pole
frequency of $f_p=4762$ Hz. The $1/e$ ringdown time of the mode
cleaner is therefore $\tau = 1/4\pi f_{MC} = 16.7$ \micro s and the
finesse is $\mathcal{F} = \mathrm{FSR}/2f_p = 1282$.





\section{Carrier Mode Matching into the Interferometer}

The mode matching into the interferometer can be measured by comparing
the power reflected from the interferometer during a full lock to the
power reflected with a single bounce off the RM.

First, we will look at the interferometer visibility--what percentage
of power sent to the interferometer is reflected by it. 

Then, we'll
consider the sources of reflected light and deduce from the visiblity
measurements just how much of the reflected light is due to a mode
mismatch. 

For Livingston, we will see that the visibility is a very
close approximation of the mode mismatch. However, at Hanford, much of
the reflected light is due to an impedance mismatch, so further
analysis of the visibility is presented to determine the mode
matching.


\subsection{Interferometer visibility}
The visibility is a measure of how much carrier light is reflected from the
locked interferometer compared to how much carrier light is sent in. Ideally,
all of the light should be coupled into the interferometer. However,
any mode mismatch of the input beam to the cavity or any mismatch of
interferometer losses to PRM reflectivity (impedance mismatch, see
Section \ref{sec:impedance}) will
cause carrier light to be reflected. The visibility summarizes the
compound effect of all of these sources of reflected light.

To measure the visibility, we need to know only a few numbers. We must
have a measure of how much light is sent to the interferometer for
normalization purposes and we must have a measure of the DC reflected
power when the interferometer is not locked (all light is reflected
off of the RM) and when it is locked. The visibility is then given
by:
\begin{equation}
\mbox{visibility} = 1 -
\frac{P_{REFL_{locked}}}{P_{REFL_{unlocked}}} \frac{P_{IN_{unlocked}}}{P_{IN_{locked}}}
\end{equation}

We have two measures of how much light is being sent into the
interferometer and several of how much light is
reflected. \texttt{L1:IOO-MC\_PWR\_IN} is a pick off of the light on
the PSL table just before it 
enters vaccuum and \texttt{L1:IOO-MC\_TRANS\_SUM} is a pick off of the light just after
the MC. \texttt{L1:LSC-REFL\_DC} is a record of the power at the REFL
port as seen by one of the LSC REFL PDs. There are others, like WFS3
sum, WFS4 sum, and the trigger PD that we could also look at. Together, these channels tell
a story about the percentage of incoming light rejected by the interferometer.

\begin{figure}
\begin{centering}
\includegraphics{figures/llo_mm1630bigger.pdf}
\caption[End of an 8.7 W lock at Livingston on Feb. 23, 2010]{End of
  an 8.7 W lock at Livingston on Feb. 23, 2010. The mode cleaner
  re-locks at 0.5~W about 20~seconds after lock loss and then the
  power is increased to 2~W. Comparison of the minimum intereferometer
  reflected power during lock and the maximum reflected power out of
  lock provides a measure of interferometer mode matching.}
\label{fig:llomm1630}
\end{centering}
\end{figure}

An example for LLO is shown in Figure \ref{fig:llomm1630}. The lock
stretch ends at $t=5$ min. Note that the amount of reflected light is 
increasing up to the end of the lock as the interferometer is losing
stability. When lock is lost, the common mode servo kicks the mode
cleaner out of lock too, and the MC trans power drops to 0. About 15 seconds later the MC relocks and then
the power into it increases. About 15 seconds after that, at 5.5 minutes, the MC WFS turn
on, improving the alignment of the MC to the input beam and we
see another step in the power getting through the MC. The
interferometer is still not locked, so all light (except for $2.7\%$) is reflected off of
the recycling mirror, as seen by REFL DC. The downward spikes are the
result of interferometer flashes, instances of all mirrors lining up
correctly to let some light in. 

The reason we look at two records of the light sent into the
interferometer is because from one power to the next, each gives a
different ratio of power increase. Since we don't
have a good reason to trust one more than the other, we'll take the
average of the two ratios for use in the vilibility calculation. In
this example, PWR IN says that the locked state compared to the
unlocked state had $87/6.5 = 13.4$ times more power. However, MC TRANS
says that ratio is $326/21 = 15.5$. We'll meet in the middle and use a
locked to unlocked input power ratio of 14.5. 

Then, we will look at the average minimum of REFL when the
interferometer is locked (here, about 290) and the average maximum
when it is not (241) for a measure of power reflected. The dark value,
of course, must be subtracted from each--in this case it's -22.5. The
visibility is thus:
\begin{equation}
\mathrm{visibility} = 1 - \frac{290-22.5}{15.5(241-22.5)} = 92.1\%.
\end{equation}
The exact same calculation made for a pseudo-random sampling of
Livingston lock stretches is shown in Table~\ref{table:llo_vis}.

\begin{table}
\centering
\begin{tabular}{l l l l l l}
 & & & & Visibility: & \\
Segment & Date & Time of day & PWR IN & REFL DC & trigger PD \\
\hline\hline
0169 & July 31, 2009 & afternoon & 7.2 W & 92.1\% & 92.0\% \\
1009 & Nov. 20, 2009 & evening & 5.7 W & 91.7\% & 91.8\% \\
1380 & Jan 6, 2010 & evening & 5.5 W & 92.3\% \\
1630 & Feb 23, 2010 & morning & 8.7 W & 92.1\% & 91.8\% \\
2180 & May 20, 2010 & evening & 11.6 W  & 91.6\% & 91.7\% \\
2223 & May 27, 2010 & evening & 11.5 W  & 91.9\% & 91.2\% \\
2424 & June 29, 2010 & night & 9.0 W & 91.9\% \\
2999 & Sep. 14, 2010 & evening & 9.7 W & 91.6\% \\
3159 & Oct. 2, 2010 & morning & 4.1 W & 91.8\%\\
3272 & Oct. 20, 2010 & night & 7.1 W & 91.5\% & 92.1\% \\
\hline
\end{tabular}
\caption{Livingston}
\label{table:llo_vis}
\end{table}

\begin{table}
\centering
\begin{tabular}{l l l l l l l l}
 & & & & Visibility: & & \\
Segment & Date & Time of day & PWR IN & REFLPD1 & WFS3 DC & WFS4 DC &
trigger PD\\
\hline\hline
0128 & July 20, 2009 & morning & 7.9 W & 75.5\% & 94.0\% & 93.9\% & 94.8\%\\
0685 & Nov. 9, 2009 & evening & 13.9 W & 72.5\% & 93.9\% & 94.1\% & 94.7\%\\
0999 & Jan 1, 2010 & morning & 14.0 W & 81.1\% & 93.8\% & 94.2\% & 94.4\%\\
2026 & July 9, 2010 & night & 20.3 W & 80.1\% & 92.6\% & 92.6\% & 93.4\% \\
2340 & Oct 5, 2010 & night & 20.1 W & 78.0\% & 91.6\% & 92.1\% & 93.3\%\\
\hline
\end{tabular}
\caption{Hanford}
\label{table:lho_mm}
\end{table}


\subsection{Impedance matching}
\label{sec:impedance}
The lower limit for the amount of power in the interferometer's reflected beam 
can be calculated simply based on the reflectivities of the 
mirrors. A cavity is impedance matched when the input and output
couplers have the same reflectivity--all 
light sent at the cavity is coupled into it. If there's a difference between the reflectivities of the two 
mirrors, the cavity is over- or under-coupled and light will be reflected. 

Treating the interferometer arms as a single mirror that forms a cavity with the RM, we want the RM 
transmission to match the transmission of the arms. This must include all losses in the interferometer 
such as absorption, scattering, and ETM transmission. Design estimates resulted in an RM power 
transmission of 2.7\%. 

If losses don't equal 2.7\% in reality, then we have an impedance mismatch and we will see light at the 
REFL port. We can measure the losses implicitly via the carrier recycling gain and therefore provide a 
good estimate of power in REFL due to impedance mismatch. The
amplitude reflectivity of the interferometer is:
\begin{equation} 
r_{ifo} = \frac{r_{arms}-r_{rm}}{1-r_{rm}r_{arms}}.
\end{equation}
The composite arm cavity amplitude
reflectivity is $r_{arms}$ and the RM amplitude reflectivity is
$r_{rm} = \sqrt{0.973}$. 

It is not so simple to know what $r_{arms}$ is
in practice. A precise measure of all losses in the arms would be
needed. Therefore, we turn to writing $r_{arms}$ in terms of a
quantity that we can measure quite well, the power recycled Michelson
carrier gain $G_{cr}$:
\begin{equation}
G_{cr} = g_{cr}^2 = \left[\frac{t_{rm}}{1-r_{rm}r_{arms}}\right]^2.
\end{equation}
Solving for $r_{arms}$ as a function of 

Figure \ref{fig:reflectivity} shows $R_{ifo} = r_{ifo}^2$ as a
function of $G_{cr}$. Curves for a couple different RM reflectivities
are shown to give an idea of how the interferometer reflectivity would
change for minor mis-approximations of the RM reflectivity. 
\begin{figure}
\begin{centering}
\includegraphics[width=1.0\textwidth]{figures/R_ifo.pdf}
\caption[Interferometer reflectivity due to impedance
mismatch]{Interferometer reflectivity due to impedance mismatch. The
  percentage of power incident on the RM that is reflected by the
  interferometer is a function of carrier recycling gain and RM
  reflectivity. The carrier recycling gain is 39 for LLO and 75 for
  Hanford. The recycling mirror power transmission is nominally
  2.7\%.}
\label{fig:reflectivity}
\end{centering}
\end{figure}

Experimentally, the recycling gain is measured as
\begin{equation}
G_{cr} = g_{cr}^2 = T_{rm} \frac{\mathrm{NPTRX + NPTRY}}{2},
\end{equation}
which assumes the NPTRs are calibrated such that NPTRX=NPTRY=1 during a
single arm lock. Figure \ref{fig:Gcr} shows the measured carrier
recycling gain for each interferometer during the S6 science
segments. 

\begin{figure}
\begin{centering}
%\includegraphics[width=1.0\textwidth]{figures/S6_Gcr.pdf}
\caption{Carrier recycling gain during S6.}
\label{fig:Gcr}
\end{centering}
\end{figure}



\subsection{Mode matching}
Any difference between the amount of light seen in REFL and what's expected from impedance mismatch 
can be contributed to imperfect mode matching.

For Livingston, the story is pretty straightforward. The interferometer is nearly perfectly impedance 
matched, so all light at the REFL port is due to imperfect mode matching. Thus, the LLO mode 
mismatch during S6 was 8\%.

We can't quite put together a clean case for Hanford yet as there's so much uncertainty in both the 
recycling gain and the visibility numbers. The carrier recycling gain may be as low as 53 or as high as 75. 
The visibility is either 77.5\% or 93.5\%. Table \ref{table:Gcr}
summarizes the possiblities.

\begin{table}
\centering
\begin{tabular}{l l l l l}
& $G_{cr}$ & impedance mismatch & visibility & mode mismatch\\
\hline\hline
LLO & 39 & 0.07\% & 91.84\% $\pm$ 0.07\% & 8\%\\
LHO & 53 & 4.0\% & 93.5\% & 2.5\% \\
       &      &            & 77.5\% & 18.5\% \\
       & 75 & 18.4\% & 93.5\% & impossible\\
       &      &             & 77.5\% & 4.1\% \\ 
\hline
\end{tabular}
\caption{Summary of interferometer parameters used to determine mode matching.}
\label{table:Gcr}
\end{table}
