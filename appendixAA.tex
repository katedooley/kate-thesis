\chapter{Appendix}

\section{Misaligned cavity axis}
\label{sec:misaligned_cavity}
Here I provide the geometric argument that shows how to calculate the
tilt $a$ and displacement $\alpha$ of a cavity as a function of mirror
misalignment. Cavity tilt is defined by the angle formed between
the line that connects the two beam spots (as given by Eq. \ref{eq:x}) and the line joining the
centers of the mirrors. Cavity displacement uses the same two lines,
yet is defined by the distance between them at the location of the
waist of the resonant spatial mode. Based on pure geometry, the cavity
displacement and tilt are:
\begin{equation}
\left\llbracket \begin{array}{c}
a \\
\alpha \end{array} \right\rrbracket = \frac{1}{L}
\left\llbracket \begin{array}{cc}
z_2 & z_1\\
-1 & 1\end{array} \right\rrbracket
\left\llbracket \begin{array}{c}
x_1\\
x_2 \end{array} \right\rrbracket
\label{eq:cavitydisptilt}
\end{equation}
where $z_i$ is the distance to the waist from mirror $i$ calculated as:
\begin{eqnarray}
z_1 &=& \frac{g_2 (1-g_1) L}{g_1+g_2 - 2 g_1g_2} \\
z_2 &=& L - z_1.
\end{eqnarray}

Clearly, we can combine Eqs. (\ref{eq:x}) and
(\ref{eq:cavitydisptilt}) to arrive at an equation directly relating
mirror tilt to cavity displacement and tilt: 
\begin{equation}
\left\llbracket \begin{array}{c}
a \\
\alpha \end{array} \right\rrbracket = \frac{1}{1-g_1g_2}
\left\llbracket \begin{array}{cc}
g_2z_2 + z_1 & z_2 + g_1z_1\\
-g_2 + 1 & -1 + g_1\end{array} \right\rrbracket
\left\llbracket \begin{array}{c}
\theta_1\\
\theta_2 \end{array} \right\rrbracket.
\label{eq:cavitydisptilt_mirrorangle}
\end{equation}



\section{Power in a misaligned cavity}
\label{sec:cavitypower}
I'll show how to calculate the power in a cavity as a function of
cavity axis displacement and tilt. Combined with the results of
Eq. \ref{eq:cavitydisptilt_mirrorangle} we determine how the power
build-up in a cavity depends on a single mirror's angular
displacement.
% %This can be compared with data to calibrate
% %the optical lever error signals as will be shown in Section \ref{sec:oplevcal}. 

The field of a lowest-order Gaussian laser beam along one axis at the beam waist is:
\begin{equation}
\psi(x) = U_0(x) = \left[ \frac{2}{\pi w_0^2} \right]^{1/4} \exp{\left[-\left[\frac{x}{w_0}\right]^2\right]}
\end{equation}
where $w_0$ is the beam waist radius and $U_0$ is the lowest-order
Hermite polynomial. The Hermite polynomials are orthonormal,
ie. $\langle U_i \mid U_j \rangle = \delta_{ij}$. For example, the
next to lowest order polynomial is:
\begin{equation}
U_1(x) = \left( \frac{2}{\pi w_0^2} \right)^{1/4} \frac{2x}{w_0}
\exp{[-(x/w_0)^2]} = \frac{2x}{w_0} U_0(x)
\end{equation}

\subsection{Displaced cavity}
The field of a cavity with a \emph{displaced} z-axis at the cavity waist is:
\begin{align}
\psi \prime (x) =& \psi(x-a) \\
 =& U_0(x-a) \\
 =& c_0U_0(x) + c_1 U_1(x) + c_2 U_2(x) + ...\\
\end{align}
where $a$ is the displacement of the axis and $c_i$ are constants.

%\subsubsection{Power} 
We want to know $c_0$, the projection of the displaced cavity field onto the beam field: 
\begin{eqnarray}
c_0 &=& \langle \psi \mid \psi \prime \rangle \\
&=& \int_{-\infty}^\infty \psi(x) \psi \prime(x) \, dx \\
 &=& \exp{[-a^2/2 w_0^2]}
\label{eq:c_0}
\end{eqnarray}
The power in this mode is the square of the overlap of the two fields: 
\begin{eqnarray}
P_0&=& \left| \langle \psi \mid \psi \prime \rangle \right| ^2\\ 
&=& \exp{[-[a/w_0]^2]} 
\end{eqnarray}

%\subsubsection{$U_1$ field}
For the purpose of wavefront sensing, we need to know the amplitude,
$c_1$, of the first order $U_1$ field. This can be approximated as
demonstrated in Anderson \cite{Anderson1984Alignment} using
the Taylor series expansion of the exponential in $\psi \prime(x) =
U_0(x-a)$, assuming a displacement $a$ that's small compared to waist
size $w_0$.
\begin{eqnarray}
\psi \prime(x) &=& \left[ \frac{2}{\pi w_0^2} \right]^{1/4}
\exp{\left[-\left[\frac{x-a}{w_0}\right]^2\right]} \\
&=& \left[ \frac{2}{\pi w_0^2} \right]^{1/4} \left[1 -
  \left[\frac{x-a}{w_0} \right]^2 + \mathcal{O}(a^4) \right] \\
&=& \left[ \frac{2}{\pi w_0^2} \right]^{1/4} \left[ \frac{2xa}{w_0^2}
  \left[1-\frac{x^2}{w_0^2} + ... \right] + \left[ 1 -
    \frac{x^2}{w_0^2} + \frac{1}{2} \frac{x^4}{w_0^4}
    - ... \right] + \mathcal{O}(a^2) \right] \\
&=& \left[ \frac{2}{\pi w_0^2} \right]^{1/4} \left[1 +
  \frac{2xa}{w_0^2} + \mathcal{O}(a^2) \right] \exp{\left[ -\left[\frac{x}{w_0}\right]^2
  \right]} \\
&=& U_0(x) + \frac{a}{w_0} U_1(x) + ...
\end{eqnarray}
Notice that here we find $c_0 = 1$, which is consistent with the
exact result of Eq.~\ref{eq:c_0} when we apply our $a^2 \approx 0$
approximation. We find that the amplitude of the first order
Hermite-Gauss field for a displaced cavity is
\begin{equation}c_1 = a/w_0.
\end{equation}



\subsection{Tilted cavity}
The field of a cavity with a \emph{tilted} z-axis at the cavity waist is a tad more complex to derive. We assume the tilt, $\alpha$, is small such that $\sin{\alpha} \approx \alpha$ and $\cos{\alpha} \approx 1$. Also, we assume the beam divergence angle, $\theta_0=\lambda / \pi w_0$, is small such that the wavefronts near the waist can be considered parallel to one another. 

Here, the important quantity to consider is the phase of the cavity field at the cross-section of the beam waist.  The phase is either advanced or retarded compared to that of the beam:
\begin{eqnarray}
\psi \prime(x) &=& \psi(x \prime) \exp{[-i k z \prime]} \\
 & \approx& \psi(x \cos{\alpha}) \exp{[-i k x \sin{\alpha}]} \\
 &\approx& \psi(x) \exp{[- i k x \alpha]} \\
 &=& U_0(x) \exp{[- i k x \alpha]}
\label{eq:psi_tilted}
\end{eqnarray}
where $k=2\pi / \lambda$ and $\lambda$ is the wavelength of the laser light.

%\subsubsection{Power} 
The overlap of the fields of the beam and tilted cavity is $ \exp{[-\alpha^2/2 \theta_0^2]} $. Therefore the power is:
\begin{equation}
P_0 = \exp{[-(\alpha/\theta_0)^2]}.
\end{equation}

%\subsubsection{$U_1$ field}
An expansion of the exponential in Eq. \ref{eq:psi_tilted} for a small
tilt $\alpha$ gives:
\begin{eqnarray}
\psi \prime(x) &=& U_0(x) [1 + i k x \alpha + \mathcal{O}(\alpha^2) ] \\
&=& U_0(x) + \frac{i k \alpha w_0}{2} U_1(x) + \mathcal{O}(\alpha^2).
\end{eqnarray}
Therefore, the amplitude of the first order Hermite-Gauss field for a
tilted cavity is 
\begin{equation}
c_1 = i k \alpha w_0 / 2.
\end{equation}


\subsection{Displaced and tilted cavity}
The most general case, of course, is to have a cavity axis that is both displaced \emph{and} tilted at the beam waist:
\begin{equation}
\psi \prime(x) = \psi(x-a) \exp{[-i k (x-a) \alpha]}.
\end{equation}
We find:
\begin{equation}
\langle \psi \mid \psi \prime \rangle = \exp{\left[- \frac{a^2}{2 w_0^2} \right]} \exp{\left[-\frac{\alpha^2}{2 \theta_0^2}\right]} \exp{\left[- \frac{i a \alpha}{x_0 \theta_0}\right]}
\end{equation}
and
\begin{equation}
P_0 = \exp{\left[- \frac{a^2}{w_0^2} \right]}
\exp{\left[-\frac{\alpha^2}{\theta_0^2}\right]}.
\label{eq:pwr_disptilt}
\end{equation}



\section{Initial DC alignment of the interferometer}
\label{sec:initial_alignment}
After any kind of in-vacuum work, the DC alignment of the mirrors is
usually too poor for the interferometer to lock. A bootstrapping
process of tweaking the alignment by hand is necessary, assuming the
mirrors start out pointing in generally the right direction, as is
usually the case. As pointed out in \ref{sec:alignment_overview}, the
QPDs at the end stations are the fixed reference points for the
overall alignment, so this process begins with making sure the light
reaches them. We then adjust the rest of the mirrors to maximize power
build-up in the arms and to maximze spatial overlap of the light
reflected from each arm.

An outline of the process is presented here. ``Misalign'' means to
intentionally point a mirror so far away from any known good positions
as to eliminate it from the configuration.  ``Align'' and ``restore''
mean to bring a mirror or configuration to the best known
position(s). Centering the beam on a mirror is accomplished by using
the suspension cage surrounding the mirror as a reference. Camera
images and QPD readback provide the signals used for beam centering. 

\begin{itemize}
\item[] \textbf{X-arm} \vspace{-10pt}
\item restore the x-arm (misalign RM, ITMY, and ETMY, align ITMX and ETMX) \vspace{-10pt}
\item use ITMX to center the beam on QPDX \vspace{-10pt}
\item use ETMX to center the beam on ITMX \vspace{-10pt}
\item with x-arm locked, use MMT3 to maximize the x-arm power build-up (NPTRX, can expect
  about 95\%) \vspace{-10pt}
\item save the MMT3, ITMX, and ETMX alignment settings 
\item[] \textbf{Y-arm} \vspace{-10pt}
\item restore the y-arm (misalign ITMX and ETMX, align ITMY and ETMY) \vspace{-10pt}
\item use ITMY to center the beam on QPDY \vspace{-10pt}
\item use ETMY to center the beam on ITMY \vspace{-10pt}
\item with y-arm locked, use BS to maximize the y-arm power build-up (NPTRY, can expect
  about 90\%) \vspace{-10pt}
\item save the BS, ITMY, and ETMY alignment settings 
\item[] \textbf{Relative x-arm and y-arm} \vspace{-10pt}
\item note AS beam position on camera while toggling between x-arm and
  y-arm locks\vspace{-10pt}
\item use ETMs to align the two AS beams \vspace{-10pt}
\item restore the Michelson (misalign ETMs, align ITMs) \vspace{-10pt}
\item use BS to make AS port as dark as possible \vspace{-10pt}
\item re-do y-arm alignment if ambitious 
\item[] \textbf{Recycling mirror} \vspace{-10pt}
\item restore the PRM (misalign ETMs, align ITMs and RM)
  \vspace{-10pt}
\item use RM to center beam on ETMY cage 
\item[] \textbf{Restore full interferometer--off you go!}
\end{itemize}



\section{Wavefront sensors}
A wavefront sensor is a quadrant photodiode equipped with RF
electronics. 

\subsection{Photodiodes}
\label{sec:pds}
The basic photoconductive photodiode is depicted in Fig. \ref{}. A
negative voltage applied to the anode, represented by the line, does
not yield any current across the photodiode until electrons are
released by the energy of photons striking the diode. Electrons flow
in the direction of the cathode which is connected to ground,
producing a current
\begin{equation}
i_\gamma = \frac{q_e}{h \nu} P \epsilon
\end{equation}
where $q_e$ is the electron charge, $h$ is Planck's constant, $\nu$ is
the frequency of the incident light, $P$ is the power of the light,
and $\epsilon$ is the quantum efficiency of the diode. The quantity
$q_e/h \nu$ is known as the \emph{responsivity}. For LIGO where
$\lambda = 1064$ nm, the responsivity is $0.86$ A/W.



\section{Channel names}
It is often confusing why there are so many channels with almost
identical names. And just what those names mean can also sometimes be
elusive. The best advice I have to offer about how to figure out for
one's self what the root of the channel name conveys about the data it
represents is to browse the MEDM screens. The MEDM screens pictorially
show the flow of which channels are derived from others, and whether
they sit before or after an excitation point, for example. They also
sometimes offer hints as to exactly what calculation is done under the
scene in the front-end code to compute new channels. 

The topic I want to cover here is the difference between the
\texttt{OUT}, \texttt{OUT16}, \texttt{OUTPUT}, and \texttt{OUTMON}
channel name extenstions. They appear in all of the subsystems, and it
is crucial to realize that some are good for analyzing data and some
good for getting only a sense in real-time of what's going on. Some
are typically recorded to disk, and some are not. These concepts
extend to other common name extensions as well. 

