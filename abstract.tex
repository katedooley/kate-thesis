\begin{abstract}
  A prediction of Einstein's general theory of relativity,
  gravitational waves are perturbations of the flat space-time
  Minkowski metric that travel at the speed of light. Indirectly
  measured by Hulse and Taylor in the 1970s through the energy they
  carried away from a binary pulsar system, gravitational waves have
  yet to be directly detected. The Laser Interferometer
  Gravitational-wave Observatory (LIGO) is part of a global network of
  gravitational-wave detectors that seeks to directly detect
  gravitational waves and to study their sources. 

  LIGO operates on the principle of measuring the gravitational wave's
  physical signature of a strain, or relative displacement of inertial
  masses. An extremely small effect whose biggest of expected
  transient signals on Earth is on the order of one part in $10^{23}$,
%  \textcolor{blue}{(verify this!)}, 
gravitational-wave strain can only
  be measured by detectors so sensitive to displacement as to brush
  into the realm of quantum physics. To improve their 
  sensitivities and to demonstrate advanced technologies, the LIGO
  observatories in Hanford, WA and Livingston, LA underwent an upgrade
  between fall 2007 and summer 2009 called Enhanced LIGO. This
  dissertation focuses on the experimental challenges of one of the
  goals of the upgrade: operating at an increased laser power.

I present the design and characterization of two of the interferometer
subsystems that are critical for the path towards higher laser power:
the Input Optics and the Angular Sensing and Control.
%\textcolor{blue}{Write this final paragraph describing work.}

\end{abstract}
