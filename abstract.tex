\begin{abstract}
  A prediction of Einstein's general theory of relativity,
  gravitational waves (GWs) are perturbations of the flat space-time
  Minkowski metric that travel at the speed of light. Indirectly
  measured by Hulse and Taylor in the 1970s through the energy they
  carried away from a binary pulsar system, gravitational waves have
  yet to be detected directly. The Laser Interferometer
  Gravitational-wave Observatory (LIGO) is part of a global network of
  gravitational-wave detectors that seeks to detect directly
  gravitational waves and to study their sources.

  LIGO operates on the principle of measuring the gravitational wave's
  physical signature of a strain, or relative displacement of inertial
  masses. An extremely small effect whose biggest of expected
  transient signals on Earth is on the order of one part in $10^{23}$,
%  \textcolor{blue}{(verify this!)}, 
  gravitational-wave strain can only be measured by detectors so
  sensitive to displacement as to encounter the effects of quantum
  physics. To improve their sensitivities and to demonstrate advanced
  technologies, the LIGO observatories in Hanford, WA and Livingston,
  LA underwent an upgrade between fall 2007 and summer 2009 called
  Enhanced LIGO. This study focuses on the experimental challenges of
  one of the goals of the upgrade: operating at an increased laser
  power.

  I present the design and characterization of two of the
  interferometer subsystems that are critical for the path towards
  higher laser power: the Input Optics (IO) and the Angular Sensing
  and Control (ASC) subsystems. The IO required a new design so its
  optical components would not be susceptible to high power effects
  such as thermal lensing or thermal beam drift. The ASC required a
  new design in order to address static instabilities of the arm
  cavities caused by increased radiation pressure. In all, I
  demonstrate the capability of an interferometric GW detector to
  operate at several times the highest of laser powers previously
  used.
\end{abstract}
