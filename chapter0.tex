\chapter{Purpose of this work}
The purpose of this work is to demonstrate the capability of an
interferometric gravitational wave detector to operate at several
times the highest of laser powers previously used. From a na\"ive
standpoint, more power is desirable since strain sensitivity improves
by $\sqrt{P}$ in the high frequency ($>$ 200~Hz) shot-noise-limited
region. However, as detectors become more sensitive at low frequencies
($<$ 70 Hz) in the years to come, radiation pressure noise will become
the dominant noise source, making high laser power operation a design
trade-off. Currently, seismic noise limits low frequency sensitivity,
so exploring the technical world of increasing the laser power is a
fruitful adventure.

%  (and to pave the way
% forward to realizing Advanced LIGO's goals of )






Once the theory of radiation pressure's effect on angular mechanical
transfer functions was fully appreciated and published in 2006 by
Sidles and Sigg (Ref. \cite{Sidles2006Optical}), the concern arised that radiation
pressure might be the factor limiting Initial LIGO's ability to
increase the input power. Although radiation pressure torques were
proven to not be the cause of difficulty in operating at higher powers
\cite{Hirose2010Angular}

My work ...





Operation of Initial LIGO was limited to 7~W input power due to
uncontrolled radiation pressure torque instabilities in the arm
cavities. \textcolor{blue}{WRONG!!!}. Explained theoretically by Sidles and Sigg
\cite{Sidles2006Optical}, measured experimentally by Hirose
\cite{Hirose2010Angular}, and modeled numerically by Barsotti
\cite{Barsotti2009modeling}, the effect of radiation pressure torque
on angular alignment needed to be addressed in practice in order for
Enhanced LIGO to succeed in operating at powers greater than 7~W. We
present the re-designed Angular Sensing and Control (ASC) system as
implemented on the Enhanced LIGO detectors and show results of its
performance with up to 20~W input power, demonstrating good agreement
between theory, experiment and model.

The use of more power also complicates interferometer operations
because of thermal effects. The optics which condition the laser for
use in the interferometer experienced degradation in their performance
in Initial LIGO as the result of absorbing too much heat. Less
absorptive optical components were chosen with the goal of conquering
thermal issues at the source, and changes were made to the
architecture of the Input Optics to compensate for any residual
effects. We present the re-designed Input Optics and their thermal
performance with up to 30~W input power.

