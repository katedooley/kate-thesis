\chapter{Purpose of this work}
The purpose of this work is to demonstrate the capability of an
interferometric gravitational wave detector to operate at several
times the highest of laser powers previously used. From a na\"ive
standpoint, more power is desirable since strain sensitivity improves
by $\sqrt{P}$ in the high frequency ($>$ 200~Hz) shot-noise-limited
region. However, as detectors become more sensitive at low frequencies
($<$ 70 Hz) in the years to come, radiation pressure noise will become
the dominant noise source, making high laser power operation a design
trade-off. Currently, seismic noise limits low frequency sensitivity,
so exploring the technical world of increasing the laser power is a
fruitful adventure.

%Although offering an improvement to the shot-noise-limited sensitivity, 
More power introduces a host of radiation pressure and thermally
induced side effects that must be addressed for effective
interferometer operation. Concerns about the practical difficulties of
handling high power effects abounded during Initial LIGO when
operating at the design power of 10~W proved more difficult and less
straight-forward than expected. To achieve the Advanced LIGO design
sensitivity, a whopping 160~W of input power is needed. Without an
understanding of the thermal and radiation pressure problems at 10~W,
Advanced LIGO becomes all the more daunting of a goal. 

The work presented in this dissertation was carried out during
Enhanced LIGO to verify and investigate the predicted and unforeseen
effects of as much as 25~W of laser power. I present the design and
the measurements I made of the performance of two of the
interferometer subsystems that are affected by an increase in laser
power: the Input Optics and the Angular Sensing and Control. I show
that the thermal and radiation pressure effects on these subsystems
are well understood and should not pose \textcolor{blue}{unsolvable,
  pick different word} problems for
Advanced LIGO.
% As an
% epilogue, I make projections of what can be expected for Advanced LIGO
% based on the results of my experimental studies. 


\section{The Input Optics high power story} 
The performance of the Initial LIGO Input Optics degraded as the
result of absorbing too much heat while the input power ramped up to
7~W. Particular issues that needed to be addressed for any further
increase in power included thermal steering of the beam rejected by
the interferometer, a decrease in the optical isolation, and thermal
lensing that drove the spatial mode of the beam directed at the
interferometer away from optimal. I describe the design of the
improved Input Optics which includes less absorptive optical
components in order to conquer thermal issues at the source and
changes to the design architecture that compensate for any residual
effects. I also present the set of measurements I made to characterize
their performance with up to 30~W input power. I show that we can
expect the Input Optics to perform well for Advanced LIGO.



\section{The Angular Sensing and Control high power story}
Radiation pressure creates optical torques, a concept first recognized
in 1991 by Solimeno et al. \cite{Solimeno1991FabryPerot}. However, the
theory of radiation pressure's effect on angular mechanical transfer
functions was not fully appreciated and published until 2006 by Sidles
and Sigg \cite{Sidles2006Optical}. The concern arised that radiation
pressure might be the factor limiting Initial LIGO's ability to
increase the input power. Eiichi Hirose showed that the optical torque
was present and measurable, but that it was \emph{not} limiting
Initial LIGO's power \cite{Hirose2010Angular}. The concern of the
optical torque's role in cavity dynamics shifted to Enhanced and
Advanced LIGO, which were designed to operate at four times and 20
times the laser power of Initial LIGO, respectively. Lisa Barsotti
developed a numerical model of the angular sensing and controls for
Enhanced LIGO, specifically including radiation pressure torque. She
showed that, in principle, the radiation pressure torque can be
controlled without detrimental consequences to the sensitivity of the
detector \cite{Barsotti2009Modeling}. I implemented Barsotti's
theoretical control scheme and measured its performance with up to
20~W of input power, demonstrating a thorough understanding of the
principles at work and providing confidence in the ability to control
radiation pressure torques in Advanced LIGO. I also improved upon
Hirose's measurement of the optical angular (anti-)spring. And through
post-analysis of angular data, I demonstrate the potential of a
technique that may be used in Advanced LIGO for reducing the angular
control signals.



\section{Outline of this dissertation}
The organization of this
dissertation is as follows. Chapter~2 provides the astrophysical
motivation...
