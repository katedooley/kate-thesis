% user defined commands %
% Here is where you define optional commands such as macros, new commands,
% and new environments to be used in your paper


% optional command to prevent a word from breaking across a line %
\hyphenchar\font=-1

% UF template specific commands %

% Commands to produce proper bullet list
\newlength{\widthOfItem}
\let\Itemize=\itemize
\let\endItemize=\enditemize
\renewenvironment{itemize}{%
	\begin{Itemize}
		\setlength{\itemsep}{0.5\baselineskip}
		\setlength{\labelwidth}{2em}
		\setlength{\listparindent}{.32in}%
		\setlength{\leftmargin}{.32in}
		\setlength{\rightmargin}{0in}
		\settowidth{\widthOfItem}{\labelitemi}
		\setlength{\labelsep}{\leftmargin-\widthOfItem}
		\renewcommand{\labelitemii}{--}
		\singlespacing}{%
	\end{Itemize}}

% Commands for enumerated lists
% \newcounter{ufcount}
% \renewenvironment{enumerate}{
% 	\setcounter{ufcount}{1}
% 	\begin{list}{%
% 		\arabic{ufcount}.}{%
% 		\setlength{\itemsep}{0.5\baselineskip}
% 		\setlength{\labelsep}{.24in}
% 		\setlength{\labelwidth}{.1in}
% 		\setlength{\listparindent}{.34in}%
% 		\setlength{\leftmargin}{.35in}
% 		\setlength{\rightmargin}{0in}
% 		\usecounter{ufcount}}
% 		\singlespacing
% 	}{\end{list}}

% Shorcut commands for misc stuff %

% new command set that should be inserted before short captions (<2 lines), in order to left align the caption as per the editorial office mandate %

%\makeatletter
%\long\def\@makecaption#1#2{%
%  \vskip\abovecaptionskip
%  \sbox\@tempboxa{#1: #2}%
%  \ifdim \wd\@tempboxa >\hsize
%    #1: #2\par
%  \else
%    \global \@minipagefalse
%    \hb@xt@\hsize{\box\@tempboxa\hfil}%
%  \fi
%  \vskip\belowcaptionskip}
%\makeatother


% shortcut for setting up inserting \prime command in mathmode to avoid errors %
\newcommand{\p}{^{\prime}}

% shortcuts for prime color text
% \newcommand{\red}{\textcolor[rgb]{1.00,0.00,0.00}}
% \newcommand{\green}{\textcolor[rgb]{0.00,1.00,0.00}}
% \newcommand{\blue}{\textcolor[rgb]{0.00,0.00,1.00}}

%=======================================================
% math notation

% Spaces
\newcommand{\field}[1]{\ensuremath{\mathbb{#1}}}
\newcommand{\Kfield}{\field{K}}
\newcommand{\reals}{\field{R}}
\newcommand{\complex}{\field{C}}
\newcommand{\integers}{\field{Z}}
\newcommand{\naturalNumbers}{\field{N}}
\newcommand{\Space}[1]{\ensuremath{\mathcal{#1}}}

% theorem like environments
\newtheorem{theorem}{Theorem}[section]
\newtheorem{proposition}[theorem]{Proposition}
\newtheorem{lemma}[theorem]{Lemma}
\newtheorem{corollary}[theorem]{Corollary}
\newtheorem{definition}{Definition}
\newtheorem{example}{Example}
\newtheorem{remark}{Remark}
\newtheorem{question}{Question}
\newcounter{mypropcounter}[subsection]
\newtheorem{property}[mypropcounter]{Property}
\newcounter{myclaimcounter}[subsection]
\newtheorem{claim}[myclaimcounter]{Claim}

% object notation
\newcommand{\mtx}[1]{\ensuremath{\mathbf{#1}}}
\newcommand{\x}{\mtx{x}}
\newcommand{\randomVariable}[1]{\ensuremath{\mathbf{#1}}}
\newcommand{\randomProcess}[2][t]{\ensuremath{\mathbf{#2}(#1)}}
\newcommand{\cdotdot}{\ensuremath{(\cdot, \cdot)}}
\newcommand{\fdot}{\ensuremath{(\cdot)}}

% Operators
\newcommand{\E}[2][]{\ensuremath{\mathrm{E}_{#1}\left\{#2\right\}}}
\newcommand{\ip}[2]{\ensuremath{\left\langle #1 , #2 \right\rangle}}
\newcommand{\innerProd}[2]{\ip{#1}{#2}}
\newcommand{\norm}[1]{\ensuremath{\left\|#1\right\|}}
\newcommand{\abs}[1]{\ensuremath{\left|#1\right|}}

%=======================================================
% Spike trains

\newcommand{\eventSpace}{\ensuremath{\mathcal{T}}}
\newcommand{\probabilitySpace}{\ensuremath{(\Omega,\mathcal{B},P)}}
\newcommand{\spkTrain}[1]{\ensuremath{S_{#1}}}
\newcommand{\spkTrainIndexSet}{\ensuremath{\mathcal{I}}}
\newcommand{\spkTrainSpace}{\ensuremath{\mathcal{S}(\eventSpace)}}
\newcommand{\spkTrainSet}[2][]{\ensuremath{\{t^{#1}_{#2} \in \eventSpace:#2 = 1, \ldots, N_{#1}\}}}

% spike train measures
\newcommand{\dVP}{\ensuremath{d_{\mbox{\footnotesize VP}}}}
\newcommand{\dvR}{\ensuremath{d_{\mbox{\footnotesize vR}}}}
\newcommand{\dCS}{\ensuremath{d_{\mbox{\footnotesize CS}}}}
\newcommand{\dCC}{\ensuremath{d_{\mbox{\footnotesize CC}}}}

%=======================================================
% Point processes, kernel and CIP macros

\newcommand{\eventSpace}{\ensuremath{\mathcal{T}}}
\newcommand{\probabilitySpace}{\ensuremath{(\Omega,\mathcal{B},P)}}
\newcommand{\pp}[1]{\ensuremath{p_{#1}}}
\newcommand{\pointProcessSpace}{\ensuremath{\mathcal{P}(\eventSpace)}}
\newcommand{\pointProcessSet}[2][]{\ensuremath{\{t^{#1}_{#2} \in \eventSpace:#2 = 1, \ldots, N_{#1}\}}}
\newcommand{\pointProcessWav}[2][t]{\ensuremath{s_{#2}(#1)}}
\newcommand{\filteredPointProcess}[2][t]{\ensuremath{q_{#2}(#1)}}
\newcommand{\quantt}{\ensuremath{\check{t}}}

\newcommand{\kernel}[1]{\ensuremath{k#1}}
\newcommand{\kernelInputSpace}{\ensuremath{\eventSpace\times\eventSpace}}
\newcommand{\kernelHilbertSpace}{\ensuremath{\HilbertSpace_{\kernel{}}}}
\newcommand{\dimKernelSpace}{\ensuremath{|\kernelHilbertSpace|}}
\newcommand{\kernelMapping}[2][]{\ensuremath{\Phi^{#1}_{#2}}}
\newcommand{\kernelEigenfunc}[1]{\ensuremath{\phi_{#1}}}
\newcommand{\innerProdHk}[2]{\ensuremath{\left\langle#1, #2\right\rangle_{\kernelHilbertSpace}}}

\newcommand{\CIP}{\ensuremath{V}}
\newcommand{\CIPpp}[2]{\ensuremath{\CIP(\pp{#1},\pp{#2})}}
\newcommand{\CIPhat}{\ensuremath{\hat{\CIP}}}
\newcommand{\CIPkernelFunc}[2][\cdot]{\CIP(#2,#1)}%\ensuremath{V_{#2}(#1)}}
\newcommand{\CIPkernelInputSpace}{\ensuremath{\pointProcessSpace\times\pointProcessSpace}}
\newcommand{\CIPHilbertSpace}{\ensuremath{\HilbertSpace_{\CIP}}}
%\newcommand{\dimCIPkernelSpace}{\ensuremath{|\VCIPHilbertSpace|}}
\newcommand{\CIPkernelMapping}[1]{\ensuremath{\Psi_{#1}}}
\newcommand{\CIPkernelEigenfunc}[1]{\ensuremath{\psi_{#1}}}
\newcommand{\innerProdHv}[2]{\ensuremath{\left\langle#1, #2\right\rangle_{\CIPHilbertSpace}}}

\newcommand{\CIK}{\ensuremath{I}}
\newcommand{\HS}[1][]{\ensuremath{\mathcal{H}_{#1}}}
\newcommand{\HilbertSpace}[1][]{\ensuremath{\mathcal{H}_{#1}}}
\newcommand{\CIKHS}{\HilbertSpace[\CIK]}
\newcommand{\CIKst}[2]{\ensuremath{\CIK(\spkTrain{#1},\spkTrain{#2})}}
\newcommand{\mCIK}{\ensuremath{I}}
\newcommand{\mCIKst}[2]{\ensuremath{\mCIK(\spkTrain{#1},\spkTrain{#2})}}
\newcommand{\mCIKpp}[2]{\ensuremath{\mCIK(\pp{#1},\pp{#2})}}
\newcommand{\mCIKhat}{\ensuremath{\hat{\mCIK}}}
\newcommand{\mCIKernelFunc}[2][\cdot]{\mCIK(#2,#1)}
%\newcommand{\mCIKInputSpace}{\ensuremath{\pointProcessSpace\times\pointProcessSpace}}
\newcommand{\mCIKHS}{\HilbertSpace[\mCIK]}
%\newcommand{\dimCIPkernelSpace}{\ensuremath{|\VCIPHilbertSpace|}}
\newcommand{\mCIKernelMapping}{\ensuremath{\Lambda}}
\newcommand{\mCIKernelEigenfunc}{\ensuremath{\psi}}
\newcommand{\ipMCIK}[2]{\ensuremath{\ip{#1}{#2}_{\mCIKHS}}}

\newcommand{\RKHSpcaeigfcn}{\ensuremath{\psi}}
\newcommand{\RKHSpcaeigval}{\ensuremath{\rho}}

%=======================================================
% other commands & environments

\newcommand{\equationref}[1]{equation~\ref{#1}}
\newcommand{\Equationref}[1]{Equation~\ref{#1}}
\newcommand{\figureref}[1]{figure~\ref{#1}}
\newcommand{\Figureref}[1]{Figure~\ref{#1}}
\newcommand{\tableref}[1]{table~\ref{#1}}
\newcommand{\Tableref}[1]{Table~\ref{#1}}

\newcommand{\etal}{et al.}
\newcommand{\figcontent}[2][tbp]{\begin{figure}[#1]\centering #2 \end{figure}}
\newcommand{\tabcontent}[2][tbp]{\begin{table}[#1]\centering #2 \end{table}}
\newenvironment{cfig}[1][tbp]{\begin{figure}[#1]\centering}{\end{figure}}

