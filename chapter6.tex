\chapter{ANGULAR SENSING AND CONTROL CHARACTERIZATION AND PERFORMANCE
  IN THE RADIATION PRESSURE EIGENBASIS}
\label{ch:characterization}




% In this chapter I show how the angular displacements are sensed and
% why control filters implemented in the eigenbasis of radiation
% pressure torques is best. 

The primary aspect of the Enhanced LIGO ASC upgrade was to switch the
control servo from the sensor basis to the natural radiation pressure
eigenmode basis, and to keep the contamination to DARM at a minimum. I
present in this chapter the design of the new basis and measurements I
made to characterize it and its effect on DARM.

A critical aspect of the characterization of any system is to
calibrate the data in physical units to facilitate comparison to
models and to make meaningful statements. Because the LIGO data is
collected digitally, the units are naturally in digital counts. Part
of my work was therefore to calibrate each of the relevant ASC
channels to physical units. I include the details of the calibrations
in Appendix~\ref{ch:ASCcal}.

Also, as shown in Section~\ref{sec:oplevspectra} almost all of the mirror
motion is in fact due to the ground. Therefore, the measurements I
made of the ASC are very sensitive to the particular state of seismic
noise. I include in Appendix \ref{sec:groundmotion} seismic spectra
from the time of each measurement I present.


\section{The ASC Change of Basis}
The basis for angular control in Initial LIGO was that of the physical
sensors. The wavefront sensors are located such that they sense common
and differential ETM and common and differential ITM angular
motions. The input matrix was diagonal and the control filters were
designed to feed back to those sets of motions via the output
matrix. This does not, however, lend itself to easily handling
radiation pressure torque. Because the sensor basis is not the
radiation pressure eigenbasis of Section \ref{sec:eigenbasis}, each
control servo handled combinations of the soft and hard modes. Due to
radiation pressure, each mirror has two resonances, a more complicated
plant than that offered by the eigenbasis which has a single resonance
for each mode.  The change in control basis from Initial LIGO to
Enhanced LIGO was a rather straightforward operation of changing only
the ASC input matrix and the ASC output matrix as described in the
following subsections. Refer to Figure~\ref{fig:ASCcontrolservo} for
the locations of these matrices in the servo.

Although we changed the meaning of the WFS control signals for
Enhanced LIGO, we did not update the convention for their naming as
seen in the channel names: WFS1, WFS2A, WFS2B, WFS3, and WFS4. This
can be misleading because of the similarity to the sensor names
(WFS1Q, WFS2I, WFS2Q, WFS3I, and WFS4I), potentially leading one to
assume a one-to-one correspondence. Although a one-to-one
correspondence was accurate for Initial LIGO, it is not true for
Enhanced LIGO. Therefore, for clarity in this dissertation, I refer to
the WFS control signals by the radiation pressure eigenbasis degrees
of freedom (DOFs) they represent:
\begin{itemize}
\item differential soft (dSoft) \vspace{-10 pt}\\
\item common soft (cSoft) \vspace{-10 pt}\\
\item differential hard (dHard) \vspace{-10 pt}\\
\item common hard (cHard) \vspace{-10 pt}\\
\item recycling mirror (RM)
\end{itemize}
Differential and common refer to the comparison of the sign of the mode in each arm.

%\textcolor{blue}{make block figure!}



\subsection{WFS Input Matrix}
\label{sec:inputmatrix}
The optical gain for each of the optics' motions is not concentrated
at any one particular port, although it can appear more strongly in
one location compared to another. In order to make use of as much of
the information as possible, we must use all sensors that witness a
particular motion. However, when there are multiple signals at a
particular place, the WFS error signals tell us the sum of all optical
gains at its location. The amount of optical gain at each detector for
each motion at one particular frequency forms the \emph{sensing
  matrix}. The inverse of the sensing matrix is known as the
\emph{input matrix}, which tells how to take the appropriate weighted
sum of signals in order to reconstruct a particular motion. The
procedure for measuring the sensing matrix is as follows:
\begin{itemize}
\item excite one of the mirrors (or specific combination of mirrors)
  at frequency $f$ \vspace{-10pt}
\item demodulate each of the WFS signals at $f$ \vspace{-10pt}
\item normalize to the phase of the excitation readback \vspace{-10pt}
\item repeat for each mirror (or sets of mirrors)
\end{itemize}
A key aspect of the measurement is that a notch filter at frequency
$f$ is engaged so the control servo does not suppress our excitation.

An example calibrated sensing matrix in the radiation pressure
eigenbasis as taken during a 10~W lock is shown in
Table~\ref{table:sensing}.\footnote{Refer to Appendix
  \ref{sec:cal_sensing} and \ref{sec:cal_opticalgain} for a
  description of how the calibration is done.} Rows represent
excitation and columns are the wavefront sensors. Before inverting to
create the input matrix, the smallest of the elements (which are more
or less equivalent to the elements for which optical gain is also
expected to be weak), are artificially set to zero. This avoids the
contamination of strong signals by those with weak signal-to-noise
ratios. The elements that remain after this process are highlighted by
boxes. Note that the sensing matrix is in fact composed of two
sub-matrices: one for the differential degrees of freedom, and one for
the common degrees of freedom. Also, WFS1Q has particularly strong
signal compared to the other wavefront sensors. We see in
Section~\ref{sec:WFSolgs} that this allows us to provide much more
control to the differential soft mode compared to the other modes.

\begin{table}
\centering
\caption[WFS optical gain matrix]{Angular optical gain at 9.7~Hz in units of
  Volts per degree of freedom microradian (pitch). Numbers in
  \textcolor{gray}{gray} are the measurement results that have
  coherences less than 0.9. Boxes highlight the elements actually used
  in the control system. All other elements are set to zero.}
% April 14, 2010
\begin{tabular}{l l l l l l}
\hline
WFS1Q & WFS2Q & WFS2I & WFS3I & WFS4I &  \\
\hline
\fbox{2.0}   & \textcolor{gray}{0.03} &\textcolor{gray}{0.06} & \textcolor{gray}{-0.008}  &  \textcolor{gray}{0.01} & dSoft \\
\fbox{0.31}  & \fbox{-0.03} &\textcolor{gray}{-0.04} &  \textcolor{gray}{0.002} & \textcolor{gray}{-0.01} & dHard \\
\textcolor{gray}{0.02} & \textcolor{gray}{-0.01} &  \fbox{0.18} & \fbox{\textcolor{gray}{-0.02}} &  \fbox{\textcolor{gray}{-0.10}} & cSoft \\
\textcolor{gray}{0.17} & \textcolor{gray}{-0.01} & \fbox{-0.21} &  \fbox{\textcolor{gray}{0.007}} & \fbox{-0.12} & cHard \\
\textcolor{gray}{0.09} & \textcolor{gray}{-0.01} & \fbox{-0.21}  &  \fbox{0.04} & \fbox{-0.21} & RM \\
\hline
\end{tabular}
\label{table:sensing}
\end{table}

% \begin{table}
% \centering
% \caption[WFS optical gain matrix]{Optical gain at 9.7~Hz in units of WFS counts / dof radian
%   (pitch). Numbers in \textcolor{gray}{gray} are the measurement
%   results that have coherences less than 0.9. Boxes highlight the
%   elements actually used in the control system. All other elements are
%   set to zero. \textcolor{blue}{change to Volts!}} 
% \begin{tabular}{l l l l l l}
% \hline
% WFS1Q & WFS2I & WFS2Q & WFS3I & WFS4I &  \\
% \hline
% \fbox{5.8e+12}   & \textcolor{gray}{3.3e+09} &\textcolor{gray}{6.7e+09} & \textcolor{gray}{-1.3e+09}  &  \textcolor{gray}{2.2e+09} & dSoft \\
% \fbox{9.2e+11}  & \fbox{-3.7e+09} &\textcolor{gray}{-4.0e+09} &  \textcolor{gray}{3.4e+08} & \textcolor{gray}{-1.7e+09} & dHard \\
%    \textcolor{gray}{5.4+10} & \textcolor{gray}{-1.3e+09} &  \fbox{2.1e+10} & \fbox{\textcolor{gray}{-3.9e+09}} &  \fbox{\textcolor{gray}{-1.7e+10}} & cSoft \\
%    \textcolor{gray}{4.9e+11} & \textcolor{gray}{-1.6e+09} & \fbox{-2.4e+10} &  \fbox{\textcolor{gray}{1.2e+09}} & \fbox{-1.9e+10} & cHard \\
%    \textcolor{gray}{2.6e+11} & \textcolor{gray}{-1.1e+09} & \fbox{-2.4e+10}  &  \fbox{7.0e+09} & \fbox{-3.5e+10} & RM \\
% \hline
% \end{tabular}
% \label{table:sensing}
% \end{table}

%\textcolor{blue}{Paragraph discussing sensing matrix numbers.}



\subsection{WFS Output Matrix}
The WFS output matrix determines how to convert the radiation pressure
eigenbasis control signals into individual mirror control signals. It
is the basis transformation matrix, $S$, as defined in
Eq.~\ref{eq:S}. The matrix is arbitrarily normalized so the largest
element is $1$, and it is repeated with appropriate sign changes to
form differential and common soft and hard modes of the two arms. The
output matrix is shown in Table~\ref{table:output}, where the $r$ is
0.91 for Livingston and 0.87 for Hanford. 

\begin{table}
\centering
\caption[WFS output matrix]{WFS output matrix (pitch). For the Livingston cavity geometry
  $r=0.91$ and for Hanford $r=0.87$.}
\begin{tabular}{l l l l l l}
\hline 
dSoft & dHard & cSoft & cHard & RM & \\
\hline 
1 & r & 1 & r & 0 & ETMX\\
-1 & -r & 1 & r & 0 & ETMY \\
r & -1 & r & -1 & 0 & ITMX\\
-r & 1 & r & -1 & 0 & ITMY\\
 0 & 0 & 0 & 0 & 1 & RM\\
\hline
\end{tabular}
\label{table:output}
\end{table}



\subsection{Diagonalizing the WFS Drive Matrix}
\label{sec:mirrorgains}
The mirror gain matrix (introduced in Fig.~\ref{fig:ASCcontrolservo})
is a diagonal matrix that modifies the amplitude of the control signal
to each optic by some factor close to unity. It corrects for
experimentally measured deviations from the theoretical output
matrix. The purpose of the mirror gain matrix, $\mathbf{M}$, is to
ensure that when a particular WFS is driven, only its DOF is
excited. We call the matrix of observed DOF motions per WFS excitation
the drive matrix, $\mathbf{D}$.

The mirror gain matrix is what we tune to make the drive matrix
diagonal. The drive matrix can be represented as:
\begin{equation}
\mathbf{D} = \mathbf{U M C}.
\end{equation}
A WFS control signal is multiplied first by the output (control)
matrix, $\mathbf{C}$, then by the mirror gains, and then put into
hardware with an unknown transfer function, $\mathbf{U}$, to create a
physical torque on the mirrors. To make $\mathbf{D}$ diagonal, we
start with $\mathbf{M}=\mathbf{1}$, experimentally construct the drive
matrix, and calculate the new mirror gain matrix,
$\mathbf{M_{\mathrm{new}}}$, which must have the property:
\begin{equation}
\mathbf{M_{\mathrm{new}}} = \mathbf{U^{-1}} \mathbf{C^{-1}} = \mathbf{M C D^{-1} C^{-1}}.
\end{equation}
Note that the set of unknown transfer functions $\mathbf{U}$ is
eliminated.

I measured $\mathbf{D}$ by recording the demodulated optical lever
error signals during the sensing matrix measurement (refer to
Sec.~\ref{sec:inputmatrix}). Because the optical levers provide a
record of the motion of the test masses, combining their calibrated
responses via the output matrix (Table~\ref{table:output}) allows one
to convert the individual mirror motions during each DOF excitation to
motions in the radiation pressure DOF basis. \footnote{The optical
  lever calibration procedure and measurement results are found in
  Appendix~\ref{sec:oplevcal}.}

The mirror gain matrix is shown in Table~\ref{table:mirrorgains}. The
30\% difference with the model is a result of uncertainty in the
cavity geometry. The resulting drive matrix, corresponding to the same
measurement time as the sensing matrix of Table~\ref{table:sensing},
is presented in Table~\ref{table:excitations_calibrated}. Columns are
excitations and rows are the pitch motions at 9.7~Hz in units of
\microrad. The absolute amplitude of DOF motions for each excitation
is not of significance; only the amplitudes in each column should be
compared. Ideally, this drive matrix should be diagonal and it is up
to at most a factor of two.

\begin{table}
\centering
\caption[Mirror gains for diagonalization of drive matrix]{Mirror
  gains for diagonalization of drive matrix.} 
\begin{tabular}{l l l l l l}
\hline
ETMX & ETMY & ITMX & ITMY & RM & \\
\hline
1.33 & 1.38 & 0.96 & 0.87 & 1.0 \\
\hline
\end{tabular}
\label{table:mirrorgains}
\end{table}

\begin{table}
\centering
\caption[Actual eigenbasis motion during sensing matrix
excitations]{Actual eigenbasis motion during sensing matrix
  excitations as witnessed by the optical levers. Columns are
  excitations and rows are the pitch motions at 9.7~Hz in units of
  \microrad. The mirror gains are selected to make this matrix as
  diagonal as possible.} 
\begin{tabular}{r@{.}l r@{.}l r@{.}l r@{.}l r@{.}l l}
\hline
\multicolumn{2}{l}{dSoft} & \multicolumn{2}{l}{dHard}  & \multicolumn{2}{l}{cSoft} & \multicolumn{2}{l}{cHard} & \multicolumn{2}{l}{RM} & \\
\hline
   \textbf{5}&\textbf{1e-06} & -5&2e-08  & 6&1e-07 & -3&8e-08 &  -1&8e-07 & dSoft\\
  -3&4e-07 &  \textbf{5}&\textbf{0e-06}  &  7&3e-07 & -1&0e-06 &  2&4e-07 & dHard\\
  -4&1e-07 & -3&3e-08 &  \textbf{5}&\textbf{9e-06} &  6&8e-07 &  2&5e-07 & cSoft\\
  -6&4e-07 & -5&9e-07 &  1&1e-06 &  \textbf{5}&\textbf{7e-06} &  4&7e-07 & cHard\\
  -1&6e-07 & -1&8e-06 & -5&5e-07 &   2&6e-06 &  \textbf{5}&\textbf{6e-06} & RM\\
\hline
\end{tabular}
\label{table:excitations_calibrated}
\end{table}


% This now allows us to complete the
% first part of obtaining an optical gain matrix. Each row of the
% sensing matrix should be divided by the amplitude of that dof's
% excitation, the diagonal elements of Table
% \ref{table:excitations_calibrated}. The result of doing so gives the
% angular optical gain of the interferometer in terms of WFS digital
% counts per radian.




\section{Sensing Matrix Stability}
As the interferometer conditions change, so does the sensing
matrix. The inverse of the sensing matrix, the input matrix, is
hardcoded in the digital control servo and not actively
updated. Therefore, it can be expected that the ASC performance may
not be stable. 

By design, the input matrix is not exactly the inverse of the sensing
matrix, meaning the system is not completely diagonal. For example,
the input matrix times the sensing matrix is, by design:
\begin{equation}
\left\llbracket \begin{array}{ccccc}
0.79 & 0 & 0.30 & 0 & 0 \\
0 & 1.0 & 0 & 0 & 0 \\
0 & 0 & 1.0 & 0 & 0 \\
0 & 0 & 0 & 1.0 & 0 \\
0 & 0 & 0 & 0 & 1.0 
\end{array} \right\rrbracket
\end{equation}

Over time, the sensing matrix changes enough that the system is even less 
diagonal. Only 10 minutes after having measured and created the input
matrix, the product of the input matrix with a newly measured new
sensing matrix is: 
\begin{equation}
\left\llbracket \begin{array}{ccccc}
0.79 & 0 & 0.32 & 0 & 0 \\
0 & 1.0 & 0 & 0 & 0 \\
-0.02 & 0 & 1.0 & 0 & 0 \\
0 & -0.02 & 0 & 1.0 & 0.02 \\
0 & 0.02 & 0 & 0.05 & 1.03 
\end{array} \right\rrbracket
\end{equation}

After one week, it is:
\begin{equation}
\left\llbracket \begin{array}{ccccc}
0.67 & 0 & 0.27 & 0 & 0 \\
0 & 0.91 & 0 & 0.03 & -0.03 \\
0.09 & 0 & 0.84 & 0 & 0 \\
0 & 0.15 & 0 & 0.83 & 0.27 \\
0 & 0.06 & 0 & 0.07 & 1.04 
\end{array} \right\rrbracket
\end{equation}

And after three weeks, it is:
\begin{equation}
\left\llbracket \begin{array}{ccccc}
1.4 & 0 & 0.55 & 0 & 0 \\
0 & 1.8 & 0 & -0.09 & 0.15 \\
-0.08 & 0 & 2.0 & 0 & 0 \\
0 & -0.22 & 0 & 1.4 & 0.64 \\
0 & 0.30 & 0 & -0.72 & 2.9 
\end{array} \right\rrbracket\
\end{equation}

Despite these significant changes, the interferometer remained stable
and the sensitivity remained constant. This shows that the ASC is a
very robust sensing and control system.


\section{Input Beam Motion}
The beam centering and QPD servos operate up to only about 50~mHz,
meaning the beam-centering degree of freedom is uncontrolled at higher
frequencies. Because beam spot motion on the test masses couples to
DARM, anything that causes the beam's position on the test mass to
change on time scales faster than half a minute becomes itself a
direct noise source for DARM. The HAM seismic isolation tables from
which the input optics are suspended have resonant ``stack'' modes
from about 0.8~Hz to 3~Hz. The excess table motion at these
frequencies is transmitted to the MMTs. Jitter on the pointing of the
input beam is thus a primary contender for beam spot motion on the
test masses.

% can't move the input beam faster than the interferometer can follow it!

The wavefront sensor servos are the mechanism by which input beam
motion is impressed on the test masses; they are responsible, among
other things, for making the interferometer follow the input beam up
to several Hz. The WFS detect differences between the angle of the
cavity (as determined by the angles of the mirrors) and the angle of
the beam impinging the cavity. If either the input beam or the cavity
angle changes, the WFS will move the mirrors to correct for the angle
mismatch. Thus, even if the mirrors are perfectly quiet, a
non-stationary input beam will result in mirror motion and mirror
motion in turn creates beam spot motion as seen from Eq.~\ref{eq:x}.

I measured the impression of the input beam motion on the mirrors by
increasing the gain of the common-degree-of-freedom WFS servos (cHard,
cSoft, RM) for about 10~minutes. Comparing the amount of angular motion
of the mirrors as witnessed by the optical levers from this time of
high common WFS gain to a time with nominal WFS gain and similar
seismic motion, we can see the effect
directly. Figure~\ref{fig:inputbeam_impression} shows comparison
spectra, demonstrating how there is higher test mass motion around
1~Hz when the common WFS gains are higher. The rms mirror motion also
increases by about 20\%.

\begin{figure}
\begin{centering}
\subfigure{\includegraphics[width=0.5\textwidth]{figures/ETMXrms_highnom.pdf}}\subfigure{\includegraphics[width=0.5\textwidth]{figures/ITMXrms_highnom.pdf}}
\subfigure{\includegraphics[width=0.5\textwidth]{figures/ETMYrms_highnom.pdf}}\subfigure{\includegraphics[width=0.5\textwidth]{figures/ITMYrms_highnom.pdf}}
\subfigure{\includegraphics[width=0.5\textwidth]{figures/RMrms_highnom.pdf}}
\caption[Impression of input beam motion on the core mirrors]{Input
  beam motion impression on the core mirrors (pitch). Residual mirror
  motion as witnessed by the optical levers when the common WFS gains
  (cSoft, cHard, RM) are increased to $2.5\times$ nominal is compared
  to residual mirror motion when the WFS gains are nominal. Dashed
  lines are the root-mean-square of the amplitude spectral density
  integrated from the right. Both spectra come from a time of similar
  seismic activity (typical weekday afternoon noise), shown in
  Figure \ref{fig:seismic_highgain}.}
\label{fig:inputbeam_impression}
\end{centering}
\end{figure}

\begin{figure}
\begin{centering}
\subfigure{\includegraphics[width=0.5\textwidth]{figures/testWFS1_highnom.pdf}}\subfigure{\includegraphics[width=0.5\textwidth]{figures/WFS2A_highnom.pdf}}
\subfigure{\includegraphics[width=0.5\textwidth]{figures/WFS2B_highnom.pdf}}\subfigure{\includegraphics[width=0.5\textwidth]{figures/WFS3_highnom.pdf}}
\subfigure{\includegraphics[width=0.5\textwidth]{figures/WFS4_highnom.pdf}}
\caption[Comparison of WFS error signals (the residual motion) during
a time of normal operation and a time when the common WFS gains were
$2.5\times$ higher than nominal]{Comparison of WFS error signals (the
  residual motion) during a time of normal operation and a time when
  the common WFS gains were $2.5\times$ higher than nominal. This
  excludes gain peaking as a cause of the extra mirror motion
  witnessed during the time of high gain
  (Figure~\ref{fig:inputbeam_impression}). Dashed lines are the
  root-mean-square of the amplitude spectral density integrated from
  the right. Figure~\ref{fig:seismic_highgain} shows the ground motion
  spectra at the time of this measurement.}
\label{fig:WFS_inputbeam}
\end{centering}
\end{figure}

It is possible for the extra mirror motion to result from gain peaking
of the WFS servos. However, a plot of the WFS error signals during the
time of nominal gain and high gain shows that there is not, in fact,
any evidence of gain peaking. The common WFS spectra in
Figure \ref{fig:WFS_inputbeam} do not show any extra noise when their
gains are higher. It is worth noting that the higher gain is evident
in the common WFS spectra by the extra suppression seen below
1~Hz. Also, the differential WFS spectra are unchanged, as
expected. It can therefore be concluded with reasonable certainty that
the increase in test mass motion between 1 and 2~Hz during this test
is indeed due to the WFS impressing input beam motion on the mirrors.


% A more quantitative study of the effects of input beam motion is to
% measure transfer coefficients between the input beam motion and the
% mirror angular (or beam spot) motion. During a full interferometer
% lock I put lines in MMT1, MMT2, and MMT3 at 1.05~Hz, 1.25~Hz, and
% 0.85~Hz, respectively, selecting excitation amplitudes large enough to
% appear in the common WFS spectra. I compared the amplitudes of the
% lines in the MMT spectra with those in the WFS spectra. The counts to
% counts transfer coefficients are shown in Table
% \ref{table:inputbeam_TFcoeffs}. Because sensing is flat, I use the shape
% of the WFS loop to extrapolate this transfer coefficient at one
% frequency to other frequencies. The result is a transfer function
% useful for making a noisebudget of input beam motion to test mass
% motion. Selecting a science mode time of typical day-time seismic
% noise, I made a noisebudget as found in Figure \ref{fig:inputbeam_NB}.


% \begin{table}
% \centering
% \caption[MMT to WFS transfer coefficients]{MMT to WFS transfer
%   coefficients. \textcolor{blue}{Make this!!!}}
% \begin{tabular}{l l l l l l}
% \hline
%           & WFS1Q & WFS2I & WFS2Q & WFS3I & WFS4I \\
% \hline
% MMT1 & \\
% MMT2 & \\
% MMT3 & \\
% \hline
% \end{tabular}
% \label{table:inputbeam_TFcoeffs}
% \end{table}



% \begin{figure}
% \begin{centering}
% %\includegraphics[width=1.0\columnwidth]{figures/}
%   \caption[Mirror motion due to input beam impression]{Noise budget of
%     mirror motion due to input beam impression. \textcolor{blue}{Make
%       this!!}}
% \label{fig:inputbeam_NB}
% \end{centering}
% \end{figure}



\section{The Marginally-stable Power Recycling Cavity}
The power recycling cavity (PRC) is the linear cavity formed by the RM
and ITMs. Because the radius of curvature of both the RM and the ITMs
points in the same direction and the waist is well outside the
Rayleigh range of the mirrors, the cavity is geometrically
unstable. For example, in its cold state at LLO the $g$-factor of the
cavity is 1.00005 and at LHO it's 1.00003. The beam in the PRC is not
spatially contained and the cavity is degenerate with respect to
higher order modes. The heating of the ITMs from the kilowatts of
power in the arm cavities together with the ITM thermal compensation
system (TCS) serve the role of making the PRC geometrically stable for
interferometer operation. The heating and cooling of the ITMs is a
very complicated process and therefore not very precise, so the value
of the hot PRC's $g$-factor is usually not constant.

%\textcolor{blue}{differentiate magnitude of SPOB vs fluctuations in
%SPOB.} 
The changing $g$-factor has potentially severe consequences
for the ASC. Because of its geometry, the power build-up in the PRC is
very sensitive to both the mirror angles and the $g$-factor. Power
fluctuation is detrimental because the signal to noise ratios of the
sensors that probe the PRC light degrade due to the presence of
increased junk light that contributes shot noise but not
signal. WFS1Q, WFS2I, and WFS2Q are the most sensitive to the PRC
because their signals are derived from the 25~MHz sidebands. Their
sensitivity to mirror motion is therefore subject to change. Because
achieving a flat power build-up in the PRC is a difficult task (too
much motion in the PRC is quite often a cause of lock loss when making
measurements), we must update the real-time control system to reflect
their changing sensitivities. Otherwise, the mirror angles will not be
controlled accurately.

An estimate of the expected power fluctuations based on the $g$-factor
and RM motion is a straightforward exercise when using
Eq. \ref{eq:cavitydisptilt_mirrorangle} and Eq. \ref{eq:pwr_disptilt}
as derived in Appendix~\ref{appendix:ASCextra}. If we estimate the
$g$-factors of the RM and ITM as $g_{RM} = 1+\delta$ and $g_{ITM} = 1
- \delta$ (where $\delta =6 \times 10^{-4}$ for LLO the cold state)
and approximate the distance of each mirror to the cavity waist as $z$
because the two mirrors are very close to each other compared to the
waist location, then Eq. \ref{eq:cavitydisptilt_mirrorangle} reduces
to:
\begin{equation}
\left\llbracket \begin{array}{c}
a_{PRC} \\
\alpha_{PRC} \end{array} \right\rrbracket = 
\left\llbracket \begin{array}{cc}
z(2+\delta)/\delta & z(2-\delta)/\delta \\
-1/\delta & -1/\delta \end{array} \right\rrbracket
\left\llbracket \begin{array}{c}
\theta_{RM}\\
\theta_{ITM} \end{array} \right\rrbracket.
\end{equation}


\begin{figure}
\begin{centering}
\includegraphics[width=1.0\columnwidth]{figures/prc_power.pdf}
\caption[Theoretical dependence of power recycling cavity power on
$g$-factor and mirror angle]{Dependence of power build-up in the power
  recycling cavity on the PRC's $g$-factor and the RM tilt. TCS is
  necessary for stabilizing the PRC's geometry and therefore its
  sensitivity to mirror motion. For simplicity, the ITM is assumed
  stationary in these plots.}
\label{fig:prc_power}
\end{centering}
\end{figure}

Figure \ref{fig:prc_power} plots the power in the PRC as a function of
$\theta_{RM}$ for several values of $\delta$, demonstrating
the sensitivity of the PRC to the ITM heating.  For example, the
typical RM angular displacement of $10^{-7}$ rad results in a 66\%
power loss when the PRC $g$-factor is very near instability with a
value of $1-0.0001$. Only as the $g$-factor moves further from $1$
does the angular motion of the RM have less and less of an effect on
the power build-up.



\subsection{Power Scaling}
\label{sec:powerscaling}
The signal at the wavefront sensors is proportional to the amplitude
of the sidebands, or the square root of the sideband power. Thus, as
the PRC $g$-factor and therefore the power in the recycling cavity
changes, so do the WFS1 and WFS2 optical gains. In order to
compensate for this $g$-factor dependence, we multiply the WFS\{1Q,
2I, 2Q\} error signals in real-time by
\begin{equation}
\frac{1}{P_{in}} \left[\frac{\mathrm{NSPOB}}{350}\right]^{-1/2}
\end{equation}
and WFS3I and WFS4I by $1/P_{in}$. NSPOB is the normalized sideband
power in the PRC as measured by the $2f$-demodulated POB signal, and
the 350 is the reference NSPOB, treated as nominal. Thus, during
interferometer operation, all WFS signals are normalized to input
power and are not dependent on the PRC power. This correction to the
WFS signals is called power scaling.

To verify that the WFS optical gains do indeed scale with the sideband
power as expected, I tracked the WFS optical gain as $g$ changes. I
excited three of the test masses (ETMX, ITMX, RM) at three different
frequencies (9.7~Hz, 10.7~Hz, and 11.7~Hz, respectively) during a full
interferometer lock and changed the TCS settings so that over the
course of 15 minutes the $g$-factor steadily changed. Demodulating
each of the WFS signals at each of the three excitation frequencies as
a function of time shows how the strength of the signal at the WFS due
to the motion of these three mirrors changes. To compensate for the
difference in pendulum responses to the excitations, I multiplied the
demodulated signals for a particular excitation $f$ by $(f/9.7)^2$. I
also normalized the response by the phase of the mirror's motion as
witnessed by the optical levers.

\begin{figure}
\begin{centering}
\subfigure{\includegraphics{figures/trackSPOB.pdf}}\subfigure{\includegraphics{figures/trackWFS1Q.pdf}}
\subfigure{\includegraphics{figures/trackWFS2I.pdf}}\subfigure{\includegraphics{figures/trackWFS2Q.pdf}}
\subfigure{\includegraphics{figures/trackWFS3I.pdf}}\subfigure{\includegraphics{figures/trackWFS4I.pdf}}
\caption[Measured dependence of the WFS error signals on the power
recycling cavity geometry]{WFS optical response to test mass motion as
  a function of power recycling cavity geometry. WFS1Q, WFS2I, WFS2Q
  are more sensitive to test mass motion as the power in the recycling
  cavity increases. Therefore, to achieve a dependable feedback
  system, we scale the error signals in real-time, forcing their
  responses to be flat with power. This range of PRC power is low for
  normal operations.}
%\textcolor{blue}{plot as ratio of first value and include fitted lines for satisfaction.}}
\label{fig:WFStrack}
\end{centering}
\end{figure}

The results are shown in Figure \ref{fig:WFStrack}, and include a plot
of how NSPOB changed over time. As expected, WFS1Q, WFS2I, and WFS2Q
show dependence on the PRC power, and therefore the $g$-factor. The
WFS3 and WFS4 sensing elements are flat. Fitting lines to each of the
tracked elements, we find a good fit with the expected power of $1/2$
dependence. 



\subsection{Sideband Imbalance}
Another important effect of the PRC on the ASC signals is the
balancing of the upper and lower 24.4~MHz sideband amplitudes. SPOB is
the product of the amplitude of the upper and lower sidebands, but the
total sideband power is the sum of the power in the lower sideband and
the power in the upper sideband. Therefore, if the upper and lower
sidebands are not the same, SPOB does not accurately represent the
power in the cavity. This creates inaccuracies in the WFS power
scaling.

We set up a temporary optical spectrum analyzer at a pickoff of the
anti-symmetric port beam in order to measure amplitudes of the
sidebands. Without any TCS, we saw that with less than 6~W input
power, the lower sideband is smaller than the upper sideband, at 6~W
the amplitudes are equal, and above 6~W, the upper sideband is smaller
than the lower. Thus, if TCS is not tuned perfectly at all times, we
can expect unequal sidebands.
% \textcolor{blue}{April 1, 2009 notebook. I don't have OSA data, only
%   sketches in notebook. Worth saying anything?}




% \section{DC readout related measurements}
% \begin{itemize}
% \item RF created from DC offset beam moving on WFS1
% \item RF vs DC vs power comparison of (AS) beam spot motion on WFS1
% \end{itemize}


% \section{ASC noisebudget}
% \begin{itemize}
% \item seismic - breakdown of soure of motion
% \item L2A
% \item input beam 
% \item electronics noise
% \item shot noise
% \end{itemize}




\section{WFS Servo Open Loop Transfer Functions}
\label{sec:WFSolgs}

The open loop transfer function, or open loop gain (OLG), is an
informative measure of the characteristics of a control servo. It is
the product of each of the elements of the loop, and is often
summarized as being the product of the plant, $H$, with the control
filters, $G$:
\begin{equation}
\mathrm{OLG} = HG.
\end{equation}
In our case, the plant is the radiation-pressure-modified pendulum
transfer function (Eq.~\ref{eq:modalTF}) and the control filters
(found in Appendix~\ref{sec:WFSfilters}) are the digital filters
between the input and output matrices. The amplitude of the OLG tells
the suppression the control loop provides and the phase tells about
the stability of the system.

\begin{figure}
\begin{centering}
\subfigure{\includegraphics[width=1.0\columnwidth]{figures/allolgs_6W_mags.pdf}}
\subfigure{\includegraphics[width=1.0\columnwidth]{figures/allolgs_6W_phases.pdf}}
\caption{Open loop gains (pitch) of the 5 WFS loops as measured with 6 W
  input power.}
\label{fig:olgs6W}
\end{centering}
\end{figure}

We measure the open loop transfer function while the loop is
closed. This is done by injecting a swept-sine excitation just before
the digital filters and taking the ratio of the signals just before
and just after the excitation. Figure~\ref{fig:olgs6W} shows the open
loop transfer functions of each of the wavefront sensor loops as
measured during a 6~W lock. As anticipated from the large differential
soft signal seen by WFS1 in the sensing matrix measurement
(Table~\ref{table:sensing}), that is the mode for which we can and do
provide the strongest suppression. However, it is also conditionally
stable (due to a pole at zero that is engaged after the servo is
turned on), as seen by the phase dropping below $-180^{\circ}$ at two
different frequencies. The unity gain frequency (UGF) must be held
between roughly 2~Hz and 10~Hz. As shown here, the dSoft UGF is at
5~Hz. For the other degrees of freedom, the UGFs are all between 0.8
and 1~Hz. Each loop has a phase margin of $30^{\circ}$ to
$40^{\circ}$.

Figure~\ref{fig:DSolgs} shows measurements of how the dSoft (WFS1) OLG
changes with power. The 6~W measurement data is the same as that
presented in Figure~\ref{fig:olgs6W}. From 1~W to 14~W input power the
suppression provided by the control loop remains constant, but the
stability of the loop changes. The phase margin at the UGF decreases
by about $10^{\circ}$ and a third $-180^{\circ}$ phase crossing
appears at about 500~mHz. The digital filters are fixed with power, so
the observed changes are due to a changing plant. This is precisely
the result of the radiation pressure angular spring. I present the
direct measurement of the plant's modified transfer function in
Section~\ref{sec:siggsidles_measured}.

\begin{figure}
\begin{centering}
\subfigure{\includegraphics[width=1.0\columnwidth]{figures/wfs1_olgs_mags.pdf}}
\subfigure{\includegraphics[width=1.0\columnwidth]{figures/wfs1_olgs_phases.pdf}}
\caption{Open loop gains (pitch) of the differential soft (WFS1) loop as measured at four
  different powers.}
\label{fig:DSolgs}
\end{centering}
\end{figure}

% \begin{figure}
% \begin{centering}
% \subfigure{\includegraphics[width=1.0\columnwidth]{figures/wfs2b_olgs_mags.pdf}}
% \subfigure{\includegraphics[width=1.0\columnwidth]{figures/wfs2b_olgs_phases.pdf}}
% \caption{Open loop gains (pitch) of the differential hard (WFS2B) loop as measured at four
%   different powers.}
% \label{fig:DHolgs}
% \end{centering}
% \end{figure}




\section{Residual Angular Motion}
The residual beam spot motion on the test masses is shown in
Figure~\ref{fig:bsm}. We see the rms beam spot motion on the ETMs is
1~mm and on the ITMs it is 0.8~mm which meets the requirements of
Section~\ref{sec:tolerance}. The beam spot motion calibration method
and results are presented in Appendix~\ref{sec:bsmcal}.

\begin{figure}
\begin{centering}
\subfigure{\includegraphics[width=1.0\textwidth]{figures/BSMpit.pdf}}
\subfigure{\includegraphics[width=1.0\textwidth]{figures/BSMyaw.pdf}}
\caption[Beam spot motion on the ITMs and ETMs during a 16~W
lock]{Beam spot motion on the ITMs and ETMs during a 16~W lock at
  night. Ground motion at the time of this measurement is shown in
  Figure~\ref{fig:seismic_bsm}}%(May21, 2010, 06:02:30 UTC).}
\label{fig:bsm}
\end{centering}
\end{figure}


% \begin{figure}
% \begin{centering}
% \subfigure{\includegraphics{figures/DS_darkres.pdf}}\subfigure{\includegraphics{figures/CS_darkres.pdf}}
% \subfigure{\includegraphics{figures/DH_darkres.pdf}}\subfigure{\includegraphics{figures/CH_darkres.pdf}}
% \subfigure{\includegraphics{figures/RM_darkres.pdf}}
% \caption[Residual motion of the radiation pressure eigenbasis degrees
% of freedom]{Residual motion of the radiation pressure eigenbasis
%   degrees of freedom using calibrated WFS error signals during a 10~W
%   lock compared to dark noise and the background motion as calculated
%   using the open loop transfer function measurements. The sensor dark
%   noises (see Figure~\ref{fig:WFSdarknoise}) are put in the RP
%   eigenbasis by propagation through the WFS power scaling and WFS
%   input matrix. Ground motion at the time of the residual motion
%   spectra is found in Figure~\ref{fig:seismic_locked}.}
% \label{fig:WFSresidual}
% \end{centering}
% \end{figure}

We quantify the effect of the ASC on the radiation pressure eigenbasis
degrees of freedom by comparing spectra of the residual eigenbasis
motion during lock to the equivalent eigenbasis motion as witnessed by
the optical levers out of lock. The comparisons cannot be perfect
because the spectra are necessarily taken at different times and
therefore with different seismic noise conditions. However, the effect
of the ASC is stark at low frequencies where gain is high, as is seen
in Figure~\ref{fig:ASConoff}. At 0.01~Hz angular motion of all degrees
of freedom is suppressed by at least one order of magnitude. The
typical residual rms angular motion is $10^{-7}$
rad/$\sqrt{\mathrm{Hz}}$. The effect of the high gain for the
differential soft degree of freedom, in particular, is seen here,
where order of magnitude suppression is seen already at 1~Hz.

\begin{figure}
\begin{centering}
\subfigure{\includegraphics[width=0.5\textwidth]{figures/onoff_DS.pdf}}\subfigure{\includegraphics[width=0.5\textwidth]{figures/onoff_CS.pdf}}
\subfigure{\includegraphics[width=0.5\textwidth]{figures/onoff_DH.pdf}}\subfigure{\includegraphics[width=0.5\textwidth]{figures/onoff_CH.pdf}}
\subfigure{\includegraphics[width=0.5\textwidth]{figures/onoff_RM.pdf}}
\caption[Angular motion suppression due to the ASC]{Demonstration of
  angular motion suppression down to 4~mHz due to the ASC. 
%Same as Figure~\ref{fig:WFSresidual}, but the background (here called
%``no  ASC'') 
The background motion (``no ASC'') is the RP eigenbasis reconstruction of optical lever signals
  when the interferometer is not locked. Data are taken 45 minutes
  apart, and the ground motions are shown in
  Figures~\ref{fig:seismic_nolock} and \ref{fig:seismic_locked}. The
  differences in ground motion explains the discrepancies between 1~Hz
  and 10~Hz.}
\label{fig:ASConoff}
\end{centering}
\end{figure}

The magnitudes of the beam spot motion and the residual mirror motion
are consistent and reasonable. For example, for $10^{-7}$ rad of soft
or hard mode motion in one arm, the maximum cavity tilt and
displacement are 0.12~\micro rad and 1.02~mm, respectively, as found
in Table~\ref{table:cav_geometric}.

Figure~\ref{fig:mirror_onoff} shows the same data as
Figure~\ref{fig:ASConoff} except in the mirror basis instead of the
radiation pressure eigenmode basis. The ASC on/off comparison of
mirror motion is interesting because it shows that the mirrors
actually move more with respect to the ground when they are controlled
by the ASC than when they are not controlled by the ASC. This is to be
expected because the ground motion is different at each mirror and the
job of the ASC is to control the motions of the mirrors with respect
to each other, not with respect to the ground.

\begin{figure}
\begin{centering}
\subfigure{\includegraphics{figures/onoff_ITMX.pdf}}\subfigure{\includegraphics{figures/onoff_ITMY.pdf}}
\subfigure{\includegraphics{figures/onoff_ETMX.pdf}}\subfigure{\includegraphics{figures/onoff_ETMY.pdf}}
\subfigure{\includegraphics{figures/onoff_RMmir.pdf}}
\caption[Individual mirror motion with and without ASC]{Individual
  mirror motion with and without the ASC. The mirrors move more with
  respect to their local grounds when the interferometer is controlled
  than when they're on their own. Data are taken 45 minutes apart, and the ground
  motions are shown in Figures~\ref{fig:seismic_nolock} and
  \ref{fig:seismic_locked}.}
\label{fig:mirror_onoff}
\end{centering}
\end{figure}



\section{ASC to DARM Noise Budget}
\label{sec:asc2darm}
One of the most important figures of merit for the ASC is how much
noise it contributes to DARM. As introduced in
Section~\ref{sec:tolerance}, the combination of beam spot motion on
the mirror together with the angular motion of the mirror creates an
angle to length (A2L) coupling. As long as the length displacement due
to this coupling is well below the desired displacement sensitivity,
the ASC has done a good enough job. However, being of the most
complex of interferometer systems, the ASC does in fact limit the
strain sensitivity.

To measure the coupling, a broadband noise injection rather than a
typical swept-sine injection is necessary because of the non-linearity of the
A2L process (in particular, it is the convolution of the beam spot and
mirror angle spectra). In addition, for the measurement itself to be
linear, the amplitude of the noise injection must be only large enough
for an effect to be observed; it must not overwhelmingly
dominate. Therefore, for both DARM and the appropriate ASC channel, I
subtracted quadratically a quiet, no-noise spectrum, $q$, taken
immediately prior to the measurement from the spectrum with noise,
$n$, so as to not assume that the entirety of the observed effect is
due to the excitation:
\begin{equation}
N = \sqrt{n^2 - q^2}.
\end{equation}
The transfer function from the ASC to DARM is then:
\begin{equation}
\mathrm{ASC} \rightarrow \mathrm{DARM} = \frac{N_{\mathrm{DARM}}}{N_{\mathrm{ASC}}}
\end{equation}
which can be multiplied by an ASC signal at any time to determine a
noise budget.

\begin{figure}
\begin{centering}
\includegraphics[width=1.0\columnwidth]{figures/ASC2DARM_TFs.pdf}
\caption[WFS to DARM transfer functions]{ASC to DARM transfer
  function for four of the five wavefront sensor loops. The RM to DARM
  transfer function could not be measured because the contribution is
  too small. The fitted curves can be multiplied by the WFS error
  signals at any time to calculate the ASC noise contribution to
  DARM.}
\label{fig:asc2darmTF}
\end{centering}
\end{figure}

There are two places in the ASC loop from where I made transfer
function measurements by putting in a 40 to 110~Hz broadband
excitation: the suspension angular control
point,\footnote{i.e. \texttt{L1:SUS-ETMX\_ASCPIT\_EXC}} and the WFS
error point.\footnote{i.e. \texttt{L1:ASC-WFS1\_PIT\_EXC}} Each
provides ultimately the same information, but allows the ASC noises to
be broken down in different bases. The view using the WFS (radiation
pressure eigenmode) basis is useful for commissioning purposes,
whereas the optic basis is better suited for including the ASC noise
in full interferometer noise budget plots. The primary reasons are
that the suspension control channel is recorded to disk and it also
serves as the transfer function for the optical lever noise to DARM.
Figure~\ref{fig:asc2darmTF} shows the set of WFS to DARM measured
transfer functions, as an example. The units are DARM meters per WFS
error point digital counts, so that the digital WFS error signal may
be multiplied by the transfer function at any time.

\begin{figure}
\begin{centering}
\subfigure[]{\includegraphics{figures/opticspittoDARM.pdf}}
\subfigure[]{\includegraphics{figures/opticsyawtoDARM.pdf}}
\caption[Optic to DARM noise budget]{Optic to DARM noise budget during a
  14~W lock. A) Pitch. B) Yaw.}% (July 17, 2009).
\label{fig:optic2DARM}
\end{centering}
\end{figure}

\begin{figure}
\begin{centering}
\subfigure[]{\includegraphics{figures/WFSpittoDARM.pdf}}
\subfigure[]{\includegraphics{figures/WFSyawtoDARM.pdf}}
\caption[WFS to DARM noise budget]{Wavefront sensor to
  DARM noise budget during a 14~W lock. A) Pitch. B) Yaw.} % (July 17, 2009).}
\label{fig:wfs2DARM}
\end{centering}
\end{figure}

The noise budget for both pitch and yaw at a time when the
interferometer was locked with 14~W input power is shown in
Figure~\ref{fig:optic2DARM} for the optics basis and in
Figure~\ref{fig:wfs2DARM} for the WFS basis. Each optic's contribution
to DARM is the same within about a factor of two, except for the RM,
which is not included in these plots. We were not able to measure the
transfer function for RM motion to DARM because so large of an
excitation was required to see an effect that the interferometer would
lose lock. I proved this indication that the RM's contribution to DARM
is indeed insignificant by observing no change in DARM upon turning
off the WFS4 cut-off filters. In the WFS basis, the soft modes
contribute more to DARM than the hard modes.

Because two different angular control signals, the WFS and the optical
levers, independently contribute to the suspension angular control, we
can separate their contributions in the noise budget. Furthermore, we
compute the quadrature sum of the pitch and yaw contributions,
assuming these two degrees of freedom are de-coupled, thus creating an
upper limit for the ASC contribution to the DARM noise
budget. Figure~\ref{fig:asc2darm} shows the final summary of ASC noise
in DARM for a 16~W lock at night.

\begin{figure}
\begin{centering}
\includegraphics[width=1.0\columnwidth]{figures/ASC2DARM.pdf}
\caption[Total WFS and optical lever noise contribution to DARM during
a 16 W lock at night]{Total WFS and optical lever noise contribution
  to DARM during a 16 W lock at night. Pitch and yaw contributions are
  added in quadrature under the assumption they are
  de-coupled. Seismic spectra at the time of this measurement are found
  in Figure~\ref{fig:seismic_NB}.}
\label{fig:asc2darm}
\end{centering}
\end{figure}

The important message is that the angular sensing and control is, in
fact, a limiting noise source for frequencies between 20 and
55~Hz. The ASC becomes less and less of a primary noise source as
frequency increases, and by 100~Hz the ASC noise floor is a factor of
10 below DARM. The specific structure of the noise contributions,
including the apparent notches, is a direct result of the shape of the
control filters. Imperfections in the estimate of DARM below 50~Hz
arise because the transfer function is not perfectly stable in time.
The seismic noise contribution to DARM (not shown) does in fact sit
just below the ASC floor, so the interferometer sensitivity to GWs is
not dramatically hindered by the ASC. An example, however, of how to
reduce the ASC noise floor, is found through evaluating the WFS
control filters.
% The next two sections, however,
% provide realized and theoretcial examples of how to reduce the ASC
% noise floor.



%\subsection{Tuning the Cut-off Filters} 

\begin{figure}
\begin{centering}
\includegraphics[width=1.0\textwidth]{figures/cutoffWFS1_DARMcompare.pdf}
\caption[Effect of the WFS1 lowpass filter cutoff frequency on strain
sensitivity.]{Effect of the WFS1 lowpass filter cutoff frequency on
  strain sensitivity.}
% \textcolor{blue}{gray legend line missing?}}
\label{fig:WFS1cutoff}
\end{centering}
\end{figure}

The cut-off frequency of the lowpass filters for the WFS control are
of particular importance in the DARM noise budget. The lowpass filter
is necessary for suppressing the impression of sensing noise on
suspension control signals. Steepening the cut-off frequency results
in less sensing noise impression, but each pole used to achieve the
steeper drop-off introduces an extra $90^{\circ}$ of phase
loss. Likewise, lowering the cut-off frequency reduces noise
impression, but it pushes the phase loss to lower frequencies. The
effect is a decrease in phase margin of the WFS loops for a particular
UGF, which leads to gain peaking and a greater likelihood of loop
instability. A fine balance must therefore be found between loop
stability and noise impression. Because WFS1 (differential soft) is
the dominant contibutor to the noise budget, we put forth effort to
tune its low pass filter's frequency. Figure~\ref{fig:WFS1cutoff}
demonstrates the effect on DARM of decreasing the WFS1 cut-off filter
frequency from 35~Hz to 30~Hz.


\section{Seismic Feed-forward to the ASC}
We show in Section~\ref{sec:asc2darm} that angular control limits the
strain sensitivity at low frequencies and we see from
Figure~\ref{fig:h_all} that Advanced LIGO aims to improve the low
frequency sensitivity by up to two orders of magnitude. Although much
of the improvement comes from better seismic isolation, it is prudent
to develop methods to reduce the angular control noise for future
interferometers. A promising technique that I investigated using
Enhanced LIGO data is seismic feed-forward to the ASC.

The concept is to use Wiener filtering \cite{Wiener1975Extrapolation}
to predict the ASC signals from seismometer signals and feed-forward
the filtered sesimometer data in real time. The Wiener filter
coefficients are chosen to minimize the mean square difference between
the WFS error signals and the filtered seismometer signals. For this
to work, there must be a correlation between the seismic and ASC
signals. I designed FIR Wiener filters from simultaneous seismometer
and ASC error signal time series and found that with a sufficiently
long Wiener filter (on the order of 1024 taps) and 30 minutes worth of
data, we can accurately predict the ASC error signal from 0.1 to
20~Hz. Figure~\ref{fig:wienerFF} shows the comparison of the nominal
dSoft (WFS1) error signal with the reduced error signal that one can
hope to achieve through feed-forward. Called the residual, this
reduced error signal is the result of subtracting the
seismometer-predicted ASC error signal from the original error signal.
We see that if implemented, feed-forward may reduce the rms angular
mirror motion by a factor of two. Further studies may reveal even
greater reduction can be achieved.

\begin{figure}
\begin{centering}
\includegraphics[width=1.0\textwidth]{figures/wienerDU_FF.pdf}
\caption[Demonstration of potential reduction of WFS error signals
using seismic feed-forward]{Demonstration of potential reduction of
  WFS error signals using seismic feed-forward.}
\label{fig:wienerFF}
\end{centering}
\end{figure}

There are several ways in which feed-forward might be
accomplished. The two primary options are to send the
seismometer-predicted ASC drive to either the mirror coils or to the
hydraulic external pre-isolator (HEPI) seismic isolation tables. The
benefit of both of these methods is that they would physically reduce
the mirror motion and therefore reduce the need for ASC feedback. This
would allow for more flexible loop shape design such as lower cut-off
filters and result in less impression of ASC noise in DARM. The
simplest of these two options is sending the signal to the coils
because the transfer function from the coils to the mirror motion is
well known. A disadvantage is that the total coil currents would not
be reduced, meaning the noise from magnetic domain flipping,
Barkhausen noise, would not be reduced. Feeding-forward to HEPI would
reduce the Barkhausen noise, but would require a carefully measured
transfer function from HEPI to angular mirror motion. Seismic
feed-forward to HEPI to reduce LSC signals was demonstrated during
Enhanced LIGO \cite{DeRosa2011Global}.

An alternative method to reducing the ASC noise in DARM does not
involve the seismometers or feed-forward to the ASC, but feed-forward
to DARM itself. From a carefully measured ASC to DARM transfer
function, the predicted contribution of ASC noise in DARM can be
subtracted from DARM in real time. This type of feed-forward was
implemented in Initial and Enhanced LIGO for the Michelson (MICH) and
power recycling cavity (PRC) loops \cite[Ch. 2]{Ballmer2006LIGO}. 

Any of these methods may be implemented in post-processing of the DARM
data, but there are benefits to doing it in real-time. The primary
advantage is the actual reduction of mirror motion which would make
for an all-around more stable interferometer. Second, eliminating
known noises from the strain spectrum is very useful for
commissioning. It allows us to then see other noises in real-time and
to be more efficient at the endless game of noise hunting.


% \section{Advanced LIGO}
% The Advanced LIGO interferometers will have more massive mirrors, a stable
% recycling cavity, and more circulating power. 


% \begin{figure}
% \begin{centering}
% \includegraphics{figures/dSoftFF.pdf}
% \caption{}
% \label{fig:}
% \end{centering}
% \end{figure}



\section{Experimental Measurement of the Radiation Pressure Angular Spring}
\label{sec:siggsidles_measured}

I directly measured the expected radiation-pressure-modified torque to
angle transfer function of the LIGO arm cavities that I first
introduced in Chapter~\ref{ch:eigenmodes}. Through a clever
measurement method, I witness both the stable and unstable modes
without any post-data-taking manipulation.

\begin{figure}
\begin{centering}
\includegraphics{figures/RPmeasurement_thesis.pdf}
\caption[Demonstration of radiation pressure eigenbasis torque to
angle transfer function measurement]{Demonstration of radiation
  pressure eigenbasis torque to angle transfer function
  measurement. Through a proper choice of measurement locations within
  the ASC servo, the plant's transfer function can be singled out.}
\label{fig:RPTFmeasurement}
\end{centering}
\end{figure}

The digital control system in which the angular control feedback
system is implemented provides a convenient milieu in which to measure
the response of the optomechanical system. By injecting a disturbance
somewhere in the loop and measuring the response at selected points in
the loop, we can produce a measurement of the optomechanical system
that is not sensitive to the details of the control system. Here we
use this system to produce measurements of the opto-mechanical plant at
several different operating powers, demonstrating the modifications due
to radiation pressure, i.e. the soft and hard modes, sometimes also
referred to as the Sidles-Sigg effect.

The various elements of the plant and the control system are depicted
in Figure~\ref{fig:RPTFmeasurement}.  For this measurement, I took
transfer functions from the torque input to the resulting angular
displacement (as measured by the WFS), both in the radiation pressure
eigenbasis. Simultaneously, I injected an excitation into the control
leg of the servo loop, as is done for the sensing matrix mesaurement
described in Section~\ref{sec:inputmatrix}. The resulting measurement
reproduces the transfer function of the opto-mechanical plant,
independent of the control system.

Results are shown in Figures~\ref{fig:hardTF} (hard mode) and
~\ref{fig:softTF} (soft mode); least-squares fits of second-order
transfer functions are made to the data.  In the hard mode plot, we
can clearly see the increase of the resonant frequency with power,
from $\sim0.65$ Hz at 1~W input power to $\sim$0.95~Hz at 10~W input
power.  Simultaneously, in the soft mode plot, we see the resonance
decrease in frequency as the power is increased from 1~W to 6~W.  When
the input power is increased to 10~W and beyond, the resonance
disappears; the plant has become statically unstable.

\begin{figure}
\begin{centering}
\subfigure{\includegraphics{figures/hardmag.pdf}}
\subfigure{\includegraphics{figures/hardphase.pdf}}
\caption{Hard opto-mechanical mode measurement and fit for several
 powers.}
\label{fig:hardTF}
\end{centering}
\end{figure}

\begin{figure}
\begin{centering}
\subfigure{\includegraphics{figures/softmag.pdf}}
\subfigure{\includegraphics{figures/softphase.pdf}}
\caption{Soft opto-mechanical mode measurement and fit for several
 powers.}
\label{fig:softTF}
\end{centering}
\end{figure}

These measurements show a clear confirmation of the Sidles-Sigg theory
and demonstrate a successful power-independent diagonalization of the
sensing and control of the opto-mechanical plant.



\section{Summary}
The ASC performed as it needed to for Enhanced LIGO to operate at
higher powers without introducing an excess of control noise. The ASC
does limit strain sensitivity at the lowest end of the detection band,
but we offer possible solutions to decrease that noise in future
detectors. We directly measured the radiation pressure angular spring,
confirming our theoretcial understanding of the basic physics that
drives the ASC design. Although the ASC in neither base was limiting
us, the experience gained for Advanced LIGO is priceless. Advanced
LIGO will have heavier mirrors with a different geometry such that
radiation pressure torque will not play so large a role
\cite{Barsotti2010Alignment} and its power recycling cavity will be
stable \cite{AdvLigoSysDesign}.




