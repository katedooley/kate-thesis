\chapter{Angular sensing and control characterization and performance in the radiation pressure eigenbasis} 





% The Enhanced LIGO goal of increasing the input power to 30 W from the
% Initial LIGO 7 W makes radiation pressure torques cross into the realm
% of significance. In particular, the soft opto-mechanical mode which
% approached instability for Initial LIGO powers actually becomes
% unstable for Enhanced LIGO powers. The static instability requires
% high DC gain. The sensors in Initial LIGO, however, were not tuned in
% to specifically look for the combined mirror motions that create the
% unstable mode. The only way to provide adequate DC control to prevent
% the mirrors from falling apart would be to increase the gain of all of
% the angular control loops. Since some of the sensors are less good
% than others, this would result in extreme impression of noise on
% DARM. Thus, to maintain a reasonable noise budget while reigning in
% the static instability, we need to make the sensors specifically pick
% out the mirror motions that together create this detrimental
% mode. This is the basis of the ASC work for Enhanced LIGO--switching
% the sensors to the radiation pressure eigenmode basis, and increasing
% the gain of the single loop that is sensitive to only the unstable mode. 

% In this chapter I show how the angular displacements are sensed and
% why control filters implemented in the eigenbasis of radiation
% pressure torques is best. 



% \section{Sensing}
% There are five main subsets of sensor systems for the ASC:
% \begin{itemize}
% \item OSEMs
% \item optical levers (MMT3, RM, BS, ITMX, ITMY, ETMX, ETMY) \vspace{-10pt}
% \item camera image (BS)
% \item quadrant photodiodes (QPDX, QPDY) \vspace{-10pt}
% \item wavefront sensors (WFS1, WFS2, WFS3, WFS4) \vspace{-10pt}
% \end{itemize}
% Together, these sensors need to provide enough information to derive
% adequate control signals for 9 mirrors in both pitch and yaw. Both
% absolute motion (AC) and relative motion (DC) need to be
% suppressed. The OSEMs and optical levers provide local AC control to
% the mirrors. The video camera and the QPDs control the pointing of the
% input beam; the video camera works at DC and the QPDs at both DC and
% AC. The WFS provide the top level fine tuning of DC and AC control to
% the 5 main mirrors.


% \subsection{OSEMs}
% The most basic level of control is local damping of each suspended
% optic provided by the OSEMs. This is always on, even when the
% interferometer loses lock. It works by sending current through the
% OSEM coils to keep a constant amount of light on the OSEM shadow
% sensor.


% \subsection{Optical levers}
% The optical levers are local to each large optic and provide a record
% at all times of pitch and yaw pointing of each mirror with respect to
% the ground. The mirrors are velocity-damped only by the optical
% levers; they are not controlled at DC. The optical lever is a HeNe
% laser beam that reflects off of the mirror and onto a QPD. Both the
% laser and QPD are mounted on heavy piers to reduce the seismic noise
% contribution to the QPD signal. The optical lever provides a feedback
% signal to the mirror's coils to reduce the sensed motion. Each large
% optic has its own independent optical lever loop which is almost
% always on, even when the interferometer is out of lock. The optical
% lever loops provide the second level of controlled stabilization of
% the mirrors, after only the local damping.


% \subsection{Camera image}
% The most primitive sensor is that of the physical video camera.  The
% video camera monitors the location of the spot on the beam splitter,
% thus serving as a sensor of the pointing of the input beam. The image
% of the speckle of light reflected off of the beam splitter (see
% Fig. \ref{fig:BCS}) is fed into a labview program which integrates the
% intensity of the image to identify the coordinates of the center of
% the beam spot. This is compared with a hardcoded desired center
% location and a mirror upstream, MMT1 is moved to redirect the input
% beam, minimizing the difference between the desired and actual beam
% spot location on the BS.

% \begin{figure}
% \begin{centering}
% \includegraphics[width=0.8\textwidth]{figures/BCSspiricon.pdf}
% \caption{Image of beam on beam splitter as used for the beam centering servo.}
% \label{fig:BCS}
% \end{centering}
% \end{figure}


% \subsection{QPDs}  
% QPDX and QPDY see the small amount light that is transmitted through
% the ETMs, providing a monitor of the modal axes of the arm cavities. Together with the , they maintain the alignment of the input . QPDX 


% \subsection{WFSs}  
% The wavefront sensors provide the most sophisticated form of
% measuring angular motion of the mirrors. Their frame of reference is
% the fundamental Gaussian mode of the interferometer cavities (x-arm,
% y-arm, recycling cavity) as aligned to ???. All of the mirrors
% globally follow the pointing of the input beam (via the QPDs) which is
% in turn stabilized to the center of the beam splitter (via the BCS) at frequencies well
% below the pendulum resonances. The job of the wavefront sensors is to
% keep the mirrors aligned to one another up to several Hz within this
% hierarchy of alignment. The WFS provide DC and AC control to the
% mirror angles. 




% \section{Control}
% When the gain is really high, comparing the calibrated error and
% control signals shows just what the control loop is doing. The error
% is the residual and the control is what there would be without the
% loop.

% error * (1+G) is the motion without the loop, where G is the open loop gain.

% \begin{itemize} 
% \item optical lever calibration
% \item residuals, perhaps for different kinds of seismic
% \item compare to mirror motion with no ifo (demonstrate ASC suppression)
% \end{itemize}


The basis for angular control in Initial LIGO was the sensor
basis. The wavefront sensors are located such that they sense
common/differential ETM/ITM angular motion and the filters were
designed to feed back to those sets of motions. This does not, however,
lend itself to easily handling radiation pressure torque. Since the
WFS basis is not the radiation pressure eigenbasis of Section
\ref{sec:eigenbasis}, each control servo handled combinations of the
soft and hard modes. Due to radiation pressure, each mirror has two
resonances, a more complicated plant than that offered by the
eigenbasis which has a single resonance for each mode.

Since the soft mode is statically unstable, it needs high
(\textcolor{blue}{how high?}) DC gain at all times. In the WFS sensor
basis, that would mean uniformly increasing the gains of all of the
control loops, since the control of the soft mode is split amongst the
loops. The problem is that some of the WFS sensors are less good than
others, so a uniform gain increase would impress extra sensing noise
from all WFS on DARM. 

This poses a problem because increasing the gains impresses
sensing noise on the mirrors, affecting DARM, and impresses input beam
motion on the mirrors, making the interferometer less steady as a whole.

The meat of the Enhanced LIGO ASC upgrade was to switch the control servo
from the sensor basis to the natural radiation pressure eigenmode
basis, and to keep the contamination to DARM at a minimum. 



\section{Calibrations}
Since the data is collected digitally, the units are in digital
counts. This must be converted into physical units in order to make
meaningful statements beyond relative comparisons. Two ways to make a
calibration are to \textcolor{blue}{(think about this more!)}
\begin{itemize}
\item work backwards through the control system and electronics chain \vspace{-10pt}
\item inject an analog signal with a known amplitude 
\end{itemize}
I describe here the calibrations I made of some of the angular sensing
channels, all the while demonstrating particular examples of these
methods.


\subsection{Beam spot motion}
A quantity of interest is how much the beam moves on the ITMs and
ETMs. It is this beam spot motion which, together with the mirror
angular motion, creates a length signal that contributes noise to
DARM. An elegant way of following the motion of the beam on the test
masses is to track pickoffs of the light transmitted or reflected from
the mirrors. We have such signals naturally available for the ETMs and
ITMs from the QPDs which are otherwise used for ASC sensing. For
example, QPDX and QPDY see the light transmitted through each of the
ETMs and WFS2 sees the pickoff of light from the wedge of ITMX.

To calibrate the counts of the QPD and WFS2 pitch and yaw error
signals, \linebreak \texttt{L1:ASC-QPDY\_\{PIT,YAW\}\_IN1} and
\texttt{L1:ASC-WFS2\_\{DCPitchMon, DCYawMon\}}, I moved the beam a
known distance on the test mass, $\Delta x$, and recorded the
corresponding $\Delta y$ of the QPD and WFS2 readback. The ratio
$\Delta x /\Delta y$ is the calibration from counts to meters. The
details of the procedure are described below.


\subsubsection{Moving the beam} 
Moving the beam on the mirrors is straightforward because of the ASC
system. All that we need to do is introduce an offset to the setpoint
of the of the beam centering aspect of the ASC servo. For the ETMs we
put a DC offset in the \texttt{L1:ASC-QPD\{X,Y\}\_\{PIT,
  YAW\}\_\{OFFSET\}} channel and for the ITMs we changed the $X$ and
$Y$ targets of the beam splitter beam centering servo.

\subsubsection{Measuring how much the beam has moved} 
The more difficult task is measuring just how much the beam has
moved. For this, we make use of the lever arm mechanism of angle to
length coupling. \textcolor{blue}{(This should be explained in the
  previous chapter, so maybe just reference it.)} The idea is that
when the axis of rotation of a mirror coincides with the center of the
beam, any tilt of the mirror about this axis does not affect the path
length of the reflected beam. However, if there is a mismatch between
rotation axis and beam location, then the light will pick up a
longitudinal phase shift when the mirror is tilted. During a full
interferometer lock, this is recorded by DARM.

The concept of the measurement is to move the axis of rotation of the
mirror so that it passes through the center of the beam. We use the
OSEMs to change the location of the axis of rotation, and we use DARM
to determine when the axis is aligned with the beam center. For
example, if we drive the top two osems more than the bottom two osems,
we've created an axis of rotation that sits below the center of
mass. The result of such tuning is an effective rebalance of the
center of mass of the mirror so that it is aligned with the center of
the beam. The procedure is:
\begin{enumerate}
\item Shake the mirror at some frequency $f$ (we use
39.5 Hz) during a full lock \vspace{-10pt}
\item Demodulate DARM at $f$ for several different sets of osem gains \vspace{-10pt}
\item Fit a quadratic to the demodulated data to pinpoint the osem gains that
  minimize the coupling to DARM
\end{enumerate}

\begin{figure}
\begin{centering}
\includegraphics[width=0.5\columnwidth]{figures/geometry_mirror_osems.pdf}
\caption[Diagram of mirror and osem geometry]{Geometry of osems and
  mirror as used for calculating the location of the axis of rotation
  when the torques are unequal. \textcolor{blue}{Need latex in
    inkscape!}}
\label{fig:mirror_osem_geometry}
\end{centering}
\end{figure}

Relating the osem gains to absolute beam position on the mirror
requires only the geometry of the mirror and osem setup as sketched in
Fig. \ref{fig:mirror_osem_geometry}. We estimate the osem locations as
being on the edge of the mirror such that the length $d$ of one side
of the square that they form is given by $d =\sqrt{2} R$, where $R =
12.5$ cm is the radius of the mirror. Then, collapsing the four osems
into a representative two at the centers of two opposite sides of the
square and assigning them gains of $1 + \alpha$ and $-(1-\alpha)$ for
a force $F$, we can evaluate where the pivot point $x$ is located by
setting the sum of the torques equal to zero:
\begin{equation}
F [1+\alpha] x = F [1-\alpha][d-x].
\end{equation}
Therefore, the beam location relative to center, $\Delta x$, is
\begin{equation}
\Delta x := \frac{d}{2} - x = \alpha \frac{d}{2},
\end{equation}
and for a change in a pitch or yaw coil gain, the change in beam
position, $\Delta x$, is:
\begin{equation}
|\Delta{x}| = \frac{|\Delta{\mbox{gain}}| R}{\sqrt{2}} .
\end{equation}

The final calibrations of these channels are shown in Table
\ref{table:bsmcal}. \footnote{A minor
  technicality is that since there are no filters between the QPD
  error signals and the offset channel, their units are exactly the
  same. Thus, calculating meters of beam spot motion as a function of
  offset serves to calibrate the error point. For convenience, this is
  what I did.}

\begin{table}
\centering
\begin{tabular}{l l l l}
\hline
         & ETMX & ETMY & ITMs \\
\hline
pitch & 1.03e-5 m/ct & 1.21e-5 m/ct & 5.52e-2 m/ct \\
yaw & 0.88e-5 m/ct & 0.80e-5 m/ct & 4.79e-2 m/ct \\
\hline
\end{tabular}
\caption[Beam spot motion calibrations]{Calibrations to be used with
  the QPDX, QPDY, and WFS2 DC pitch and yaw error signals for a
  measure of beam spot motion.} 
\label{table:bsmcal}
\end{table}



\subsection{Angular mirror motion}
The optical levers provide a straightfoward measure of individual
mirror motion. The channels I calibrated were of the form
\texttt{L1:SUS-ETMX\_OPLEV\_\{P,Y\}ERROR}, the optical lever error
signals for each of the large optics. I made use of the dependence of
power in a misaligned cavity to calibrate the ETM and ITM optical
levers, and used a less precise, rudimentary method to calibrate the
RM, BS, and MMT3 optical levers.


\subsubsection{ETM and ITM optical levers} 
I calibrated the arm cavity optical levers by tracking the power loss
in the locked arm as one of its mirrors is tilted. The closed form
expression for cavity power as a function of mirror tilt is derived in
Appendix \ref{sec:cavitypower}. All that is needed is a quadratic fit
to the data collected. From the fit parameters, I can determine the
factor, $\Delta \theta / \Delta y$, which converts the digital counts
of the optical lever channel, $y$, to units of radians.

To make the measurement, I locked a single arm and maximized the power
build up. Then I slowly stepped the pitch or yaw pointing of one of
the mirrors away to one side of resonance, and then back and to the
other side, repeating this several times. All the while, I recorded
the optical lever error signal of the mirror whose angle I was
changing, and the power in the arm as determined from
\texttt{L1:LSC-NPTR\{X,Y\}\_OUT16}, the amount of light transmitted
through the ETMs.

From Eq. \ref{eq:pwr_disptilt}, we see that the power in the arm, $P$,
is a function of the form 
\begin{equation}
P = P_{max} \exp{[-b (y-y_0)^2]},
\label{eq:OLcalfit}
\end{equation}
where $b$ is related to physical cavity axis displacement $a$ and tilt
$\alpha$ by $by^2~=~(a/w_0)^2+(\alpha/\theta_0)^2$. In order to relate
the optical lever signal, $y$, to physical cavity parameters, we
divide by $\Delta{\theta}^2$ and rearrange to get:
\begin{equation}
\frac{\Delta \theta}{\Delta y} = \sqrt{b} \left[
  \left[\frac{\Delta a/\Delta\theta}{w_0}\right]^2 +
  \left[\frac{\Delta\alpha/\Delta\theta}{\theta_0}\right]^2
\right]^{-\frac{1}{2}} .
\end{equation}
The terms in the numerators on the right hand side are fixed constants
based on the cavity geometry and can be calculated using
Eq. \ref{eq:cavitydisptilt_mirrorangle}. The measurement data and fits
are shown for both pitch and yaw in Fig. \ref{fig:OLcal}. The ETM
optical levers make use of a broader range of optical lever signal
than do the ITMs. (Also note that the maximum power in the y-arm is about
10\% less than that in the x-arm. This is true at both Hanford and
Livingston, and is due to the priority given to the x-arm in the
alignment scheme, as explained in Appendix
\ref{sec:initial_alignment}.)

\begin{figure}
\begin{centering}
\subfigure{\includegraphics[width=0.5\columnwidth]{figures/olcal_pitch.pdf}}\subfigure{\includegraphics[width=0.5\columnwidth]{figures/olcal_yaw.pdf}}
%\includegraphics[width=1.0\columnwidth]{figures/oplevcal_pitch.pdf}
\caption[Optical lever calibration data.]{Optical lever calibration data and
  fits to Eq. \ref{eq:OLcalfit}.}
\label{fig:OLcal}
\end{centering}
\end{figure}


\subsubsection{RM, BS, and MMT3 optical levers}
To calibrate the RM, BS and MMT3 optical levers, I used my own eyes
and the camera images and known dimensions of the ETM beam cages. With
the interferometer unlocked for the BS and MMT3 calibrations, and the
power recycled Michelson locked for the RM calibration, I moved the
optics in pitch and yaw, tracking the beam's movement on the ETM
cages. For yaw, I moved the mirrors until the beam was centered on
each vertical suspension post, and for pitch I moved the beam from the
center of the mirror to the top of the cage. Small angle
approximations and geometry are the only tools needed to relate beam
displacement to mirror angle.

The final $\Delta\theta / \Delta x$ calibrations of all optical levers
are in Table \ref{table:oplevcal}. 

\begin{table}
\centering
\caption[Optical lever calibrations]{Calibrations to be used
  with the optical lever error signals 
  for a measure of angular mirror motion. Units are \microrad/ct.}
\begin{tabular}{l l l l l l l l}
\hline
        & ETMX & ETMY & ITMX & ITMY & RM & BS & MMT3 \\
\hline
pitch & 49.4 & 43.0 & 14.9 & 15.6 & 61.9 & 47.3 & 57.4 \\
yaw & 50.7 & 43.3 & 20.1 & 20.2 & 42.5 & 63.5 & 55.5 \\
\hline
\end{tabular}
\label{table:oplevcal}
\end{table}





\subsection{WFS error signals}
The WFS error signals are physically Watts of power at the
detectors. Converting the digital counts to Watts requires working
backwards through the electronics, and can be divided into two
parts. First, we must calibrate the WFS demodulation chain to
backtrack WFS counts into voltage of signal at the input to the
mixer. Second, we must convert voltage at the mixer into Watts at the
sensor based on properties of the photodetector RF electronics.

\subsubsection{Counts to Volts} 
I used an HP signal generator to create an RF sine wave at 100 Hz away
from the local oscillator frequency and put this into each quadrant of
the demod boards. Comparing the peak to peak voltage of this input
sine wave to the peak to peak amplitude in counts as recorded by a dtt
time series gives the Volts per count conversion.

For efficiency purposes in making the measurement, we put the output
of the HP into an RF splitter so that we could inject a signal into 2
quadrants at once. For WFS1 and WFS2, I dialed the frequency to
24.48433100 MHz with an amplitude of -30~dBm. With an oscilloscope, I
measured each output of the RF splitter to have 13.6~mV peak to
peak. For WFS3 and WFS4, I used a frequency of 61.21097360~MHz and
amplitude of -30~dBm, which read out after the splitter as a 14 mV
peak to peak. The calibrations are presented in Table
\ref{table:demodcal}.

\begin{table}
\centering
\begin{tabular}{l l l l l l}
 & q1 & q2 & q3 & q4 & average \\
\hline\hline
 WFS1  & 0.35 &   0.32 &   0.34 &   0.35 & 0.34 \\
 WFS2  & 8.8 &   8.6 &   8.7  &  8.5 & 8.7       \\
 WFS3  & 6.4 &   5.8 &   5.8 &   5.7 & 5.9     \\
 WFS4  & 6.3 &   5.4 &   5.3  &  7.4 & 6.1    \\
\hline
\end{tabular}
\caption[Demodulation chain calibration for each quadrant of each
  WFS]{Demodulation chain calibration for each quadrant of each
  WFS. Units are \micro V/count.}
\label{table:demodcal}
\end{table}

It should be noted that the demod chain calibration numbers for all
quadrants of a particular WFS are about the same, meaning the
demodulation chain does not distort the error signals. 


\subsubsection{Volts to Watts}
The number of volts of signal created by a certain power on the WFS is
determined by the transimpedance (Volts/Amp) and responsivity
(Amps/Watt) of the diode. A model of the 25 MHz WFS transimpedance is
shown in Fig. \ref{fig:wfs25MHzTF}. The responsivity is detailed in
Section \ref{sec:pds}.

\begin{figure}
\begin{centering}
%\includegraphics[width=1.0\textwidth]{plots/wfs25MHzTF.pdf}
\caption{LISO model of the 25 MHz resonant RF WFS front end electronics.}
\label{fig:wfs25MHzTF}
\end{centering}
\end{figure}






\section{Optical Gain}
The WFS optical gain is a measure of how sensitive wavefront sensors
are to mirror motion.  We want to know how much power in Watts a WFS
detects when a particular mirror, or combination of mirrors, is
excited by a certain number of radians. We measure the WFS optical
gain all the time during commissioning, yet under the disguise of the
name \emph{sensing matrix}. In practice, the sensing matrix is never
calibrated into real units, but we do that here for enlightenment
using the calibrations of the previous section.




\subsection{Diagonalizing the ASC drive matrix}







\section{Input beam motion}
%\textcolor{blue}{Breakdown of source of mirror motion.}

The beam centering servo only operates up to about 10~mHz, meaning the
beam-centering degree of freedom is uncontrolled at higher
frequencies. The source of beam de-centering on the mirrors is input
beam motion. The HAM seismic isolation tables from which the input
optics are suspended have resonant ``stack'' modes from about 0.8~Hz
to 3~Hz. The excess table motion at these frequencies is transmitted
to the MMTs. 

I measured the impression of the input beam motion on the mirrors by
increasing the gain of the common-degree-of-freedom WFS servos (CH,
CS, RM) for about 10~minutes. Comparing the amount of angular
motion of the mirrors from this time of high common WFS gain to a time
with nominal WFS gain and similar seismic motion, we can see the
effect directly. Fig. \ref{fig:inputbeam_impression} shows comparison
spectra, demonstrating how there is higher test mass motion around
1~Hz when the common WFS gains are higher.

\begin{figure}
\begin{centering}
\subfigure{\includegraphics[width=0.5\textwidth]{figures/ETMX_highnom.pdf}}\subfigure{\includegraphics[width=0.5\textwidth]{figures/ITMX_highnom.pdf}}
\subfigure{\includegraphics[width=0.5\textwidth]{figures/ETMY_highnom.pdf}}\subfigure{\includegraphics[width=0.5\textwidth]{figures/ITMY_highnom.pdf}}
\subfigure{\includegraphics[width=0.5\textwidth]{figures/RM_highnom.pdf}}
\caption[Impression of input beam motion on the core mirrors]{Input
  beam motion impression on the core mirrors. Mirror motion when the
  common WFS gains (WFS2a, WFS3, WFS4) are increased by a factor of
  2.5 is compared to mirror motion when the WFS gains are
  nominal. Both spectra come from a time of similar seismic activity
  (typical weekday afternoon noise). \textcolor{blue}{But what is the point?
    The modal WFS spectra simply show more suppression, as expected!}}
\label{fig:inputbeam_impression}
\end{centering}
\end{figure}

A more quantitative study of the effects of input beam motion is to
measure transfer coefficients between the input beam motion and the
mirror angular (or beam spot) motion. During a full interferometer
lock I put lines in MMT1, MMT2, and MMT3 at 1.05~Hz, 1.25~Hz, and
0.85~Hz, respectively, selecting excitation amplitudes large enough to
appear in the common WFS spectra. I compared the amplitudes of the
lines in the MMT spectra with those in the WFS spectra. The counts to
counts transfer coefficients are shown in Table
\ref{table:inputbeam_TFcoeffs}. Since sensing is flat, I use the shape
of the WFS loop to extrapolate this transfer coefficient at one
frequency to other frequencies. The result is a transfer function
useful for making a noisebudget of input beam motion to test mass
motion. Selecting a science mode time of typical day-time seismic
noise, I made a noisebudget as found in Fig. \ref{fig:inputbeam_NB}.


\begin{table}
\centering
\caption[MMT to WFS transfer coefficients]{MMT to WFS transfer
  coefficients. \textcolor{blue}{Make this!!!}}
\begin{tabular}{l l l l l}
\end{tabular}
\label{table:inputbeam_TFcoeffs}
\end{table}



\begin{figure}
\begin{centering}
%\includegraphics[width=1.0\columnwidth]{figures/}
  \caption[Mirror motion due to input beam impression]{Noise budget of
    mirror motion due to input beam impression. \textcolor{blue}{Make
      this!!}}
\label{fig:inputbeam_NB}
\end{centering}
\end{figure}


\textcolor{blue}{Wiener filter beam spot motion to show corner seismic is primary
contributor when the WFS gains are high. Also show seismic noise 
coherent with mirror motion in general.}



\section{The marginally-stable Power Recycling Cavity}
The power recycling cavity (PRC) is the linear cavity formed by the RM
and ITMs. Since the radius of curvature of both the RM and the ITMs
points in the same direction, the cavity is geometrically
unstable. For example, in its cold state at LLO the $g$-factor of the
cavity is 1.00005 and at LHO it's 1.00003. The beam in the PRC is not
spatially contained and may be made of many higher order modes. The
heating of the ITMs from the kilowatts of power in the arm cavities
together with the ITM thermal compensation system (TCS) serve the role
of making the PRC geometrically stable for interferometer
operation. The heating and cooling of the ITMs is a very complicated
process and therefore not very precise, so the value of the hot PRC's
$g$-factor is usually not constant.

The changing $g$-factor has potentially severe consequences for the
ASC. Because of its geometry, the power build-up in the PRC is very
sensitve to both the mirror angles and the $g$-factor. Power
fluctuation is detrimental because the signal to noise ratios of the
sensors that probe the PRC light degrade due to the presence of
increased junk light that contributes shot noise but not
signal. WFS1Q, WFS2I, and WFS2Q are the most sensitive to the PRC
because their signals are derived from the 25~MHz sidebands. Their
sensitivity to mirror motion is therefore subject to change. Since
achieving a flat power build-up in the PRC is a difficult task (too
much motion in the PRC is quite often a cause of lock loss), we must
update the real-time control system to reflect their changing
sensitivites. Otherwise, the mirror angles will not be accurately
controlled.

An estimate of the expected power fluctuations based on the $g$-factor
and RM motion is a straightforward excercise when using
Eq. \ref{eq:cavitydisptilt_mirrorangle} and Eq. \ref{eq:pwr_disptilt}
as derived in the Appendix. If we estimate the $g$-factors of the RM
and ITM as $g_{RM} = 1+\delta$ and $g_{ITM} = 1 - \delta$ ($\delta =6
\times 10^{-4}$ for LLO the cold state) and approximate the distance
of each mirror to the cavity waist as $z$ since the two mirrors are
very close to each other compared to the waist location, then
Eq. \ref{eq:cavitydisptilt_mirrorangle} reduces to:
\begin{equation}
\left\llbracket \begin{array}{c}
a_{PRC} \\
\alpha_{PRC} \end{array} \right\rrbracket = 
\left\llbracket \begin{array}{cc}
z(2+\delta)/\delta & z(2-\delta)/\delta \\
-1/\delta & -1/\delta \end{array} \right\rrbracket
\left\llbracket \begin{array}{c}
\theta_{RM}\\
\theta_{ITM} \end{array} \right\rrbracket.
\end{equation}

Fig. \ref{fig:prc_power} plots the power in the PRC as a function of
$\theta_{RM}$ for several values of $\delta$, demonstrating
the sensitivity of the PRC to the ITM heating.  For example, the
typcial RM angular displacement of $10^{-7}$ rad results in a 66\%
power loss when the PRC $g$-factor is very near instability with a
value of $1-0.0001$. Only as the $g$-factor moves further from $1$
does the angular motion of the RM have less and less of an effect on
the power build-up.

\begin{figure}
\begin{centering}
\includegraphics[width=1.0\columnwidth]{figures/prc_power.pdf}
\caption[Theoretical dependence of power recycling cavity power on
$g$-factor and mirror angle]{Dependence of power build-up in the power
  recycling cavity on the PRC's $g$-factor and the RM tilt. TCS is
  necessary for stabilizing the PRC's geometry and therefore its
  sensitivity to mirror motion. For simplicity, the ITM is assumed
  stationary in these plots.}
\label{fig:prc_power}
\end{centering}
\end{figure}


\subsection{SPOB power scaling}
The effect of the marginally stable recycling cavity can be seen by
tracking the ASC sensing matrix elements as $g$ changes. I excited
three of the test masses (ETMX, ITMX, RM) at three different
frequencies (9.7~Hz, 10.7~Hz, and 11.7~Hz, respectively) during a full
interferometer lock and changed the TCS settings so that over the
course of 15 minutes the $g$-factor steadily changed. Demodulating
each of the WFS signals at each of the three excitation frequencies as
a function of time shows the strength of the WFS response to the motion
of these three mirrors over the course of the study. To compensate for
the difference in pendulum responses to the excitations, I multiplied
the demodulated signals for a particular excitation $f$ by
$(f/9.7)^2$. I also normalized the response by the phase of the
mirror's motion as witnessed by the optical levers.

The results are shown in Fig. \ref{fig:WFStrack}. As expected, WFS1Q,
WFS2I, and WFS2Q show dependence on the PRC power, and therefore the
$g$-factor. The WFS3 and WFS4 sensing elements are flat. The power in
the PRC as measured by the $2f$-demodulated POB signal, NSPOB,
is also shown in Fig. \ref{fig:WFStrack} for this time period. In
order to compensate for this $g$-factor dependence, we multiply the
WFS\{1Q, 2I, 2Q\} error signals in real-time by
\begin{equation}
\frac{1}{P_{in}} \left[\frac{NSPOB}{350}\right]^{-1/2}
\end{equation}
and WFS3I and WFS4I by $1/P_{in}$, where the 350 is the reference
NSPOB, treated as nominal. Thus, during interferometer operation, all
WFS signals are normalized to input power and are not dependent on the
PRC power. This correction to the WFS signals is called SPOB power
scaling.



\begin{figure}
\begin{centering}
\subfigure{\includegraphics[width=0.5\textwidth]{figures/spob_gaintracking.pdf}}\subfigure{\includegraphics[width=0.5\textwidth]{figures/WFS1Q_track.pdf}}
\subfigure{\includegraphics[width=0.5\textwidth]{figures/WFS2I_track.pdf}}\subfigure{\includegraphics[width=0.5\textwidth]{figures/WFS2Q_track.pdf}}
\subfigure{\includegraphics[width=0.5\textwidth]{figures/WFS3I_track.pdf}}\subfigure{\includegraphics[width=0.5\textwidth]{figures/WFS4I_track.pdf}}
\caption[Measured dependence of the WFS error signals on the power
recycling cavity geometry]{Measurement of the dependence of the WFS
  error signals on the power recycling cavity geometry. WFS1Q, WFS2I,
  WFS2Q are more sensitive to test mass motion as the power in the
  recycling cavity increases. To achieve a dependable feedback system,
  the error signals are scaled in real-time, forcing their responses
  to be flat with power. This range of SPOB is low for normal
  operations.}
\label{fig:WFStrack}
\end{centering}
\end{figure}


\subsection{Sideband imbalance}
Show OSA scan of AS port.






\section{DC readout related measurements}
\begin{itemize}
\item RF created from DC offset beam moving on WFS1
\item RF vs DC vs power comparison of (AS) beam spot motion on WFS1
\end{itemize}


% \section{ASC noisebudget}
% \begin{itemize}
% \item seismic - breakdown of soure of motion
% \item L2A
% \item input beam 
% \item electronics noise
% \item shot noise
% \end{itemize}




\section{The effect of the ASC}


\begin{figure}
\begin{centering}
%\includegraphics[width=0.8\columnwidth]{figures/olgs6W.pdf}
\subfigure{\includegraphics[width=1.0\columnwidth]{figures/allolgs_6W_mags.pdf}}
\subfigure{\includegraphics[width=1.0\columnwidth]{figures/allolgs_6W_phases.pdf}}
\caption{Open loop gains (pitch) of the 5 WFS loops as measured with 6 W
  input power.}
\label{fig:olgs6W}
\end{centering}
\end{figure}


\begin{figure}
\begin{centering}
\subfigure{\includegraphics[width=1.0\columnwidth]{figures/wfs1_olgs_mags.pdf}}
\subfigure{\includegraphics[width=1.0\columnwidth]{figures/wfs1_olgs_phases.pdf}}
\caption{Open loop gains (pitch) of the differential soft (WFS1) loop as measured at four
  different powers.}
\label{fig:DSolgs}
\end{centering}
\end{figure}

\begin{figure}
\begin{centering}
\subfigure{\includegraphics[width=1.0\columnwidth]{figures/wfs2b_olgs_mags.pdf}}
\subfigure{\includegraphics[width=1.0\columnwidth]{figures/wfs2b_olgs_phases.pdf}}
\caption{Open loop gains (pitch) of the differential hard (WFS2B) loop as measured at four
  different powers.}
\label{fig:DHolgs}
\end{centering}
\end{figure}

\begin{figure}
\begin{centering}
\subfigure{\includegraphics[width=0.5\textwidth]{figures/onoff_DS.pdf}}\subfigure{\includegraphics[width=0.5\textwidth]{figures/onoff_CS.pdf}}
\subfigure{\includegraphics[width=0.5\textwidth]{figures/onoff_DH.pdf}}\subfigure{\includegraphics[width=0.5\textwidth]{figures/onoff_CH.pdf}}
\subfigure{\includegraphics[width=0.5\textwidth]{figures/onoff_RM.pdf}}
\caption[Angular motion suppression due to the ASC]{Propagation of
  sensor signals from 10 W lock through input matrix and power scaling
  to eigenbasis, compared with eigenbasis reconstruction of optical
  lever signals when interferometer not locked, but optics under oplev
  damping. Data are taken 45 minutes apart. \textcolor{blue}{Perhaps
    include loop-undone backgrounds, too, using my measured OLTFs.}}
\label{fig:}
\end{centering}
\end{figure}


% \begin{figure}
% \begin{centering}
% \includegraphics[width=0.7\textwidth]{figures/ASCdofsignals.pdf}
% \caption{Top: Comparison of motion with and without the
%   ASC. Eigenbasis residual during 10 W lock, and background derived
%   from loop correction. Completely different day from next
%   plot. (Perhaps merge with next plot)}
% \label{fig:}
% \end{centering}
% \end{figure}




\section{ASC to DARM noisebudget}
Broadband effect on DARM.

\subsection{Tuning the cut-off filters} 
The cut-off frequency of the lowpass filters for the WFS control are
of particular importance in the DARM noisebudget. The lowpass filter
is necessary for suppressing the impression of sensing noise on
suspension control. 

\begin{figure}
\begin{centering}
\includegraphics[width=1.0\textwidth]{figures/cutoffWFS1_DARMcompare.pdf}
\caption[Effect of the WFS1 lowpass filter cutoff frequency on strain
sensitivity.]{Effect of the WFS1 lowpass filter cutoff frequency on
  strain sensitivity.}
\label{fig:WFS1cutoff}
\end{centering}
\end{figure}




\begin{figure}
\begin{centering}
\includegraphics[width=1.0\columnwidth]{figures/ASC2DARM_TFs.pdf}
\caption[Measured ASC to DARM transfer functions]{ASC to DARM transfer
  function for four of the five wavefront sensor loops. The RM to DARM
  transfer function could not be measured because the contribution is
  so small. The fitted curves can be multiplied by the WFS error
  signals at any time to calculate the ASC noise contribution to
  DARM. \textcolor{blue}{Need to turn cts into Watts.}}
\end{centering}
\end{figure}




\section{Feed-forward}


\section{Advanced LIGO}
The Advanced LIGO interferometers will have heavier mirrors, a stable
recycling cavity, and more circulating power. 