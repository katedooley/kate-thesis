\chapter{THE SEARCH FOR GRAVITATIONAL WAVES}

%\textcolor{blue}{Write 1-2 pages providing a synopsis of the history
%of GWs and programs to detect them.}

Einstein's predictions of general relativity opened to the scientific
community a whole new window of how to look at the universe. Just as
scientists had been building detectors to observe directly optical and
microwave radiation, they now had the theory in hand to think about
building detectors for gravitational radiation. Gravitational waves
(GW) are dynamic strains in space-time that travel at the speed of
light and are generated by non-axisymmetric acceleration of mass.

Joseph Weber of the University of Maryland and John Wheeler of
Princeton University introduced the field of gravitational wave
astronomy in the 1960s. Weber built a resonant bar, the first
instrument designed to directly observe gravitational waves
\cite{Weber1960Detection}. Although his bar never made a positive
detection, the interest in directly detecting gravitational waves
persisted, and new and more sensitive detector designs were conceived.

The most promising of new detector designs for measuring a
gravitational wave's distortion of space-time proved to be a laser
interferometer. Robert Forward of Hughes Spacecraft built the first
bench top prototype in the 1970s \cite{Forward1978Wideband}. Rai Weiss
of M.I.T. and Ron Drever of Caltech with the aid of others developed
this concept into what is becoming a worldwide array of large scale
interferometers.

The field of ground-based gravitational-wave physics is rapidly
approaching a state with a high likelihood of detecting GWs for the
first time. Such a detection will not only validate part of Einstein's
general theory of relativity, but initiate an era of astrophysical
observation of the universe through GWs.  A first detection is
expected to witness an event such as a binary black hole/neutron star
merger. This chapter provides the theoretical framework of
gravitational wave generation and presents various ways to detect GWs,
including the current status of an effort to do so. I explain the
purpose of this dissertation in the context of these current effort.


\section{The Theory of Gravitational Radiation}
Gravitational radiation is a perturbation $|h_{\mu \nu}| \ll 1$ to the
flat space-time Minkowski metric $\eta_{\mu \nu} = \mbox{diag}(-1, 1,
1, 1)$ \cite{Carroll1997Lecture}. The metric describing space-time in
the presence of gravitational radiation is therefore
\begin{equation}
g_{\mu\nu} = \eta_{\mu\nu} + h_{\mu\nu}.
\end{equation}
As in electrodynamics where one has freedom in choosing the
vector potential $\vec{A}$ for calculating the magnetic field $\vec{B}
= \vec{\nabla} \times \vec{A}$, one also has freedom in general
relativity in choosing the form of $h_{\mu \nu}$ for ease of calculation. A
convenient and popular choice is called the transverse-traceless (TT)
gauge in which
\begin{equation}
h_{\mu \nu} = 
\left\llbracket \begin{array}{c c c c} 
0 & 0 & 0 & 0\\ 
0 & h_+(t) & h_\times(t) & 0 \\
0 & h_\times(t) & -h_+(t) & 0 \\
0 & 0 & 0 & 0
\end{array} \right\rrbracket
\end{equation}
where the $+$ and $\times$ represent two linearly independent
polarizations. Without loss of generality, we consider the $h_+$
polarization in the example that follows.

For a gravitational wave traveling along the $z$-axis, the space-time
metric is given by:
\begin{equation}
ds^2 = -c^2dt^2 + [1+h_+(t)] dx^2 + [1-h_+(t)] dy^2.
\end{equation}
This says the TT coordinate system is stretched along the $x$ axis and
and compressed along the $y$ axis by a factor of 
\begin{equation}
\sqrt{1 \pm h_+(t)} \approx 1 \pm \frac{1}{2} h_+(t).
\label{eq:deltaL}
\end{equation}
Therefore, for two free masses located a proper distance $L$ from one
another along either the $x$-axis or the $y$-axis, their separation is
magnified by the factor in Eq.~\ref{eq:deltaL} in the presence of a
gravitational wave. Their coordinate separations, however, remain
constant. The gravitational wave perturbation is a dimensionless
strain:
\begin{equation}
h_+(t) = 2 \frac{\Delta L(t)}{L},
\end{equation}
where $\Delta L(t)$ is the change in separation between the free
masses, as illustrated in Fig.~\ref{fig:strain}

\begin{figure}
\begin{centering}
\includegraphics{figures/strain.pdf}
\caption[Depiction of strain]{Depiction of strain.}
\label{fig:strain}
\end{centering}
\end{figure}



\section{Sources}
Any object with an accelerating mass quadrupole moment generates
gravitational waves. The typical strain amplitudes, however, are
extremely tiny: a binary system of coalescing $1.4 \mbox{M}_\odot$
neutron stars in the Virgo Cluster (a distance of 15~Mpc) would
produce a maximum GW strain on Earth of only $10^{-21}$ at 800~Hz
\cite{Saulson1994Fundamentals}. The strain is proportional to source
mass, $M$, and velocity, $v$, and inversely proportional to the
distance from the observer, $R$:
\begin{equation}
h \approx \frac{GMv^2}{Rc^4}
\end{equation}
The quantity $G/c^4$ is what sets the scale for strain amplitudes
because of how small it is: $8.26 \times
10^{-45}$~m$^{-1}$kg$^{-1}$s$^2$. Consequently, the most promising
sources of detectable gravitational waves are nearby, fast-moving,
massive astrophysical objects that include
\begin{itemize}
\item supernovae \cite{Abbott2009Search} \vspace{-10pt}
\item binary stars (orbiting or coalescing) \cite{S5CBCnospin} \vspace{-10pt}
\item spinning neutron stars \cite{Abadie2010First} \vspace{-10pt}
\item cosmological/astrophysical background \cite{Allen1999Detecting}
\end{itemize}
and can be categorized as producing periodic, burst, or stochastic
GWs. 

Stably orbiting binary star systems comprised of black holes or
neutron stars as well as rapidly spinning non-axisymmetric pulsars are
considered periodic sources since they will produce GWs of relatively
constant frequency. These reliable sources of GWs require a long
integration time to pick out their signal above noise. 
%The Hulse-Taylor binary, for instance, falls into this category. 
Supernovae are burst sources since the gravitational
collapse will produce a short-lived, unmodeled emission of
GWs. Binaries in their final tens of milliseconds of inspiral also
fall into this category. Finally, the anisotropies in the inflation of
the universe together with the hum of all distant astrophysical
sources will create a stochastic background of radiation. Coherent
cross-correlation between multiple detectors is necessary for
measuring the constant amplitude, broad-spectrum GW
background \cite{Maggiore2000Gravitational}.

Directly detecting gravitational radiation from any such source will
reveal information that electromagnetic radiation cannot convey. The
frequency of the GW tells about the dynamical timescale of the
source. Only through GW radiation, for example, can mass and spin
properties of a black hole be revealed. A first detection is expected
to witness an event such as a binary black hole/neutron star
coalescence \cite{Abadie2010Predictions}.





\section{Methods of Detection}
In order to detect directly a gravitational wave, the instrument must
be sensitive to strain. Weber's bar and laser interferometers both
accomplish this requirement. There is a third method of detection,
however, that has already proved successful, although the detection is
not direct. Hulse and Taylor observed the rate of change of the
orbital period of a binary star system, demonstrating beautifully a
precise agreement with the predictions of GR should the rate of change
be due to gravitational radiation \cite{Hulse1975Discovery,
  Weisberg2005Relativistic}. Awarded the Nobel Prize for their work,
Hulse and Taylor's indirect evidence of GWs has fueled the field to
produce a direct detection. Newer methods under active research
include pulsar timing \cite{Hobbs2009International} and B-mode
measurements of the cosmic microwave background polarization. For an
approachable overview of the history of the field, including detector
design choices and estimated GW strain amplitudes of various sources,
refer to Ref. \cite{Linsay1983Study}.






\section{State of Ground-based Interferometry}
A network of first generation kilometer-scale laser interferometer
gravitational-wave detectors completed its integrated 2-year data
collection run in 2007, called S5. The instruments were: the American
Laser Interferometer Gravitational-wave Observatories (LIGO)\cite{Abbott2009LIGO},
one in Livingston, LA with 4 km long arms and two in Hanford, WA with
4~km and~2 km long arms; the 3~km French-Italian detector
VIRGO\cite{Acernese2008Virgo} in Cascina, Italy; and the 600~m
German-British detector GEO\cite{Luck2006Status} in Ruthe, Germany. Multiple
separated detectors increase detection confidence through signal
coincidence and improve source localization through triangulation.

The first generation of LIGO, known as Initial LIGO, achieved its
design goal of sensitivity to GWs in the 40~Hz - 7000~Hz band which
included a record strain sensitivity of
$2\times10^{-23}/\sqrt{\mathrm{Hz}}$ at 155~Hz. However, only the
closest of sources produce enough GW strain to appear in LIGO's band,
and no gravitational wave has yet been found in the S5 data. A second
generation of LIGO detectors, Advanced LIGO, has been designed to be
at least an order of magnitude more sensitive at several hundred Hz
and above and include an impressive increase in bandwidth down to
10~Hz, dramatically increasing the chances of detection. The baseline
Advanced LIGO design \cite{AdvLigoSysDesign} improves upon Initial
LIGO by featuring better seismic isolation, the addition of a signal
recycling mirror at the output port, homodyne readout, and an increase
in laser power from 10~W to 165~W.

To test some of Advanced LIGO's new technologies so unforeseen
difficulties could be addressed and so that a more sensitive data
taking run could take place, increasing the chances of detection, an
incremental upgrade to the interferometers was carried out after S5
\cite{Adhikari2006Enhanced}. This project, Enhanced LIGO, culminated
with the S6 science run from July 2009 to October 2010.  An output
mode cleaner was designed, built and installed, and DC readout of the
GW signal was implemented \cite{Fricke2011DC}. An Advanced LIGO active
seismic isolation table was also built, installed, and tested
\cite[Ch. 5]{KisselThesis}. In addition, the 10~W Initial LIGO laser
was replaced with a 35~W laser
\cite{Frede2007Fundamental}. Accompanying the increase in laser power, 
the test mass Thermal Compensation System \cite{Willems2009Thermal},
the Alignment Sensing and Control, and the Input Optics were modified.

As of the writing of this dissertation (September 2011), construction
and installation of Advanced LIGO is underway. The vacuum systems are
being retro-fitted to accompany the new layout, and at LLO the 165~W
laser has been installed. At both sites, the new seismic isolation
platforms and multi-level suspension cages are being
mass-produced. The By 2012, the first of the suspended mirrors will be
installed and testing begun. Simultaneously, VIRGO and GEO are both
undergoing their own upgrades as well \cite{Acernese2008Virgo,
  Luck2010Upgrade}. Figure \ref{fig:h_all} shows the achieved and
theoretical future noise curves of this network of ground-based GW
detectors.

\begin{figure}
\begin{centering}
\includegraphics[width=1.0\textwidth]{figures/GWnetwork_thesis.pdf}
\caption[Strain sensitivities of LIGO-VIRGO collaboration
interferometers]{Strain sensitivities of LIGO-VIRGO collaboration
  interferometers. Solid lines show achieved detector noise floors and
  dashed lines show design noise floors for future generation
  interferometers.}
\label{fig:h_all}
\end{centering}
\end{figure}


\section{Motivation for this Work}
The purpose of this work is to demonstrate the capability of an
interferometric gravitational wave detector to operate at several
times the highest of laser powers previously used. From a na\"ive
standpoint, more power is desirable since strain sensitivity improves
by $\sqrt{P}$ in the high frequency ($>$ 200~Hz) shot-noise-limited
region. However, as detectors become more sensitive at low frequencies
($<$ 40 Hz) in the years to come, radiation pressure noise will become
the dominant noise source there, making high laser power operation a
design trade-off. Currently, seismic noise limits low frequency
sensitivity, so exploring the technical world of increasing the laser
power is a fruitful adventure.

%Although offering an improvement to the shot-noise-limited sensitivity, 
More power introduces radiation pressure and thermally induced side
effects that must all be addressed for effective interferometer
operation. Concerns about the practical difficulties of handling high
power effects abounded during Initial LIGO when operating at the
design power of 10~W proved more difficult and less straight-forward
than expected. To achieve the Advanced LIGO design sensitivity, an
ambitious 160~W of input power is needed. Without an understanding of
the thermal and radiation pressure problems at 10~W, Advanced LIGO
becomes a daunting goal.

The work presented in this dissertation was carried out during
Enhanced LIGO to verify and investigate the predicted and unforeseen
effects of as much as 25~W of laser power. It also served the purpose
of enabling the operation of LIGO at higher powers and record strain
sensitivities. I present the design and the measurements I made of the
performance of two of the interferometer subsystems that are affected
by an increase in laser power: the Input Optics and the Angular
Sensing and Control. I show that the thermal and radiation pressure
effects on these subsystems are well understood. This work on the
Enhanced LIGO detectors informs design choices for Advanced LIGO.

%\textcolor{blue}{unsolvable, pick different word} 

% As an
% epilogue, I make projections of what can be expected for Advanced LIGO
% based on the results of my experimental studies. 


\subsection{The Input Optics High Power Story} 
The performance of the Initial LIGO Input Optics degraded with input
power as the result of absorbing too much heat. Particular
issues that needed to be addressed for any further increase in power
included thermal steering of the beam rejected by the interferometer,
a decrease in the optical isolation, and thermal lensing that drove
the spatial mode of the beam directed at the interferometer away from
optimal. We replaced two of the key Input Optics components and
modified the others. I describe the design of the improved Input
Optics for Enhanced LIGO which includes less absorptive optical
components in order to conquer thermal issues at the source and
changes to the design architecture that compensate for any residual
effects. I also present the set of measurements I made to characterize
the Input Optics performance with up to 30~W input power. I show that
we can expect the design of the Enhanced LIGO Input Optics to also
perform well for Advanced LIGO.



\subsection{The Angular Sensing and Control High Power Story}
Radiation pressure creates torques, a long-known concept, and the
optical torque's ability to de-stabilize optical cavities was first
recognized in 1991 by Solimeno et
al. \cite{Solimeno1991FabryPerot}. However, the theory of radiation
pressure's effect on angular mechanical transfer functions was not
fully appreciated until 2006, in the paper by Sidles and Sigg
\cite{Sidles2006Optical}. The concern arose that radiation pressure
might be the factor limiting Initial LIGO's ability to increase the
input power. Eiichi Hirose showed that the optical torque was present
and measurable, but that it was \emph{not} limiting at Initial LIGO's
power \cite{Hirose2010Angular}.\footnote{In fact, after the Enhanced
  LIGO laser was installed, and before any changes were made to the
  ASC, we successfully operated the interferometer with 14~W input
  power.} % Nov. 20, 2008 elog.
The concern of the optical torque's role in cavity dynamics shifted to
Enhanced and Advanced LIGO, which were designed to operate at four
times and 20 times the laser power of Initial LIGO, respectively. Lisa
Barsotti developed a numerical model of the angular sensing and
controls for Enhanced LIGO, specifically including radiation pressure
torque. She showed that, in principle, the radiation pressure torque
can be controlled without detrimental consequences to the sensitivity
of the detector \cite{Barsotti2009Modeling}. We implemented Barsotti's
theoretical control scheme and I measured its performance with up to
20~W of input power, demonstrating a thorough understanding of the
principles at work and providing confidence in the ability to control
radiation pressure torques in Advanced LIGO. I also improved upon
Hirose's measurement of the optical angular (anti-)spring. In
addition, through post-analysis of angular data, I demonstrate the
potential of a technique that may be used in Advanced LIGO for
reducing the angular control signals.



% \subsection{Outline of this Dissertation}
% The organization of this dissertation is as follows. Chapter~2
% describes the measurement apparatus, the LIGO interferometer, focusing
% on key aspects which are relevant for the rest of the thesis,
% including the motivation for more laser power. Chapter~3 presents the
% design and performance of the Enhanced LIGO Input Optics with up to
% 30~W input power. Chapter~4 introduces the Angular Sensing and
% Control, describing the causes of angular mirror motion, why angular
% motion matters, and how it was sensed and controlled in Initial
% LIGO. Chapter~5 derives the coupled cavity eigenfunctions that result
% from radiation pressure torque, setting the stage for changes to the
% ASC to accompany high power. Chapter~6 presents the characterization
% and performance of the Enhanced LIGO ASC with up to 20~W input power,
% and Chapter~7 shows the direct measurement of the optical
% (anti-)spring. A summary and conclusions are in Chapter~8.

