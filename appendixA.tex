%If you have a single appendix, you need to change {\chapter*{APPENDIX: THIS IS THE FIRST APPENDIX}
%to {\chapter*{APPENDIX: YOUR APPENDIX TITLE HERE} if you have two or more appendices
%you need to change {\chapter{THIS IS THE FIRST APPENDIX}} to
%{\chapter{YOUR APPENDIX TITLE HERE}}
%
%If you make these changes correctly Latex will complain bitterly about the additions to the TOC
%but will make them correctly in a manner acceptable to the Editorial Office.

\ifthenelse{\value{noa} = 1}
%...................then
{\chapter*{APPENDIX: THIS IS THE FIRST APPENDIX}
\addcontentsline{toc}{chapter}{APPENDIX: THIS IS THE FIRST APPENDIX}
\chaptermark{Appendix}
\markboth{Appendix}{Appendix}
\setcounter{chapter}{1}}
%...................else


{\chapter{Overlap integrals}}
A measure of the mode matching can be given by the amount of power
coupled from one mode into another. This is calculated as the square
of the overlap integral of two fields, $\psi_1$ and $\psi_2$, for a particular z-axis
(propagation direction) cross-section:
\begin{equation}
P = \left| \langle \psi \mid \psi' \rangle \right|^2
= \left[\int_{-\infty}^\infty \psi^\star(x) \psi'(x) \, dx \right]^2
\end{equation}
We are interested in the lowest order Hermite-Gaussian mode:
\begin{equation}
\psi(x,z) = u_0(x,z) = \left[ \frac{2}{\pi w_0^2} \right]^{1/4} \left[
  \frac{q_0}{q(z)} \right]^{1/2} \exp{\left[ \frac{-i k x^2}{2 q(z)}
  \right] }
\end{equation}
which can be rewritten as a function of $x$ and $q$ using
$q_0=i \mbox{Im}(q)$, and $w_0^2 = -2 q_0 i/k$:
\begin{equation}
u_0(x,q) = \left[ \frac{-k \mbox{Im}(q)}{\pi q^2} \right]^{1/4} \exp{\left[ \frac{-i k x^2}{2 q}
  \right] }
\label{eq:u0_xq}
\end{equation}
Eq. (\ref{eq:u0_xq}) is normalized such that $\langle u_0 \mid u_0
\rangle = 1$.

We want to know the square of the overlap integral for two fields
given by different $q$ parameters at one location $z$. First, the
overlap integral:
\begin{align}
\langle q_1 \mid q_2 \rangle
&= \int_{-\infty}^\infty u_0(x, q_1) u_0(x, q_2) \, dx \\
&= \left[ \frac{k^2 \mbox{Im}(q^\star_1) \mbox{Im}(q_2)}{\pi^2
    q^{\star 2}_1 q^2_2} \right]^{1/4} \int_{-\infty}^\infty
\exp{\left[ -\left[ \frac{-i k}{2 q^\star_1} + \frac{i k}{2 q_2}
  \right]x^2} \right] \, dx \\
&= \left[ -\frac{k^2 \mbox{Im}(q_1) \mbox{Im}(q_2)}{\pi^2
    q^{\star 2}_1 q^2_2} \right]^{1/4} \sqrt{\frac{\pi}{\left[ \frac{-i k}{2 q^\star_1} + \frac{i k}{2 q_2}
  \right]}} \\
&= \left[ \mbox{Im}(q_1) \mbox{Im}(q_2) \right]^{1/4}
\sqrt{\frac{2}{q^\star_1 q_2 \left( 1/q_2 - 1/q^\star_1 \right)}} 
\end{align}
Then, the power is given by:
\begin{align}
\left| \langle q_1 \mid q_2 \rangle \right|^2
&= \frac{2 \sqrt{\mbox{Im}(q_1) \mbox{Im}(q_2)}}  
{\sqrt{q^\star_1 q_1 q^\star_2 q_2 \left[\frac{1}{q_2} - \frac{1}{q^\star_1} \right]
    \left[\frac{1}{q^\star_2} - \frac{1}{q_1} \right]}} \\
&= \frac{2 \sqrt{\mbox{Im}(q_1) \mbox{Im}(q_2)}}{|q_2 - q^\star_1 |} 
\label{eq:power_qs}
\end{align}
Note that Eq. (\ref{eq:power_qs}) simplifies to 1 when $q_1 = q_2$ and
that this whole formulation assumes that the beams are propagating
along the same z-axis.
